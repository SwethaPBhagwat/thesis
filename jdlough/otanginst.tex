
\section{Angular instability}
\label{sec:IV} 
When operated with high intracavity laser power, suspended Fabry-Perot
cavities like the arm cavities of LIGO have a well known angular instability.
It arises from coupling the misalignment of the two cavity mirrors to radiation
pressure torques.
This is known as the Sidles-Sigg instability
%\cite{Sidles06}.
\cite{2006PhLA..354..167S}.
In this section we show that the intrinsic strength of an optical trap for
alignment degrees of freedom is generally bigger, i.e. has a bigger spring
constant than any associated Sidles-Sigg instability. 

We start with a cavity of length $L$, with $x_1,x_2$  being the position of the beam spots on mirrors 1 and 2. $\theta_1,\theta_2$ are the yaw angles of the two mirrors, and $R_1,R_2$ are their radii of curvature. The corresponding g-factors are $g_{1,2}=1-L/R_{1,2}$.
If one or both of the mirrors are slightly misaligned ($\theta_{1,2}\neq 0$), then the radiation pressure force exerts torques $T_1$ and $T_2$ on the two mirrors, given by the following relation (see for instance \cite{2006PhLA..354..167S} or \cite{Ballmer13}): 
\begin{equation}
\label{SidlesSigg_Basic}
\left(
\begin{array}{c}
T_1\\
T_2
\end{array}
\right)
=
\frac{F_0 L}{1-g_1 g_2}
\left(
\begin{array}{cc}
g_2 & -1\\
-1 & g_1
\end{array}
\right)
\left(
\begin{array}{c}
\theta_1\\
\theta_2
\end{array}
\right)
\end{equation} 
with $F_0=P_0\frac{t_1^2}{(1-X)(1-\overline{X})} \frac{2 r_2^2}{c}$ being the intra-cavity radiation pressure force. Sidles and Sigg first pointed out that, since the determinant of the matrix in this equation
%\ref{SidlesSigg_Basic}
 is negative, the two eigenvalues have opposite sign. This always leads to one stable and one unstable coupled alignment degree of freedom.

First we note that for a situation in which one mass is sufficiently heavy that we can neglect any radiation pressure effects on it (i.e. $\theta_1=0$), it is sufficient to choose a negative branch cavity (i.e. $g_1<0$ and $g_2<0$) to stabilize the setup. This is for instance the case for the example setup described in Fig. \ref{fig:angular}.

Next we want to compare the order of magnitude of this effect to the strength of an angular optical spring. If we call $h$ the typical distance of the beam spot from the center of gravity of the mirror, and $x$ the cavity length change at that spot, the order of magnitude of the optical spring torque is:
\begin{eqnarray}
%F=K_{0}\cdot x = T/h\approx \frac{F_0L}{h}\cdot \frac{x}{h}
T\approx \frac{F_0 L}{1-g_1g_2}\cdot \frac{x}{h}
\end{eqnarray}
We can express this as the strength of an optical spring located at position $h$. The corresponding spring constant $K_{SS} \approx T/(h x)$. Thus we can see that
\begin{eqnarray}
\label{eqn:KSS_def}
K_{SS} \approx \frac{F_0}{1-g_1g_2}\cdot \frac{L}{h^2}.
\end{eqnarray}
We now consider the adiabatic optical spring ($\Omega=0$) in equation \ref{eqn:K0}.  Expressed in terms of $F_0$, $K_{OS}$ becomes
\begin{eqnarray}
\label{eqn:KOS_exact}
K_{OS}=i F_0 \frac{X-\overline{X}}{(1-X)(1-\overline{X})}   2 k
\end{eqnarray}
Since we operate near the maxium of the optical spring, the order of magnitude of the resonance term can be estimated as
\begin{eqnarray}
\label{eqn:res_est}
\frac{X-\overline{X}}{(1-X)(1-\overline{X})} \approx \frac{-i}{1-|X|}
\end{eqnarray}
Thus we can estimate the magnitude of  $K_{OS}$ as
\begin{eqnarray}
\label{eqn:K0_order}
K_{OS} \approx F_0 \frac{4\pi}{\lambda}\frac{1}{1-|X|} \approx F_0 \frac{4}{\lambda} \mathcal{F}
\end{eqnarray}
where $\mathcal{F}$ is the cavity finesse.
From equations \ref{eqn:KSS_def} and \ref{eqn:K0_order} we see that the optical spring $K_{OS}$  is much larger than the Sidles-Sigg instability spring $K_{SS}$ if
\begin{eqnarray}
\label{eqn:h2}
h^2 >> \frac{\lambda L}{\pi} \frac{1}{1-g_1 g_2} \frac{\pi}{4 \mathcal{F}}
\end{eqnarray}
Now recall that the beam spot size in a Fabry-Perot cavity is given by \cite{Siegman86}
\begin{equation}
w_1^2 = \frac{\lambda L}{\pi} \sqrt{\frac{g_2}{g_1(1-g_1 g_2)}}
\label{equ:spotsize1}
\end{equation} 
Assuming a symmetric cavity ($g_1=g_2$) for simplicity, we thus find that $K_{OS}$  dominates over $K_{SS}$ if
\begin{eqnarray}
\label{eqn:h2w}
h^2 >> w_{1,2}^2 \frac{1}{\sqrt{1-g_1 g_2}} \frac{\pi}{4 \mathcal{F}}
\end{eqnarray}
This condition is naturally fulfilled since we need to operate the angular optical spring with separate beams ($h>w_{1,2}$) and a large finesse ($\mathcal{F}>>1$). Therefore the angular optical spring is indeed strong enough to stabilize the Sidles-Sigg instability.

