The direct detection of gravitational waves (GWs) promises to usher in a new
era of astronomy.  The \ac{GW} spectrum represents an entirely new window on
the universe, independent of, and complimentary to, \ac{EM} radiation.
Gravitational waves can be used to directly probe objects unobservable by EM
telescopes; e.g., the properties of black holes, the equation of state of
neutron stars, and the state of the universe prior to the emission of the
cosmic microwave background.  Joint GW and EM observations offer more
possibilities, such as understanding the progenitors of short-hard gamma-ray
bursts (GRBs) and measuring the expansion of the universe. The GW spectrum
would also give us insight into the physics of strong field gravity and
numerical solutions of the Einstein equations, as well as provide a test for
alternative theories of gravity \cite{SathyaSchutz:livingReveiw:2009}.

The U.S. Laser Interferometer Gravitational-wave Observatory (LIGO) and the
French-Italian Virgo interferometer are seeking to make the first direct
detections of gravitational waves \cite{Abbott:2007kv}.  To date, \ac{LIGO} has
completed six \emph{Science runs}. The first five of these runs were known as
\emph{initial} \ac{LIGO}. In \ac{S5}, which lasted from November 2005 to
September 2007, the \ac{LIGO} detectors reached their design sensitivity, as
they were sensitive to gravitational waves with strain amplitudes of
$\sim10^{-21}$ in the $40-7000\,$Hz frequency band \cite{Abbott:2007kv}.
\ac{S6}, also known as \emph{enhanced} \ac{LIGO} \cite{Adhikari:2006}, lasted
from July 2009 until October 2010. Hardware improvements were make to the
detectors for \ac{S6}; during this period the \ac{LIGO} detectors met and
exceeded the sensitivity of \ac{S5}. Virgo has had three science runs.
\ac{VSR1} overlapped with \ac{S5}, lasting from May 2007 until Octover 2007
\cite{Acernese:2008zzf}. During this period, joint \ac{LIGO} Virgo searches for
\acp{GW} were carried out \cite{S5VSR1Burst, S5LowMassLV,
Collaboration:2009kk}. \ac{VSR2} and \ac{VSR3} ran from July 2009 until January
2010, and from August 2011 to October 2011, respectively. During these runs the
Virgo detector operated with improved sensitivity over \ac{VSR1}. \ac{LIGO},
having just completed \ac{S6}, is currently preparing for the Advanced
\ac{LIGO} era, which may begin as early as 2014 \cite{Harry:2010zz}. Virgo will
also be upgraded on the same schedule as Advanced \ac{LIGO}
\cite{Acernese:2008zzf}. For the Advanced era, both the \ac{LIGO} and Virgo
detectors are projected to have a factor of ten improvement in sensitivity
\cite{Harry:2010zz, PSD:AV}, which will allow for \ac{GW} detections from
multiple astrophysical sources \cite{Harry:2010zz, ratesdoc}.

During these Science runs, the \ac{LSC} and the Virgo Collaboration have
carried out several searches for \acp{GW} from various astrophysical sources.
These searches are broadly grouped into four categories \cite{Abbott:2007kv}:
\begin{itemize}

\item{{\it The \ac{CBC} group}: These searches look for \acp{GW} from compact
stellar mass binaries as they ``inspiral" into each other and merge
\cite{Belczynski:2002}. The waveform from these searches can be modelled and a
matched-filter analysis used \cite{Allen:2005fk}}.

\item{{\it The Burst group}: This consists of un-modelled transient searches
that look for \acp{GW} from systems that cannot easily be modelled, such as from
core-collapse supernovae \cite{Anderson:2000yy}}.

\item{{\it The Continuous-wave group}: This search looks for \acp{GW} from a
``continuous" wave source, such as a pulsar \cite{Collaboration:S5Pulsar}}.

\item{{\it The Stochastic group}: This consists of searches for a ``stochastic
background" of broadband \acp{GW}, that, for example, could have been emitted
in the early universe \cite{Collaboration:S5stochastic}}.

\end{itemize} In this thesis we focus on the search for \acp{GW} from
\acp{CBC}. Within the \ac{CBC} group several different searches are performed.
Here, we focus on the \emph{all-sky ``low-mass" search}. This search looks for
gravitational waves from coalescing binaries that have a total mass
$\mtotal < 35\,\Msun$ and a component mass $\mathrm{\geq 1M_\odot}$ using LIGO
and Virgo data.\footnote{Throughout the rest of this thesis we will refer to
this search as the \emph{low-mass \ac{CBC} search.}} This search includes
binary neutron stars (BNS), \acp{BBH}, and \ac{NSBH} systems.

Coalescing compact binaries with a total mass $< 35\,\Msun$ have a number of
properties that make them promising candidates for detection by LIGO and Virgo
\cite{LIGOS1iul,LIGOS2iul,LIGOS2macho,
LIGOS2bbh,LIGOS3S4all,Collaboration:2009tt,Abbott:2009qj,S5LowMassLV}.  As the
binaries' components spiral into each other, they emit gravitational waves that
pass through the sensitive band of the LIGO and Virgo detectors.  These waves
can be well-modeled using the post-Newtonian approximation to General
Relativity \cite{Blanchet:1996pi,Droz:1999qx,Blanchet:2002av,
Buonanno:2006ui,Boyle:2007ft,Hannam:2007ik, pan:024014,Boyle:2009dg}.  Knowing
the morphology of the waveform provides a number of advantages, as it allows us
to use match filtering and signal-based vetoes \cite{Brown,Allen:2004}.  The
detectors have stationary, Gaussian noise that is intrinsic to the detector, as
well as non-Gaussian transients (\emph{glitches}) which come from environmental
and instrumental sources.  Match filtering is the optimal method to find weak
signals in the stationary noise since the signal-to-noise ratio (SNR) grows
with the square root of the number of cycles
\cite{SathyaSchutz:livingReveiw:2009}. For non-Gaussian glitches, signal-based
vetoes provide a powerful way to separate noise from candidates
\cite{Allen:2004,Rodriguez:2007}.  Additionally, by utilizing all the detectors
in the LIGO-Virgo network, we can perform ``coincidence tests" to ensure that
triggers from different observatories are consistent \cite{Robinson:2008}.

In this thesis we detail how searches for these ``low mass" \acp{CBC} are
carried out and we give some recent results from this search. In Chapter
\ref{ch:theory} we review the theory behind gravitational waves, showing how
they arise in General Relativity, and review the basics of the \ac{LIGO}/Virgo
interferometers. In Chapter \ref{ch:pipeline_principles} we review some of the
principles of matched filtering, and how various statistical tests are used to
search the data. Next, in Chapter \ref{ch:far}, we detail how false alarm rates
are calculated for \ac{GW} triggers. In Chapter \ref{ch:ihope_pipeline} we step
through the pipeline used to search for \acp{CBC}, which makes use of the
methods presented in Chapters \ref{ch:pipeline_principles} and \ref{ch:far}.
Chapter \ref{ch:s5_results} presents results from the low-mass \ac{CBC} search
for \acp{GW} in $6$ months of \ac{S5} data. Next, in Chapter
\ref{ch:s6_results}, we give a detailed examination of the low-mass \ac{CBC}
analysis of \ac{S6}, including tuning choices made and preliminary results from
that search. In Chapter \ref{ch:ligo_south_study} we present an analysis of the
\ac{S5} and \ac{S6} data that shows that we can detect \acp{GW} at expected
rates in Advanced \ac{LIGO}. Finally, in Chapter \ref{ch:future_developments}
we present some future directions for the low-mass \ac{CBC} pipeline.
