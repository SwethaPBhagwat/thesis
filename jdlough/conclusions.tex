\acresetall

I have shown that the optical trap works for one degree of freedom and
matches well with the theory if we include the effects due to thermal
expansion.
This thermal expansion effect adds an interesting contribution to the transfer
function. Unfortunately, this effect works against us to destabilize the mirror.

This discovery indicates that we will have to balance the power density with
absorption limits for the optics we're using in optical trapping
applications.

\section{Implications for Angular Trap Experiment}
In the case of the angular trap we need to consider the beam size on the
test mass.
If we assume the configuration as presented in section \ref{sec:II} we
need to consider a stable 3 mirror cavity which, topologicaly is a four
mirror ring cavity with the test mass acting as the 2nd and fourth mirrors.
If all of the radius of curvatures are the same, the stablity condition is
basically that of a two mirror cavity.
In any case, the 2nd mirror is at an angle to the resonant beam and will
impart an interference pattern on the surface of the mirror.
So, the power distibution on the mirror will be of the form,
%(\tcb{This should be basically right, but need to be more rigorous with eq.
%and properly compute the integrated effect})
\begin{equation}
p = 2 \sin^2(k_t x) \times e^{-(x+y)^2/w^2 } \;,
\end{equation}
where $k_t$ is the transverse k-vector across the interference pattern.
%This gives a maximum power density twice what we would have without the angle.
Since the integrated power is the same, and there is an interference pattern,
the maximum power density must increase by a factor of 2.
Therefore the integration of $p^2 \, \mathrm{d}A$ increases by a factor of 4.
%so effectively the area over
%which the beam is incident on the mirror is reduced by a factor of 2.
Thus the effect due to thermal expansion increases by a factor of 4.

%This is assuming the return beam has acquired a phase shift of $n\pi$
%when it hits the mirror.
%If we can ensure a phase shift of $(n+1/2)\pi$ for the return beam, the
%overlapping interference patterns will produce a power density profile like
%the case for the two mirror cavity.

To minimize the effects due to thermal expansion, we will particularly need to
pay attention to the beam spot size on the test mass due to the angled beam.
The stability condition for the cavity can be determined by considering the
ABCD propagation matrix for the system. Each round trip will have the following
matrix overall,
\begin{equation}
\mathbf{M} = \mathbf{M}_{R3} \mathbf{M}_L \mathbf{M}_{R2} \mathbf{M}_L
  \mathbf{M}_{R4} \mathbf{M}_L \mathbf{M}_{R2} \mathbf{M}_L \, ,
\end{equation}
where $\mathbf{M}_R$ represents the ABCD matrix for each mirror
\begin{equation}
\mathbf{M}_R =
\begin{pmatrix}
1 & 0 \\
\frac{2g-2}{L} & 1
\end{pmatrix} \,,
\end{equation}
and
$\mathbf{M}_L$ represents
the ABCD matrix for the propagation through length $L$.
The eigenvalue equation which needs to be solved is then,
\begin{equation}
\lambda^2-\mathrm{Tr}(\mathbf{M})\lambda+\mathrm{Det}(\mathbf{M}) = 0.
\end{equation}

It turns out that the solution has the form,
\begin{equation}
\lambda = \alpha \pm i \sqrt{1-\alpha^2}
\end{equation}
with,
\begin{equation}
\alpha = 1-4g_3g_2-4g_4g_2+8g_3g_2^2g_4 \, .
\end{equation}
For $|\lambda|^2 = 1$,
\begin{equation}
\lambda = e^{ i\phi}
\end{equation}
where,
\begin{equation}
\alpha = \cos(\phi)
\end{equation}

We will still want the angular stability condition for beam A of
$g_1 < 0$ and $g_2 < 0$ because the experiment we have outlined only stabilizes
the yaw angular degree of freedom.
The pitch degree of freedom still needs to be stable on it's own.

This results in the following condition (for $g_2 < 0$ in order to preserve
stability of linear trap, where $0 < g_1g_2 < 1$, and $g_1 < 0$),
\begin{equation}
0 \leq \left( 1-2g_3g_2\right)\left(1-2g_2g_4\right) \leq 1
\end{equation}
The corresponding resonant mode of the cavity will have
$R_\mathrm{beam} = R_\mathrm{mirror}$ at mirrors 3 and 4.

If we make $R_4$ infinite, i.e. flat mirror, we can have a stable resonator
with $g_3 \leq \frac{2}{g_2 - 2}$.
Since we keep $g_2$ bounded by $-1 < g_2 < 0$, it is sufficient to require
$g_3 < -1$.

For beam A we can increase the cavity length to about 9cm while still having
the stability requirement for $g_1g_2<1$ and the spot size will be a bit
larger.
We can also go to a larger radius of curvature and even longer cavity, with
the consequence of going to a more massive test mass.
Parameters for a few configurations are provided in table \ref{tab:angleparams}.
%\tcb{in a table that I will
%produce and reference here}.

%\begin{table}
%\tiny
%\begin{center}
%\begin{tabular}{|l|l|l|l|l|l|l|l|l|l|l|l|l|l|l|}
%\hline
%\multicolumn{6}{|c|}{Beam A} & \multicolumn{8}{c|}{Beam B} & TM \\
%\hline
%$L_A$ & $R_1$ & $R_2$ & $w_0$ & $w_1$ & $w_2$ & $L_B$ & $R_3$ & $R_4$ &
%  $w_0$ & $w_0$ & $w_3$ & $w_2$ & $w_4$ & mass \\
%\hline
%\hline
%7cm & 5cm & 5cm & $88\mu \mathrm{m}$ & $161\mu \mathrm{m}$ &
%  $161\mu \mathrm{m}$ & 7cm & 5cm & 5cm & $88\mu \mathrm{m}$ & $88\mu \mathrm{m}$ &
%  $161\mu \mathrm{m}$ & $161\mu \mathrm{m}$ & $161\mu \mathrm{m}$ & $0.41\mathrm{g}$ \\
%\hline
%\end{tabular}
%\end{center}
%\caption[Angular Trap Parameters]{This table provides different possible
%    angular trap configurations we can employ using existing optics.}
%\end{table}

\newcommand{\um}{\mathrm{m}}
\newcommand{\uhz}{\mathrm{Hz}}

\begin{table}
\scriptsize
\begin{center}
\begin{tabular}{|l|l|l|l|l|}
\hline
parameter & configuration 1 & configuration 2 & configuration 3 & configuration 4 \\
\hline
\hline
Beam $A$ & \cellcolor[gray]{0.8}   & \cellcolor[gray]{0.8} &
    \cellcolor[gray]{0.8} & \cellcolor[gray]{0.8} \\
\hline
$L_A[\mathrm{cm}]$    &  7     & 11          & 11      & 12 \\
$R_1[\mathrm{cm}]$    &  5     & 7.5         & 7.5     & 7.5 \\
$R_2[\mathrm{cm}]$    &  5     & 7.5         & 7.5     & 7.5 \\
$w_0[\mu\um]$         &  88    & 106         & 106     & 101 \\
$w_1[\mu\um]$         &  161   & 205         & 205     & 225 \\
$w_2[\mu\um]$         &  161   & 205         & 205     & 225 \\
$L_\mathrm{roundtrip}$ & 14cm  & 22cm        & 22cm     & 24cm \\
$\mathcal{FSR}$ & 2.141 GHz    & 1.363 GHz   & 1.363 GHz& 1.249 GHz \\
$\mathcal{F}$ & 8650           & 8650        & 8650     & 8650 \\
$\gamma$ & 778 kHz             & 495 kHz     & 495 kHz  & 454 kHz \\
\hline
Beam $B$ & \cellcolor[gray]{0.8}   & \cellcolor[gray]{0.8} &
    \cellcolor[gray]{0.8} & \cellcolor[gray]{0.8} \\
\hline
$L_{B1}[\mathrm{cm}]$ &  7        & 12        & 14     & 13.5 \\
$L_{B2}[\mathrm{cm}]$ &  7        & 13        & 14     & 14.5 \\
$R_3[\mathrm{cm}]$    &  5        & 5         & 7.5    & 7.5 \\
$R_4[\mathrm{cm}]$    &  5        & 5         & 7.5    & 7.5 \\
$w_0[\mu\um]$         &  88       & 31  & 80  & 71 \\
$w_0[\mu\um]$         &  88       & 35  & 80  & 111 \\
$w_3[\mu\um]$         &  161      & 553 & 308 & 351 \\
$w_2[\mu\um]$         &  161      & 778 & 308 & 309 \\
$w_4[\mu\um]$         &  161      & 483 & 308 & 231 \\
$L_\mathrm{roundtrip}[\mathrm{cm}]$ & 28& 50        & 56     & 56 \\
$\mathcal{FSR}$ & 1.071 GHz  & 600 MHz     & 535 MHz  & 535 MHz \\
$\mathcal{F}$ & 4960         & 4960        & 4960     & 4960 \\
$\gamma$ & 678 kHz           & 380 kHz     & 339 kHz  & 339 kHz \\
\hline
TM mass  & $0.415\mathrm{g}$ & $1.17\mathrm{g}$ & $1.17\mathrm{g}$ & $1.17\mathrm{g}$ \\
\hline
\end{tabular}
\end{center}
\caption[Angular Trap Parameters']{This table provides different possible
    angular trap configurations we can employ using existing optics.
    Beam A forms the two-mirror cavity as in the linear trap.
    Beam B forms the V-shaped offset cavity for providing angular stability
    as seen in figure \ref{fig:angular}.
    $L_A$ corresponds to the length of the Beam A cavity.
    $R_1$ and $R_2$ are the radius of curvature for the input and output mirrors
    respectively for the Beam A cavity.
    $R_3$ and $R_4$ are for the input and end mirrors respectively for the Beam B
    cavity.
    The mirror at the vertex of the V-shaped cavity is the output mirror for
    Beam A.
    The beam sizes are denoted with $w$. $w_0$ for Beam A is the waist size.
    The first beam waist under Beam B is the waist of the beam
    in the length $L_{B1}$.
    The beam sizes $w_n$, with $n>0$ are the spot sizes on the mirrors
    corresponding to $R_n$.
    The second beam waist is for the beam in length $L_{B2}$.
    $L_\mathrm{roundtrip}$ is the roundtrip length of each cavity which is used
    to compute the free spectral range ($\mathcal{FSR}$).
    The $\mathcal{FSR}$ with the finesse ($\mathcal{F}$) are used to compute the
    linewidth $\gamma$.
    }
\label{tab:angleparams}
\end{table}



