One of the earliest predictions of the Theory of General Relativity was the
existence of gravitational waves. By writing the metric $g_{\mu\nu}$ as the sum
of the flat Minkowski metric $\eta_{\mu\nu}$ and a small perturbation $h_{\mu\nu}$, 
\begin{equation}
g_{\mu\nu} = \eta_{\mu\nu} + h_{\mu\nu},
\end{equation}
and considering bodies with negligible self-gravity, 
Einstein showed\cite{Einstein:1916}
\begin{quotation}
``that these $h_{\mu\nu}$ can be calculated in a manner analogous to that of
the retarded potentials of electrodynamics.''
\end{quotation}
It follows that
gravitational fields propagate at the speed of light.  In electrodynamics, the
lowest multipole moment that produces radiation is the electric dipole; there
is no electric monopole radiation due to the conservation of electric charge.
Similarly in General Relativity, the lowest multipole that produces
gravitational waves is the quadrupole moment. Radiation from the 
mass monopole, mass dipole and momentum dipole vanish due to conservation of
mass, momentum and angular momentum respectively. Einstein also derived the
\emph{quadrupole formula} for the gravitational wave field, which states that
the spacetime perturbation is proportional to the second time derivative of
the quadrupole moment of the source.  The strength of the gravitational waves
decreases as the inverse of the distance to the source.  We can estimate
this strength at a distance $r$ by noticing that the quadrupole moment
involves terms of dimension mass $\times$ length$^2$ and so the second time
derivative of the quadrupole moment is proportional to the kinetic energy of
the source associated with non-spherical motion $E^\mathrm{ns}_\mathrm{kin}$.
Using the quadrupole formula, which we will see in equation
(\ref{eq:quadrupole}), we then approximate the strength of the gravitational
waves as
\begin{equation}
% there is a factor of 4.  d^2 I / dt^2 ~ 4 E^{ns}_{kin}.   I don't
% know if you should worry about it,  but just wanted to mention it.
h \sim \frac{G}{c^4}\frac{E^\mathrm{ns}_\mathrm{kin}}{r}.
\label{eq:strainest}
\end{equation}
The effect of a
gravitational wave is to cause the measured distance $L$ between two freely
falling bodies to change by a distance $\Delta L \sim h L$. 

Interferometers were suggested as a way of measuring the change in 
length between two test masses by Pirani in 1956\cite{Pirani:1956} and
the first working detector was constructed by Forward in
1971\cite{Forward:1971}. The fundamental designs of modern laser
interferometers were developed by Weiss\cite{Weiss:1972} and
Drever\cite{Drever:1980} in the 1970s. The principle upon which
interferometric detectors operate is to use laser light to measure the change
in distance between two mirrors as a gravitational wave passes through the
detector.  The sensitivity of an interferometer on the Earth is limited by
\emph{gravity gradient noise} at frequencies below $\sim
5$~Hz\cite{Saulson:1994}.  Any time changing distribution of matter near the
detector, for example compression waves in the Earth, cause fluctuations in
the local gravitational field. These fluctuations will cause the test masses
to move producing a spurious response in the interferometer which masks the
presence of gravitational waves. In fact, Earth based interferometers are
typically limited in sensitivity to frequencies above $\sim 10$~Hz due to the
seismic motion of the earth.

%Consider a generator of gravitational waves consisting of a pair of $10$~kg
%balls attached by a light rod. If we spin this dumbbell about its center of
%mass so the balls have a speed of $10$~ms$^{-1}$,  the strength of the
%gravitational waves $10$~m from this generator\footnote{We neglect that fact
%that at a distance of $10$~m we would not be in the wave zone of the
%generator. The purpose of this discussion is to illustrate how weakly
%gravitational waves are coupled to matter; moving into the wave zone would
%simply further decrease the strength of the waves.} would be $h \sim
%10^{-42}$.  If we tried to measure the change in distance produced by our
%laboratory generator of gravitational waves between a pair of masses
%separated by $1$~m, we would be attempting to measure a change in distance
%$\Delta L \sim 10^{-42}$~m---smaller than the Planck length.  We therefore
%turn our attention to astrophysical sources of gravitational waves.

The canonical example of an astrophysical source of gravitational waves is the
Hulse-Taylor binary pulsar, PSR~$1913+16$\cite{1975ApJ...195L..51H}. This
system is composed of two neutron stars, each of mass $\sim 1.4\,M_\odot$,
with average separation and orbital velocity of $\sim10^9$~m and $\sim
10^5$ms$^{-1}$, respectively. The period of the orbit is $7.75$~hours and the
binary is at a distance from the earth of $\sim 6$~kpc.  Hulse and Taylor
observed that the orbital period of the binary is decreasing and that the rate
of orbital energy loss agrees with the expected loss of energy due to the
radiation of gravitational waves to within
$0.3\%$\cite{Taylor:1982,Taylor:1989}.   Since the quadrupole moment
of an equal mass binary is periodic at half the orbital period, we would
expect the frequency of the gravitational waves emitted to be twice
the orbital frequency.  Thus, the gravitational waves from PSR~$1913+16$ have
a frequency $f_\mathrm{GW} \sim 10^{-4}$~Hz that is outside the
sensitive band of earth based detectors.   Nevertheless,  the orbit
will continue to tighten by gravitational wave emission, and the two
neutron stars are expected to merge in about 300 million years;   in the last
several minutes prior to merger,  the gravitational wave frequency will sweep
upward from $\sim 10$ Hz reaching about 1500 Hz just before the merger.

It is worthwhile to estimate the strength of the gravitational waves from
a neutron star binary since it informs the target sensitivity for
modern interferometric detectors.   The non-spherical kinetic energy
of this system is 
\begin{equation}
E_\mathrm{kin}^\mathrm{ns} \sim 
1.4\,\mathrm{M}_\odot (\pi a / T)^2
\label{eq:ensht}
\end{equation}
where $T$ is the binary period and $a$ is the average separation.  The period,
separation and mass of a binary are related by Kepler's third law,
\begin{equation}
T^2 = \frac{4\pi^2}{GM}a^3,
\label{eq:kep3}
\end{equation}
where $M$ is the total mass of the binary.  Using equation
(\ref{eq:strainest}), Kepler's third law and the non-spherical kinetic energy
given in equation (\ref{eq:ensht}), we can estimate the strength of the waves
from a neutron star binary as
\begin{equation}
h \sim 10^{-20} \times 
\left(\frac{6.3\,\mathrm{kpc}}{r}\right)
\left(\frac{M}{2.8M_\odot}\right)^{5/3}
\left(\frac{T}{1\,\mathrm{s}}\right)^{-\frac{2}{3}}.
\label{eq:binaryest}
\end{equation}
%The orbital velocity of the stars is
%approximately $10^9 / 10^4 \sim 10^5$~ms$^{-1}$, hence non-spherical kinetic
%energy is approximately $E^\mathrm{ns}_\mathrm{kin} \sim 10^{40}$~J and so 
When the orbital separation is $a \sim 10^5$~m,  the orbital period will be
$T\sim10^{-2}$~seconds and the gravitational wave strain will be $h \sim 5
\times 10^{-19}$.

To date, four binary neutron star systems that will merge within a Hubble time
have been discovered.   By considering the time to merger, position and
efficiency of detecting such binary pulsar systems, the galactic merger rate
for inspirals can be estimated\cite{Phinney:1991ei}.  The latest estimates of
neutron star inspirals in the Milky Way are $8.3 \times 10^{-6}$~yr$^{-1}$. 
Extrapolating this rate to the neighboring Universe using the
blue-light luminosity gives an (optimistic) estimate of the rate at
$0.3$~yr$^{-1}$ within a distance of $\sim 20$~Mpc.  To measure the waves from
a neutron star binary at this distance,  we must construct
interferometers that are sensitive to gravitational waves of strength $h \sim
10^{-22}$. An overview of the theory and experimental techniques underlying
the generation and detection of gravitational waves from binary inspiral is
presented in chapter \ref{ch:inspiral}.

A world-wide network of gravitational wave interferometers has been
constructed that have the sensitivity necessary to detect the gravitational
waves from astrophysical sources. Among these is the Laser Interferometric
Gravitational Wave Observatory (LIGO)\cite{Barish:1999}. LIGO has completed
three science data taking runs. The first, referred to as S1, lasted for 17
days between August 23 and September 9, 2002; the second, S2, lasted for 59
days between February 14 and April 14, 2003; the third, S3, lasted for 70 days
between October 31, 2003 and January 9, 2004.  During the runs, all three LIGO
detectors were operated: two detectors at the LIGO Hanford observatory (LHO)
and one at the LIGO Livingston observatory (LLO).  The detectors are not yet
at their design sensitivity, but the detector sensitivity and amount of
usable data has improved between each data taking run. The noise level is low
enough that searches for coalescing compact neutron stars are worthwhile, and
since the start of S2, these searches are sensitive to extra-galactic sources.
Using the techniques of \emph{matched filtering} described in chapter
\ref{ch:findchirp} of this dissertation, the S1 binary neutron star search set
an upper limit of
\begin{equation}
\mathcal{R}_{90\%} < 1.7 \times 10^2 \;\textrm{per year per Milky Way Equivalent Galaxy (MWEG)}
\end{equation}
with no gravitational wave signals detected. Details of this analysis can be
found in \cite{LIGOS1iul}. 

In this dissertation, we are concerned with the
search for gravitational waves from a different class of compact binary
inspiral: those from binary black holes in the galactic halo. Observations of
the gravitational microlensing of stars in the Large Magellanic cloud suggest
that $\sim 20\%$ of the galactic halo consists of objects of mass $\sim
0.5\,M_\odot$ of unknown origin. In chapter \ref{ch:macho} we discuss a
proposal that these Massive Astrophysical Compact Halo Objects (MACHOs) may be
black holes formed in the early universe and that some fraction of them may be
in binaries whose inspiral is detectable by LIGO\cite{Nakamura:1997sm}.  The
upper bound on the rate of such binary black hole MACHO inspirals are
projected to be $R \sim 0.1$~yr$^{-1}$ for initial LIGO, much higher than the
binary neutron star rates discussed above. It should be noted however, that
while binary neutron stars have been observed, there is no direct
observational evidence of the existence of binary black hole MACHOs. Despite
this, the large projected rates make them a tempting source for LIGO. In
chapter \ref{ch:pipeline} we describe an \emph{analysis pipeline} that has
been used to search the LIGO S2 data for binary black hole MACHOs\footnote{The
same pipeline has also been used to search for binary neutron star inspiral in
the S2 data and the results of this search will be presented in
\cite{LIGOS2iul}.}. Chapter \ref{ch:hardware} describes how the search
techniques were tested on data from the gravitational wave interferometers.
Finally we present the result of the S2 binary black hole MACHO search in
chapter \ref{ch:result}. 
