The Laser Interferometric Gravitational Wave Observatory
(LIGO)\cite{Barish:1999} has completed three ``science runs'' during which all
three interferometers were collecting data simultaneously under stable
operation. Analysis of the data for gravitational waves from coalescing
compact binaries has been completed for the first two
runs\cite{LIGOS1iul,LIGOS2iul} and is in progress for the third run. Two of
the target populations in these searches are binary neutron
stars\cite{thorne.k:1987} and binary black hole
MACHOs\cite{Finn:1996dd,Nakamura:1997sm}. For these low mass systems, which
have component masses below $3 M_\odot$, the waveforms of the gravitational
radiation emitted are well known\cite{Blanchet:1995ez,Blanchet:1996pi}.
Matched filtering is a common and effective technique for extracting known
signals from noise\cite{wainstein:1962}. We have implemented a matched filter
to extract inspiral signals from interferometer noise in the package
\emph{findchirp} which can be found in the LIGO Algorithm Library
(LAL)\cite{LAL}.


\section{Conventions}
There are two possible sign conventions for the Fourier transform of a time
domain quantity $v(t)$. In this thesis, we define the Fourier transform
$\tilde{v}(f)$ of a $v(t)$ to be
\begin{equation}
\label{eq:ft}
\tilde{v}(f)=\int_{-\infty}^\infty dt\,v(t)\, e^{- 2 \pi i f t}
\end{equation}
and the inverse Fourier transform to be 
\begin{equation}
\label{eq:ift}
v(t)=\int_{-\infty}^\infty df\,\tilde{v}(f)\, e^{2 \pi i f t}.
\end{equation}
This convention differs from that used in some gravitational wave literature,
but is the adopted convention in the LIGO Scientific Collaboration.


