One of the earliest predictions of the Theory of General Relativity was the
existence of gravitational waves. By considering a small perturbation
$h_{\mu\nu}$ to the metric of empty spacetime
\begin{equation}
g_{\mu\nu} = \eta_{\mu\nu} + h_{\mu\nu}
\end{equation}
and bodies with negligible self-gravity, Einstein showed that the quantities
$h_{\mu\nu}$ can be calculated in a manner analogous to the retarded
potentials of electrodynamics\cite{Einstein:1916}.  It follows from this that
gravitational fields propagate at the speed of light.  In electrodynamics, the
lowest multipole moment that produces radiation is the electric dipole; there
is no electric monopole radiation due to the conservation of electric charge.
Similarly in gravitational radiation, the lowest multipole that produces
gravitational waves is the quadrupole moment. Radiation from the lower order
mass monopole, mass dipole and momentum dipole vanish due to conservation of
mass, momentum and angular momentum respectively. Einstein also derived the
\emph{quadrupole formula} for the gravitational wave field, which states that
the spacetime perturbation is proportional to the second time derivative of
the quadrupole moment of the source.  The strength of the gravitational waves
decrease as the inverse of the distance to the source.  We can obtain a order
of magnitude estimate of this strength at a distance $r$ by noticing that
the quadrupole moment involves terms of dimension mass $\times$ length$^2$ and
so the second time derivative of the quadrupole moment will be the kinetic
energy of the source associated with non-spherical motion.  We may use
dimensional analysis to approximate the strength of gravitational waves as
\begin{equation}
h \sim \frac{G}{c^4}\frac{E^\mathrm{ns}_\mathrm{kin}}{r}
\end{equation}
where $E^\mathrm{ns}_\mathrm{kin}$ is the non-spherical kinetic energy of
the source.

If we were to build a generator of gravitational waves consisting of a pair of
$10$~kg balls moving at a speed of $10$~ms$^-1$ the strength of the
gravitational waves $10$~m from this detector would be $h \sim 10^{-42}$.  We
will see in chapter~\ref{ch:inspiral} that the effect of a gravitational wave
is to cause the measured distance $L$ between two freely falling bodies to
change by a distance $\Delta L \sim h L$. Therefore if we were to try and
measure the change in distance produced by our laboratory generator of
gravitational waves between a pair of masses separated by $10$~m, we would be
attempting to measure a change in distance of order $\Delta L \sim
10^{-41}$~m---smaller than the Planck length!  We must therefore turn our
attention to astrophysical sources of gravitational waves, such as the
Hulse-Taylor binary pulsar, PSR~$1913+16$\cite{1975ApJ...195L..51H}. This
system is composed of two neutron stars, each of mass $\sim 1.4\,M_\odot$,
at a distance from the earth of $\sim 10^{20}$~m. Hulse and Taylor observed
that the orbital period of the binary is decreasing and that the rate of
orbital energy loss agrees with the expected loss of energy due to the
radiation of gravitational waves to within
$0.3\%$\cite{Taylor:1982,Taylor:1989}. Although PSR~$1913+16$ is a highly
eccentric binary, with a periastron separation of $7\times10^{8}$~m and an
apastron separation of $3\times10^{9}$~m, we may approximate it by a circular
binary with a separation of $10^{9}$~m. We may estimate the period $T$ of the
binary using Kepler's third law
\begin{equation}
T^2 = \frac{4\pi^2}{GM}a^3
\end{equation}
where $M$ is the total mass of the binary and $a$ is the separation of the
stars, so $T\sim 10^4$~seconds. The orbital velocity of the stars is
approximately $10^9 / 10^4 \sim 10^5$~ms$^-1$, hence non-spherical kinetic
energy is approximately $E^\mathrm{kin}_\mathrm{kin} \sim 10^{40}$ and so the
strength of the gravitational waves arriving at the Earth is $h \sim
10^{-25}$. Since an equal mass binary looks identical to an observer twice
every orbital period, we would expect the frequency of the gravitational waves
emitted to be twice the orbital frequency. This implies that the gravitational
waves from PSR~$1913+16$ have a frequency of $f_\mathrm{GW} \sim 10^{-4}$~Hz.
The change in length produced by these gravitational waves, over a distance of
$\sim 10^3$~m will be $\Delta L \sim 10^{-22}$~m.

Laser interferometers were suggested as a way of measuring such small
perturbations between the length of two test masses by Pirani in
1956\cite{Pirani:1956} and the first working detector was constructed by
Forward in 1971\cite{Forward:1971}. The fundamental design of modern
interferometers were developed by Weiss\cite{Weiss:1972} and
Drever\cite{Drever:1980} in the 1970s. The principle upon which
interferometric detectors operate is to use laser light to measure the change
in distance between two mirrors, separated by $\sim 10^{3}$~m, as a
gravitational wave passes through the detector. Can such a detector measure
the gravitational waves emitted by the Hulse-Taylor pulsar? Unfortunately not
as the sensitivity of an interferometer at frequencies below $\sim 5$~Hz is
limited by \emph{gravity gradient noise.} Any time changing distribution of
matter near the detector, for example compression waves in the Earth or
passing cars, will cause fluctuations in the local gravitational field. These
fluctuations will cause the test masses to move producing a spurious response
in the interferometer which masks the presence of gravitational waves. In
fact Earth based interferometers are limited in sensitivity to frequencies
above $\sim 40$~Hz due to the seismic motion of the earth.

Due to the loss of orbital energy to gravitational radiation the orbital
period the orbital period of PSR~$1913+16$ will eventually reach a value of
$T~10^{-2}$~seconds and it will enter the sensitive band of a laser
interferometric gravitational wave detector. At this orbital period, the
separation of the neutron stars will be $a \sim 10^5$~m and their orbital velocities
$v \sim 10^7$~ms$^{-1}$. Using the same order of magnitude estimates above, we
can estimate that the strength of the gravitational waves at this frequency
will be $h \sim 10^{-21}$. To date only four binary neutron stars which will
merge within a Hubble time have been discovered.  By considering the
lifetimes, position and efficiency of detecting such binary pulsar systems,
the galactic merger rate for inspirals can be determined\cite{Phinney:1991ei}.
The latest estimates of neutron star inspirals in the Milky Way are $8.3
\times 1^{-6}$~yr$^-1$. The goal for the first generation of interferometers
are detection rates of $0.3$~yr$^{-1}$ which means that the detectors must be
sensitive to binary inspiral at a distance of $20$~Mpc. To achieve this, we
must construct interferometers that are sensitive to gravitational waves of
strength $h \sim 10^{-23}$. An overview of the theory and experimental
techniques underlying the generation and detection of gravitational waves from
binary inspiral is presented in chapter \ref{ch:inspiral}.

A world-wide network of gravitational wave interferometers has been
constructed that have to necessary sensitivity to detect the gravitational
waves from astrophysical sources. Among these is the Laser Interferometric
Gravitational Wave Observatory (LIGO)\cite{Barish:1999}. LIGO has completed
three science data taking runs. The first, referred to as S1, lasted for 17
days between August 23 and September 9, 2002; the second, S2, lasted for 59
days between February 14 and April 14, 2003; the third, S3, lasted for 70 days
between October 31, 2003 and January 9, 2004.  During the runs, all three LIGO
detectors were operated: two detectors at the LIGO Hanford observatory (LHO)
and one at the LIGO Livingston observatory (LLO).  The detectors are not yet
at their design sensitivity, however the detector sensitivity and amount of
usable data has improved between each data taking run. The noise level is low
enough that searches for coalescing compact neutron stars are worthwhile and
since the start of S2, these searches are sensitive to extra-galactic sources.
Using the techniques of \emph{matched filtering} described in chapter
\ref{ch:findchirp} of this dissertation, the S1 binary neutron star search set
an upper limit of
\begin{equation}
\mathrm{R}_{90\%} < 1.7 \times 10^2 \textrm{per year per Milky Way Equivalent Galaxy (MWEG)}
\end{equation}
with no gravitational wave signals detected. Details of this analysis can be
found in \cite{LIGOS1iul}. In this dissertation, we are concerned with the
search for the gravitational waves from a different class of compact binary
inspiral; those from binary black holes in the galactic halo. Observations of
the gravitational microlensing of stars in the Large Magellanic cloud suggest
that $\sim 20\%$ of the galactic halo is comprised of objects of mass $\sim
0.5\,M_\odot$ of unknown origin. In chapter \ref{ch:macho} we discuss a
proposal that these Massive Astrophysical Compact Halo Objects (MACHOs) may be
black holes formed in the early universe and that some fraction of them may be
in binaries whose inspiral is detectable by LIGO\cite{Nakamura:1997sm}.  The
upper bounds on the rate of such binary black holes MACHO inspirals are
projected to be $R \sim 0.1$~yr$^{-1}$, much higher than the binary neutron
star rates discussed above. It should be noted however, that while binary
neutron stars have been observed, there is no direct observational evidence of
the existence of binary black hole MACHOs. Despite this, the large projected
rates make them a tempting source for LIGO. In chapter \ref{ch:pipeline} we
describe an \emph{analysis pipeline} that has been used to search the LIGO S2
data for binary black hole MACHOs\footnote{The same pipeline has also been
used to search for binary neutron star inspiral in the S2 data and the results
of this search will be presented in \cite{LIGOS2iul}.}. Chapter
\ref{ch:hardware} describes how the search techniques were tested on data from
the gravitational wave interferometers. Finally we present the result of the
S2 binary black hole MACHO search in chapter \ref{ch:result}. 

\section{Conventions}
There are two possible sign conventions for the Fourier transform of a time
domain quantity $v(t)$. In this thesis, we define the Fourier transform
$\tilde{v}(f)$ of a $v(t)$ to be
\begin{equation}
\label{eq:ft}
\tilde{v}(f)=\int_{-\infty}^\infty dt\,v(t)\, e^{- 2 \pi i f t}
\end{equation}
and the inverse Fourier transform to be 
\begin{equation}
\label{eq:ift}
v(t)=\int_{-\infty}^\infty df\,\tilde{v}(f)\, e^{2 \pi i f t}.
\end{equation}
This convention differs from that used in some gravitational wave literature,
but is the adopted convention in the LIGO Scientific Collaboration.


