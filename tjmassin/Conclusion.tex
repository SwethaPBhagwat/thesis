Advanced LIGO's first observing run was highly successful, resulting in the first direct detection of 
gravitational waves and further tests of Einstein's Theory of General Relativity. Gravitational waves 
from two binary black hole mergers, GW150914 and GW151226, were measured at the LIGO interferometers 
and recovered from the data using matched filter search algorithms. 

Searching for gravitational waves requires an understanding of instrumental
features and artifacts that can adversely affect the output of a gravitational wave
search pipeline. Throughout the observing run, data quality vetoes were produced
to ensure that the analysis pipelines analyzed clean data \cite{GW150914-DETCHAR}.

Data quality vetoes improved the sensitivity of the PyCBC search in Advanced LIGO's first
observing run. Although the BNS bins were not dramatically affected, the distribution
of background events was notably improved in the bulk and edge bins.
In both bins, a significant tail of loud background triggers appeared
if noisy data were not removed from the search.

In all 3 bins, it is evident that CAT1 vetoes had a more significant impact on false alarm rates
than CAT2 vetoes, often providing 2-3 orders of magnitude of improvement in false alarm rate in the bulk
and edge bins. This is expected, given that CAT1 vetoes are used to remove the most egregious
data from the analysis. CAT2 vetoes had the greatest impact in the bulk and edge bins from the
analysis containing GW151226, providing at least one order of magnitude reduction in false alarm
rate compared to analyses using CAT1 vetoes only. This is due to a particularly effective CAT2 flag that
was implemented in the
analysis containing GW151226, but was not relevant during the analysis containing GW150914.

The black hole binary system GW150914 was a strong enough signal that it was louder than all background
events regardless of what data were removed from the search. As such, data quality vetoes did not
improve its significance.
The significance of LVT151012 was improved when data with excess noise were removed. Its
false alarm rate was improved from 3.09 $\mathrm{yr}^{-1}$ to 0.44 $\mathrm{yr}^{-1}$ when data quality vetoes were applied
to the PyCBC search.

The significance of the second binary black hole system discovered in O1, GW151226, was significantly
increased by the application of data quality vetoes. The false alarm rate of GW151226 decreases by a
factor of 567 when data quality vetoes are applied, which results in a clear detection ($>$5$\sigma$) from a
marginal detection candidate (3.9$\sigma$).
