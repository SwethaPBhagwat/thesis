%%%%%%%%%%%%%%%%%%%%%%%%%%%%%%%%%%%%%%%%%%%%%%%%%%%%%%%%%%%%%%%%%%%%%%%%%%%%%%%
%%% Describe the wondrous conclusions of the thesis.
%%%%%%%%%%%%%%%%%%%%%%%%%%%%%%%%%%%%%%%%%%%%%%%%%%%%%%%%%%%%%%%%%%%%%%%%%%%%%%%


The first observation runs of Advanced LIGO and Advanced Virgo detectors 
are scheduled for $2015$. By $2018$, these detectors will reach 
their design sensitivity. These second-generation terrestrial detectors
will be able to see up to $10$ times further out in the universe 
than their earlier counterparts. For a compact binary population
uniformly distributed in co-moving volume, this translates to 
a thousandfold increase in the expected detection rate.
% 
Gravitational wave searches make use of theoretical knowledge of
binary dynamics and employ modeled waveforms as filter templates.
With the increase in sensitivity, the resolution of the detectors 
for small errors in modeled waveforms also increases. In this dissertation,
we primarily focus on selecting and developing optimal waveform filters
for Advanced LIGO searches. We also validate gravitational-wave 
search algorithms using accurate numerically simulated signals injected 
into emulated detector noise.

Past binary black hole searches have used post-Newtonian (pN) and 
Effective-One-Body (EOB) waveforms as filters. While the pN waveforms are 
computationally inexpensive, they are restricted to the inspiral
regime of binary coalescence. EOB waveforms include the complete
coalescence process through inspiral, merger and ringdown, and also
the sub-dominant waveform harmonics. However, they are also 
computationally more expensive. For low mass binary black holes 
($m_1,m_2\leq 25M_\odot$),
we explore the region of the parameter space over which pN waveform
templates are sufficiently accurate, in the sense of being able to 
recover more than $97\%$ of the optimal signal-to-noise ratio, 
and where in the parameter space would searches need EOB
waveform templates.
% 
% For binaries with masses $m_1,m_2\leq 25M_\odot$, we compare the 
% inspiral-only post-Newtonian waveforms with the recently proposed
% Effective-One-Body (EOB) model~\cite{BuonannoEOBv2Main}. 
% As this EOB model is calibrated
% against high-accuracy numerical simulations of non-spinning binary 
% black holes, it is demonstrably accurate for {\it comparable}
% mass-ratio binaries. However, it is computationally more expensive
% than the post-Newtonian approximants. 
% We investigate the region of the parameter
% space of non-spinnning binaries where the accuracy of post-Newtonian
% approximants is sufficient and we can win with computational cost, as
% well as the region where EOB waveforms would be required. 
Here we approximate the waveforms with their dominant multipoles. Next,
we study the impact of ignoring sub-dominant waveform multipoles in 
searches. We find that including sub-dominant harmonics could increase
the reach of aLIGO and Virgo for binaries which have their orbital
angular momentum highly inclined to the line of sight connecting them
to the detector.

Numerical Relativity (NR) has seen recent breakthroughs and rapid progress
in simulating the merger of orbiting black holes. These are the most 
accurate solutions to Einstein's field equations available. Still, 
due to their computational cost, numerical relativity simulations 
span only the last stages of the binary inspiral, alongwith the merger
and ringdown. It is possible to join these short but accurate strong-field
simulations with post-Newtonian waveforms that cover the slow-motion
regime, to construct pN-NR {\it hybrids}. We demonstrate that, within 
the limits of current NR technology, it is possible and viable to use 
hybrid waveforms in gravitational wave searches. In addition, we show
that hybrid waveforms can cover the entire region of the binary black
hole parameter space where pN waveforms are insufficient for Advanced 
LIGO searches.


Apart from having applications as search templates, and in enhancing the
accuracy of waveform models, NR simulations
can be used to validate gravitational-wave search algorithms.
We do precisely this within the purview of the NINJA-2 project. 
Several numerical relativity groups contributed
post-Newtonian-hybridized simulations to the project. These were subsequently
injected in emulated advanced detector noise. We demonstrate the ability of
existing search algorithms to successfully {\it detect} these simulations
embedded within realistic noise. This is different from the NINJA-1 project
on a few counts, one of them being the nature of the emulated noise. In the 
NINJA-2 project, initial LIGO data with its non-Gaussian transient noise was
recolored to the expected sensitivity of the Advanced LIGO-Virgo detectors, as
opposed to colored Gaussian noise that was used in NINJA-1. 
Therefore this project provided a more robust test of our search methods, and 
provided a benchmark against which future search developments could be compared.


While the above concerns primarily comparable mass-ratio binaries, we 
also develop a waveform model for intermediate mass-ratio ones with 
$m_1/m_2 \in [10, 100]$. 
Intermediate mass-ratio systems, containing intermediate mass  
and stellar mass black holes will also be relatively more massive than stellar
mass binaries.
This would shift the frequency of the emitted gravitational radiation to 
lower values, and their late-inspiral and merger would occur in the most
sensitive frequency band of the Advanced detectors. This makes the modeling 
of the later portion of their waveforms crucial to their detection. 
%
First-order conservative self-force corrections have been derived for a
test-particle moving in the background of a supermassive Schwarzschild 
black hole. Using the form of these calculations, we formulate a 
prescription to model the early and late inspiral
of such binaries. Then, using the implicit rotation source picture
(due to Baker et al~\cite{Baker:2008}), we develop a model for the plunge and merger,
where the black holes are close and the orbits are no longer quasi-circular.
We then complete the description by stitching the quasi-normal modes emitted
by the black hole formed at merger. 
Therefore, we complete a model that captures the entire coalescence process
for intermediate mass-ratio binaries of non-spinning black holes.

To summarize, for {\it comparable} mass ratio binaries, we show that a combination
of post-Newtonian and post-Newtonian--Numerical-Relativity hybrid waveforms
would be sufficient for gravitational wave searches. This is true for the 
entire stellar-mass non-spinning binary black hole parameter space.
We also successfully validate gravitational wave search algorithms 
that have been used in the most recent LIGO-Virgo searches, using accurate 
numerical simulations injected in emulated detector noise. 
For {\it intermediate} mass ratios, we develop an accurate waveform 
model that captures the binary dynamics from the weak-field slow-motion
regime to the strong-field regime up to the merger of both compact objects. 
Therefore the work presented in this dissertation is an effort towards
arriving at optimal search filters for non-spinning binary black holes 
which are prospective sources detectable by the second-generation terrestrial
gravitational wave detectors; as well as towards validating existing search 
algorithms using an improved testing methodology.












