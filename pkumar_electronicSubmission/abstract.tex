%%%%%%%%%%%%%%%%%%%%%%%%%%%%%%%%%%%%%%%%%%%%%%
%%%%%%% ABSTRACT
%%%%%%%%%%%%%%%%%%%%%%%%%%%%%%%%%%%%%%%%%%%%%%

% Gravitational radiation is an observable consequence of General Relativity.
% The direct detection of gravitational waves will open up a new spectrum for 
% astronomy. Gravitational waves will allow us to observe astrophysical phenomena
% that are not observable by electromagnetic astronomy. 
% Gravitational-wave observations will also allow strong field tests of
% General Relativity. Experiments designed to detect gravitational waves include
% the Laser Interferometer Gravitational-wave Observatory (LIGO) detectors in
% the United States, the Virgo detector in Italy, GEO600 in Germany, the planned KAGRA
% detector in Japan and the LIGO-India (IndiGo) project. These second-generation
% terrestrial detectors are expected to begin observation in $2015$, and will
% reach their design sensitivity by $2018$.
% 
% Binary systems of compact astrophysical objects, such as black holes and 
% neutron stars, emit gravitational waves as they orbit. The emitted 
% waves carry away energy and angular momentum from the binary, causing the binary's
% orbit to shrink until the binary merges. Stellar-mass compact binaries are the primary
% targets for gravitational wave detection with the second-generation detectors, 
% as these binaries produce signals in the sensitive frequency band of these detectors
% (i.e.~$10-2000$~Hz).
% Gravitational-wave signals are likely lower in amplitude than the instrumental
% noise, making searches challenging. 
% Past gravitational wave searches with LIGO and Virgo detectors used 
% matched-filtering where the data is correlated with theoretically-modeled 
% waveform templates. 
% The matched-filter is very sensitive to the phase of the waveform templates. It 
% is therefore crucial to have faithful compact binary coalescence models to
% maximize the sensitivity of searches for compact binaries.
% 
% There has been steady progress in obtaining analytic solutions to Einstein's
% equations for the two body problem. Several perturbative schemes have been 
% developed to model the emitted gravitational waves. Most of
% the perturbative approaches assume that the binary motion is 
% mildly-relativistic and that orbital radius is large. As a result, the predicted 
% waveforms become increasingly inaccurate as the binary approaches merger. 
% A more accurate description of compact binary mergers is now accessible
% through direct numerical solutions of the Einstein's equations.
% Following the recent breakthroughs in the field of Numerical Relativity, 
% several high accuracy simulations probing different regions of the binary 
% parameter space have been performed. 
% % There has been considerable effort in the recent decades to develop systematic
% % approximations to the Einstein field equations of General Relativity. Most 
% % of such schemes use the assumption that the two bodies are moving slowly and 
% % are well separated. As the binary shrinks, these waveform {\it models} become 
% % increasingly inaccurate. More recently, numerical relativists have been 
% % able to solve the field equations of gravity numerically to high accuracy.
% % With the advent of this field, we have gained detailed knowledge of the last
% % stages of the binary motion before the black holes and/or neutron stars merge.
% However, due to the computational cost of numerical simulations, the 
% longest of these simulations currently model only the last 40 orbits
% of binary coalescence~\cite{Mroue:2013xna}
% (only one longer simulation has been performed~\cite{BelaLongSimulation},
% which has 175 orbits) compared to the 280 orbits for a $(7,7) M_\odot$
% binary in Advanced LIGO band. To bridge this gap, there is ongoing
% research on theoretical techniques that interpolate between  
% perturbative analytic solutions and numerical simulations, in an attempt to 
% capture the entire inspiral and merger process.

In this dissertation we study the applicability of different waveform models 
in gravitational wave searches for comparable mass binary black holes.
% Computational cost is expected to be a significant for advanced LIGO searches. 

We determine the domain of applicability of the computationally inexpensive closed 
form models, and the same for the semi-analytic models that have been calibrated
to Numerical Relativity simulations (and are computationally more expensive).

We further explore the option of using {\it hybrid} waveforms, constructed 
by numerically stitching analytic and numerical waveforms, as filters in
gravitational wave detection searches. Beyond matched-filtering, there 
is extensive processing of the filter output before a detection candidate can
be confirmed. 

We utilize recent results from Numerical Relativity to study
the ability of LIGO searches to make detections, using (recolored) detector data.
Lastly, we develop a waveform model, using recent self-force results, that 
captures the complete binary coalescence process. The self-force formalism was
developed in the context of extreme mass-ratio binaries, and we successfully
extend it to model intermediate mass-ratios.
