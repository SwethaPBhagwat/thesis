
% from Boyle et al.
\subsection{Matched filtering}
\label{sec:MatchedFiltering}

Current searches for gravitational waves from binary black-hole
coalescence use matched filtering to search for a waveform buried in
noise.  The matched filter is the optimal filter for detecting a
signal in stationary Gaussian noise.  Suppose that $n(t)$ is a
stationary Gaussian noise process with one-sided power spectral
density $S_n(f)$ given by $\langle \tilde{n}(f) \tilde{n}^\ast(f')
\rangle=\frac{1}{2} S_n(|f|)\delta(f-f')$.  For long integration
times, the data stream $s(t)$ output by the detector will always be
dominated by the noise.  Thus, we can simply approximate $n \approx s$
to calculate $S_{n}(f)$.

Using this power spectral density (PSD), we can define the inner
product between two real-valued signals---the data stream $s$ and the
filter template $h$---by
\begin{eqnarray}
  \label{eq:InnerProduct}
  \InnerProduct{s|h} &\equiv 2\, \Re \int_{-\infty}^{\infty}\,
  \frac{\tilde{s}(f)\, \tilde{h}^{\ast}(f)}{S_{n}(\lvert f
    \rvert)}\, d f \\ &= 4\, \Re \int_{0}^{\infty}\,
  \frac{\tilde{s}(f)\, \tilde{h}^{\ast}(f)}{S_{n}(f)}\, d f\ .
\end{eqnarray}
Then, given data $s$ which may contain either noise $n$ or noise and a
gravitational wave signal $h$,
\begin{equation}
  s = \left\{\begin{array}{l}
      n  \\
      n+h
    \end{array} \right.\ ,
\end{equation}
the matched-filter signal-to-noise ratio (SNR) is defined as
\begin{equation}
  \label{eq:InnerProductSNR}
  \rho = \frac{1}{\sqrt{\InnerProduct{h|h}}} \InnerProduct{s|h}\ .
\end{equation}


The SNR can then be used to construct a detection statistic (directly
or in combination with other statistics).  It is therefore important
to ensure that the templates used in searches accurately model the
expected waveforms to avoid reduction in the value of $\rho$. The
\emph{overlap} between two templates $h$ and $h'$ is defined as
\begin{equation}
  \label{eq:OverlapDefinition}
  \Overlap{h|h'} \equiv \frac{\InnerProduct{h|h'}}{
    \sqrt{\InnerProduct{h|h} \InnerProduct{h'|h'}}}\ .
\end{equation}
The overlap encodes the fractional loss in SNR that results from using
the template $h'$ rather than the true waveform $h$.  In a search that
uses $\rho$ as a detection statistic this corresponds to the
fractional loss in distance to which the search is sensitive.

The filter template includes arbitrary time and phase offsets, encoded
by the arrival time and phase, $\ta$ and $\phia$.  Under a change of
these quantities, the Fourier transform behaves as
\begin{equation}
  \label{eq:EffectOfTimeAndPhaseOffset}
  \tilde{h}(f) \to \tilde{h}(f)\, \e^{-2\pi i f \ta - i \phia}\ .
\end{equation}
We maximize over these two variables by calculating the inner product
as

\begin{eqnarray}
  \max_{\ta, \phia}\, \InnerProduct{s|h}
  &= \max_{\ta, \phia}\, 4\, \Re \int_{0}^{\infty}\,
  \frac{\tilde{s}(f)\, \tilde{h}^{\ast}(f)}{S_{n}(f)}\, \e^{2\pi i
    f\ta + i \phia}\, d f
  \\
  & = 4 \max_{\ta}\, \left\lvert \int_{0}^{\infty}\,
    \frac{\tilde{s}(f)\, \tilde{h}^{\ast}(f)}{S_{n}(f)}\, \e^{2\pi i
      f\ta}\, d f \right\rvert\ .
\end{eqnarray}
Note that this integral is just the (inverse) Fourier transform of the
quantity $\tilde{s}(f)\, \tilde{h}^{\ast}(f) / S_{n}(f)$ evaluated at
$\ta$.  Thus finding the maximum over $\ta$ involves taking the
Fourier transform and selecting the largest element of the finite set
that results from discretization.
