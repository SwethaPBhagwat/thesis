
\section{Matched filtering}
\label{sec:ihope_match_filter}

Let $n(t)$ be the noise, $h(t)$ a signal, and $s(t)=n(t)+h(t)$ data
from the instrument.

Consider the most general linear filter on the data, in the discrete
case this would be

\begin{equation}
\hat{s} = \sum_i s_i K_i
\end{equation}

Since we will want a real result, we require the $K_i$ to be real.  In
the continuum limit this becomes

\begin{equation}
\hat{s} = \int_{-\infty}^\infty s(t) K(t)\, dt
\end{equation}

We now define the signal strength $S$ as the expected value of
$\hat{s}$ when the signal is present:

\begin{align}
S &= \langle \hat{s} \rangle \\
&= \langle  \int_{-\infty}^\infty s(t) K(t)\, dt \rangle \\
&= \int_{-\infty}^\infty \langle s(t) K(t)\rangle \, dt \\
&= \int_{-\infty}^\infty \langle s(t) \rangle K(t) \, dt \\
&= \int_{-\infty}^\infty \langle n(t) + h(t) \rangle K(t) \, dt \\
&= \int_{-\infty}^\infty \left( \langle n(t) \rangle + \langle h(t)
\rangle \right)  K(t) \, dt \\
&= \int_{-\infty}^\infty \left( 0 + h(t) \right)  K(t) \, dt \\
&= \int_{-\infty}^\infty h(t) K(t) \, dt \\
\end{align}

Now, since $K(t)$ is real we can replace it by its complex conjugate
and then apply Parseval's theorem to write this as

\begin{align}
S &= \int_{-\infty}^\infty h(t) K^\star(t) \, dt \\
&= \int_{-\infty}^\infty \tilde{h}(f) \tilde{K}^\star(f) \, df \\
\end{align}

We characterise the noise $N$ as the rms value of $\hat{s}$ when the
signal is absent:


\begin{align}
N^2 &= \langle \hat{s}^2(t) \rangle - \langle \hat{s}(t) \rangle^2 \\
&= \langle \hat{s}^2(t) \rangle  \\
&= \int_{-\infty}^\infty K(t) K(t') \langle n(t) n(t')
\rangle\,dt\,dt' \\
&= \int_{-\infty}^\infty dt\,dt' K(t) K(t') 
\int_{-\infty}^\infty df\,df' e^{2\pi i f t}e^{-2\pi i f' t'} \langle
\tilde{n}^\star(f) \tilde{n}(f')\rangle \\
\end{align}

\fake{fill in last step}

From the definition of the PSD

\begin{equation}
N^2 = \int_{-\infty}^\infty df\,\frac{1}{2} S_n(f) |\tilde{K}(f)|^2
\end{equation}


We now introduce a new function

\begin{equation}
\tilde{u}(f) = \frac{1}{2} S_n(f) \tilde{K}(f)
\end{equation}

In terms of which we can write the signal-to-noise ration $S/N$ as


\begin{equation}
\frac{S}{N} =
\frac
  {\int_{-\infty}^\infty df\,
   \frac
     {\tilde{h}(f) \tilde{u}^\star(f)}
     {(1/2) S_n(f)}}
  {\left[\int_{-\infty}^\infty df\,
   \frac
     {\tilde{u}(f) \tilde{u}^\star(f)}
     {(1/2) S_n(f)}\right]^{1/2}}
\end{equation}

This motivates the definition of an operator mapping pairs of
functions to real numbers

\begin{equation}
\label{eq:InnerProduct}
\InnerProduct{A|B} 
 = \int_{-\infty}^\infty df\,
   \frac
     {\tilde{A}(f) \tilde{B}^\star(f)}
     {(1/2) S_n(f)}
\end{equation}

Using this the SNR becomes

\begin{equation}
\label{eq:InnerProductSNR}
\frac{S}{N} = \frac{\InnerProduct{h|u}}{\InnerProduct{u|u}^{1/2}}
\end{equation}

Now, the operator defined in~\ref{eq:InnerProduct} has the following
properties

Conjugate symmetry, $(x|y) = (y|x)^\star$

Linearity in the first argument $(ax + by|z) = a(x|z) + b(y|z)$ for
$a,b$ numbers and $x,y,z$ functions.  This follows from the linearity
of the Fourier transform.

Positive-definiteness $(x|x) \geq 0$ and $(x|x) = 0$ iff $x=0$.  This
follows from the positive-definiteness of the product $aa^\star$ for
$a \in \mathcal{C}$.

The operator therefore has all the properties of an inner product on
the vector space of functions.  We may therefore consider

\[
\frac{u}{(u|u)}
\]

to be a normalized vector.

We now seek a function $u$ (and hence $K$) that will maximize he SNR.
In this form it is clear that $u$ and $h$ must be parallel, and
therefore

\begin{equation}
\tilde{K}(f) \propto \frac{\tilde{h}}{S_n(f)}
\end{equation}

The constant cancels in the SNR, and may therefore be set to 1.

The
\emph{overlap} between two templates $h$ and $h'$ is defined as
\begin{equation}
  \label{eq:OverlapDefinition}
  \Overlap{h|h'} \equiv \frac{\InnerProduct{h|h'}}{
    \sqrt{\InnerProduct{h|h} \InnerProduct{h'|h'}}}\ .
\end{equation}

\iffalse
% from Boyle et al.
\subsection{Matched filtering}
\label{sec:MatchedFiltering}

Current searches for gravitational waves from binary black-hole
coalescence use matched filtering to search for a waveform buried in
noise.  The matched filter is the optimal filter for detecting a
signal in stationary Gaussian noise.  Suppose that $n(t)$ is a
stationary Gaussian noise process with one-sided power spectral
density $S_n(f)$ given by $\langle \tilde{n}(f) \tilde{n}^\ast(f')
\rangle=\frac{1}{2} S_n(|f|)\delta(f-f')$.  For long integration
times, the data stream $s(t)$ output by the detector will always be
dominated by the noise.  Thus, we can simply approximate $n \approx s$
to calculate $S_{n}(f)$.

Using this power spectral density (PSD), we can define the inner
product between two real-valued signals---the data stream $s$ and the
filter template $h$---by
\begin{eqnarray}
  \label{eq:InnerProduct}
  \InnerProduct{s|h} &\equiv 2\, \Re \int_{-\infty}^{\infty}\,
  \frac{\tilde{s}(f)\, \tilde{h}^{\ast}(f)}{S_{n}(\lvert f
    \rvert)}\, d f \\ &= 4\, \Re \int_{0}^{\infty}\,
  \frac{\tilde{s}(f)\, \tilde{h}^{\ast}(f)}{S_{n}(f)}\, d f\ .
\end{eqnarray}
Then, given data $s$ which may contain either noise $n$ or noise and a
gravitational wave signal $h$,
\begin{equation}
  s = \left\{\begin{array}{l}
      n  \\
      n+h
    \end{array} \right.\ ,
\end{equation}
the matched-filter signal-to-noise ratio (SNR) is defined as
\begin{equation}
  \label{eq:InnerProductSNR}
  \rho = \frac{1}{\sqrt{\InnerProduct{h|h}}} \InnerProduct{s|h}\ .
\end{equation}


\begin{equation}
\label{eq:InnerProductSNR}
2 = 2
\end{equation}

The SNR can then be used to construct a detection statistic (directly
or in combination with other statistics).  It is therefore important
to ensure that the templates used in searches accurately model the
expected waveforms to avoid reduction in the value of $\rho$. The
\emph{overlap} between two templates $h$ and $h'$ is defined as
\begin{equation}
  \label{eq:OverlapDefinition}
  \Overlap{h|h'} \equiv \frac{\InnerProduct{h|h'}}{
    \sqrt{\InnerProduct{h|h} \InnerProduct{h'|h'}}}\ .
\end{equation}
The overlap encodes the fractional loss in SNR that results from using
the template $h'$ rather than the true waveform $h$.  In a search that
uses $\rho$ as a detection statistic this corresponds to the
fractional loss in distance to which the search is sensitive.

The filter template includes arbitrary time and phase offsets, encoded
by the arrival time and phase, $\ta$ and $\phia$.  Under a change of
these quantities, the Fourier transform behaves as
\begin{equation}
  \label{eq:EffectOfTimeAndPhaseOffset}
  \tilde{h}(f) \to \tilde{h}(f)\, \e^{-2\pi i f \ta - i \phia}\ .
\end{equation}
We maximize over these two variables by calculating the inner product
as

\begin{eqnarray}
  \max_{\ta, \phia}\, \InnerProduct{s|h}
  &= \max_{\ta, \phia}\, 4\, \Re \int_{0}^{\infty}\,
  \frac{\tilde{s}(f)\, \tilde{h}^{\ast}(f)}{S_{n}(f)}\, \e^{2\pi i
    f\ta + i \phia}\, d f
  \\
  & = 4 \max_{\ta}\, \left\lvert \int_{0}^{\infty}\,
    \frac{\tilde{s}(f)\, \tilde{h}^{\ast}(f)}{S_{n}(f)}\, \e^{2\pi i
      f\ta}\, d f \right\rvert\ .
\end{eqnarray}
Note that this integral is just the (inverse) Fourier transform of the
quantity $\tilde{s}(f)\, \tilde{h}^{\ast}(f) / S_{n}(f)$ evaluated at
$\ta$.  Thus finding the maximum over $\ta$ involves taking the
Fourier transform and selecting the largest element of the finite set
that results from discretization.
\fi

\section{Estimation of the PSD}
\label{sec:ihope_psd}

\section{Trigger selection}
\label{sec:analysis_trigger_selection}

\section{The $\chisq$ test}
\label{sec:ihope_chisq}

\begin{equation}
\label{eq:chisq}
\chisq = \chisq
\end{equation}

\section{Construction of the Bank}
\label{sec:bank_metric}

\section{Detection Statistics}
\label{sec:detection_statistics}

\begin{equation}
\label{eq:new_snr}
\newsnr^2 = \begin{cases}
 \rho^2, & \chi^2_{r} \leq 1, \\ 
 \frac{\rho^2}{[(1+(\chi_r^2)^3)/2]^{1/6}}, \ & \chi^2_{r} > 1,
\end{cases}  
\end{equation}

\section{Testing the Pipeline with Hardware Injections}
\label{sec:ihope_hardware_injections}


