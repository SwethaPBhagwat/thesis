Gravitational waves are one of the most remarkable predictions of
Einstein's general theory of relativity~\cite{Einstein:1916}.  In a
sense they are an immediate consequence of reconciling gravity with
special relativity.  As nothing can travel faster than light,
information about the change in position of an object can not
propagate instantaneously.  The gravitational field therefore becomes
a dynamical entity, much like the electromagnetic field in Maxwell's
equations.

There is currently compelling indirect evidence for the existence of
gravitational waves.  These waves carry energy away from their source,
and the energy in a gravitationally-bound system of two stars or black
holes correlates directly to the separation between them.  It is
therefore possible to track the change in separation of such a binary
system, infer the power being radiated away, and compare this to the
predicted power radiated by gravitational waves.  Hulse and Taylor
were awarded the Nobel prize in 1994 for the discovery of a system
exhibiting such behavior~\cite{Hulse:1994, Taylor:1994}. Continued
observation of this system over the intervening years has shown that
the loss of energy matches the prediction from general relativity to
within 0.2\%~\cite{Weisberg:2004hi}.


Direct detection of gravitational waves would do far more than further
confirm the prediction of their existence.  Such detections would open
a new window on the universe.  Radio, X-ray and microwave astronomy
have each provided views of the Universe not available though the
others, and each has taught us a great deal.  The gravitational
spectrum is completely new, and so far unexplored.  Gravitational wave
astronomy can probe regions that the electromagnetic spectrum can not,
including black holes, the interiors of neutron stars and supernovae,
and the very early universe.  Gravitational wave astronomy also offers
a unique opportunity to test strong-field general relativity.

Attempts to directly detect gravitational waves are hampered by the
fact that they interact very weakly with matter.  This is a
consequence of the fact that gravity is comparatively a very weak
force; a small magnet can, over a short distance, balance the
gravitational pull of the entire Earth.  Nevertheless, direct
detection is possible.  There is currently a worldwide effort underway
using interferometric techniques consisting of:

\begin{itemize}

\item The Laser Interferometer Gravitational wave Observatory (LIGO)
in the US, with detectors in Hanford, WA and Livingston,
LA~\cite{Sigg:2008}.

\item Virgo, in Cascina, Italy~\cite{Acernese:2008}.

\item GEO 600, near Sarstedt, Germany~\cite{Grote:2008}.

\item The Large-scale Cryogenic Gravitational wave Telescope (LCGT) is
presently about to being construction in
Japan~\cite{0264-9381-27-8-084004}.
\end{itemize}

The LIGO, Virgo and GEO detectors have all reached their initial
design sensitivities enabling detection of gravitational waves from
the final stages of systems like the Hulse-Taylor binary out to (in
the case of LIGO) as far as 45 Megaparsecs (Mpc).  As a point of
reference, the Virgo supercluster extends about 33 Mpc and contains
thousands of galaxies, including our own.

Although there are many potential sources of gravitational waves, a
particularly promising one is the inspiral and coalescence of binary
systems containing compact objects, neutron stars or black
holes~\cite{thorne.k:1987} (collectively called \emph{compact binary
coalescence}, henceforth CBC).  Over a large range of masses the
frequencies at which such systems emit the most power are the
frequencies at which LIGO and Virgo are the most sensitive, from about
40 Hz to 1000 Hz.  In chapter~\ref{ch:theory} we will show that the
effect of gravitational waves is to change the distance between
freely-falling objects.  In this frequency band LIGO and Virgo are
sensitive to changes in length to about 1 part in $10^22$.

Despite the detectors' remarkable sensitivity, signals in the LIGO
data will likely be quiet relative to the noise levels.  We will have
more to say about this noise in Sec.~\ref{sec:noise_sources} and
chapter~\ref{ch:detchar}, but note here that it contains both
background ``static'' which may be modeled as a \emph{Gaussian random
process} and isolated \emph{non-Gaussian transients}.  Sophisticated
data analysis techniques will be required to extract the signals from
noise and infer properties of their sources.

In addition to the achievements at LIGO and other sites, recent years
have also seen major progress in the field of \emph{numerical
relativity}, the use of computers to accurately simulate systems
governed by general relativity.  From Pretorius' first successful
simulation of the final orbit and merger of two black holes in
2005~\cite{Pretorius:2005gq}, there are now routine simulations of
binary black hole systems with a wide variety of parameters, extending
through several orbits.  There are also an increasing number of
simulations of systems containing matter, including both black holes
in gaseous environments and neutron stars and supernovae, although we
will not consider simulations containing matter further in this
dissertation.


The goal of the research presented in this dissertation is to improve
the efficiencies of CBC searches.  In order to claim a detection a
candidate signal must stand out sufficiently above the noise.
Efficiency can therefore be improved either by increasing the
significance of the signal or by reducing the noise.  We take both
approaches here.

The remainder of this dissertation is organized as follows:

\begin{itemize}

\item In chapter~\ref{ch:theory} we present the theory of
gravitational waves, starting from general relativity and deriving
the effect of waves on matter.  We also discuss both analytic and
computational approaches to predicting the form of gravitational
waves.

\item In chapter~\ref{ch:ligo_detectors} we start from the effect of
gravitational waves on matter to discuss the principles of
interferometric detectors. 

\item In chapter~\ref{ch:search} we describe the search for CBC
signals in data from interferometric detectors.

\item Chapter~\ref{ch:comparison} begins the process of utilizing
predictions from numerical relativity to improve the efficiency of
searches.  We proceed by comparing the analytic models used in
searches to a high-accuracy simulation from the Caltech-Cornell group.
Based on these comparisons we derive a number of possible improvements
to the search,

\item Chapter~\ref{ch:ninja1} introduces the first Numerical Injection
Analysis (NINJA) project, a collaboration between numerical
relativists and gravitational-wave astronomers.  The goal of NINJA is
to study the effectiveness of numerous search methods at detecting
gravitational waves and extracting information about their sources.
We apply the recommendations from the previous chapter to the NINJA
data, which simulates the output from LIGO and contains signals from
numerical relativity.

\item Chapter~\ref{ch:ninja2} discusses some of the limitations of the
first NINJA project and introduces NINJA-2.  A notable feature of
NINJA-2 is the requirements placed on the NR submissions.  We discuss 
the comprehensive analyses that have been performed to ensure that the
submissions meet these requirements.

\item Chapter~\ref{ch:ninja2_results} discusses the construction of the
simulated data sets for NINJA-2 and presents preliminary results from
the CBC analysis of the data.  We note some unexpected results that
need further study and raise questions that we plan to address in
NINJA-2.

\item We then transition to the topic of detector noise.  In
chapter~\ref{ch:segdb} we describe the infrastructure used to record 
information about the state of the detectors.

\item Then in chapter~\ref{ch:detchar} we present a reduced,
simplified version of the CBC search.  This was used in the latest
runs of the LIGO and Virgo detectors to identify times of excess noise
in the detector and remove these times from analysis.  We include some
open questions about the implementation details of the search revealed
during these analyses.

\end{itemize}

Throughout this dissertation we focus on LIGO's 6th science run
(denoted ``S6'', July 7, 2009 - October 20, 2010) which overlapped
Virgo's second (``VSR2'', July 7, 2009 - January 11, 2010), and third
(``VSR3'', August 11, 2010 - October 20, 2010) science runs.  This
corresponds to ``Enhanced LIGO'', so called because this run included
a number of modifications intended for Advanced LIGO (to start in
2015) but which were available and could be added to the Initial LIGO
configuration without major construction.

