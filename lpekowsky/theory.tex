In this chapter we briefly introduce the major principles of
gravitational wave astronomy.  We start in
section~\ref{sec:general_relativity} with a review of general
relativity, starting from the relevant mathematics.  In
section~\ref{sec:gravitational_radiation} we show how general
relativity predicts the existence of gravitational radiation and
discuss some of the properties of this radiation.  Then in
section~\ref{sec:effects_of_waves} we show how gravitational radiation
affects freely-falling particles.  This will motivate the design of
the LIGO experiment to search for gravitational waves, an overview 
which is presented in the next chapter.  We then move to the
generation of gravitational waves and discuss two approaches, analytic
and numerical, for modeling the waves produced by the inspiral and
eventual merger of pairs of neutron stars and/or black holes.



section~\ref{sec:ligo_detectors}.

\section{General Relativity}
\label{sec:general_relativity}

We start with an overview of differential geometry and build to
Einstein's equations.  This is of necessity very brief, readers are
referred to the textbooks by Misner, Thorne and Wheeler~\cite{MTW} and
Carroll~\cite{carrollTextbook} for more complete treatments.

\subsection{Elements of differential geometry}

An $n$-dimensional ($C^\infty$) manifold $\mathcal{M}$ is a set of
points plus an \emph{atlas}, a set of \emph{charts} $\{\phi_i\}$ which
are invertible maps from open subsets of $\mathcal{M}$ to open subsets
of $\mathcal{R}^n$ such that


\begin{itemize}
\item For all points $p \in \mathcal{M}$ there exists an $\phi_i$ 
such that $p$ is in the domain of $\phi_i$.
\item The composition $\phi_i \circ \phi_j^{-1}$ on the 
intersections of the domains of $\phi_i$ and $\phi_j$ is a
($C^\infty$) function from $\mathcal{R}^n \to \mathcal{R}^n$.
\end{itemize}

Two natural structures on a manifold are curves, maps from
$\mathcal{R}\to\mathcal{M}$, and scalar functions,  maps from
$\mathcal{M}\to\mathcal{R}$.  Compositing a function $f$ with a curve
$\gamma(\lambda)$ gives a map from $\mathcal{R} \rightarrow
\mathcal{R}$ which may be differentiated in the usual way at a point
$p$.

\iffalse
\begin{equation*}
\frac{d}{d \lambda} fi \big|_p = 
  \frac{d}{d\lambda} (f \circ \gamma) \big|_p
\end{equation*}


Expanding this in terms of a chart whose domain includes $p$ and then
applying the chain rule
 
\begin{align*}
\frac{d}{d \lambda} fi \big|_p &= 
 \frac{d}{d\lambda} ( (f \circ \phi^{-1}) \circ (\phi \circ
\lambda) ) \\
&= \frac{d(\phi^{-1} \circ \gamma)^\mu}{d\lambda} 
\frac{\partial (f \circ \phi^{-1}) }{\partial x^\mu} \big|_p \\
&= \frac{dx^\mu}{d\lambda} \partial_\mu f \big|_p
\end{align*}

where $x^\mu$ are the coordinates on $\mathcal{R}^n$.  Finally, since
the function $f$ is arbitrary we can define

\begin{equation}
\label{eq:tangent_vector}
\frac{d}{d\lambda} = \frac{dx^\mu}{d\lambda} \partial_\mu
\end{equation}
\fi

Geometrically, taking the derivative gives the tangent vector to the
curve at the point $p$.  It is possible to associate the set of such
vectors with the set of directional derivatives, taking the partial
derivatives along the coordinates as the basis.  Henceforth this basis
will be denoted both $\partial_\mu$ and $\vec{e}_\mu$.

Note that the tangent to a curve is defined at the point $p$.  Each
point in the manifold possesses its own space of tangent vectors.
These spaces are distinct, which will be important in what follows.

We next define \emph{one-forms} as linear maps from vectors to
$\mathcal{R}$.  The set of one-forms at a point can be shown to form a
vector space, a natural basis for which can be obtained by requiring

\begin{equation*}
\vec{e}_\mu \tilde{\omega}^\nu = \delta_\mu^\nu
\end{equation*}

The components of an arbitrary form $\omega$ in this basis may be
found by applying the form to the basis vectors.

\iffalse
\begin{align*}
\omega(\vec{e}_\nu)
&= \tilde{\omega}^\mu \omega_\mu (\vec{e}_\nu) \\
&= \omega_\mu \tilde{\omega}^\mu (\vec{e}_\nu) \\
&= \omega_\mu \delta^\nu_\mu \\
&= \omega_\nu\\
\end{align*}
\fi

We can then build up arbitrary ${m \choose n}$ tensors as linear maps
from tensor products of $m$ vectors and $n$ one-forms to
$\mathcal{R}$.  The components of a tensor $T$ in some coordinates may
found by applying it to combinations of the basis vectors and 1-forms.


Finally, a ${m \choose n}$ tensor field is a map that associates to
each point $p$ in $\mathcal{M}$ an element in the space of  ${m
\choose n}$ tensors at $p$.

\subsection{The metric tensor}

A particularly important tensor in general relativity is the
\emph{metric}, a ${2 \choose 0}$ tensor that is symmetric ($g_{\mu\nu}
= g_{\nu\mu})$ and non-degenerate (the determinant of $g$ taken as a
matrix $|g_{\mu\nu}| \neq 0$.  The latter feature makes it possible to
define the inverse metric $g^{\mu\nu}$ as

\begin{equation*}
g^{\mu_\rho} g_{\rho_\nu} = \delta^\mu_\nu
\end{equation*}

Given a vector $x^\mu$ the object $g_{\mu\nu} x^\mu$ maps another
vector to a real number, and is therefore a one-form.  The metric
therefore maps between the space of one-forms and the space of vectors
at each point.  Most importantly, the metric defines a notion of
distance on the manifold.  Infinitesimally

\begin{equation}
ds^2 = g_{\mu\nu} dx^\mu dx^\nu
\end{equation}

In special relativity, in Cartesian coordinates, the metric has
components $(-1,1,1,1)$ along the diagonal, all other components are
zero.   The metric will be denoted $\eta$ and called the \emph{flat
space metric}.

\subsection{Covariant derivatives}

Since vectors are only defined at a point we need additional structure
to define derivatives of vector fields, as there is no natural way to
compare vectors that live in different spaces.  We seek an operator
$\nabla$ with the following properties

\begin{itemize}
\item Maps ${m \choose n}$ tensors to ${m \choose {n+1}}$ tensors.
This is so we may consider the directional derivative of a tensor $T$
along a vector $x$ as $x^\mu \nabla_\mu T$.
\item Reduces to partial differentiation when applies to a scalar
field.
\item Linear.
\item Satisfies the Leibniz rule, $\nabla(a b) = a\nabla b + b \nabla a$.
\end{itemize}

Such an operator applied to a vector field gives

\begin{equation*}
\nabla_\mu (x^\nu \vec{e}_\mu)
= (\partial x^\nu) \vec{e}_\mu + x^\nu (\nabla_\nu \vec{e}_\mu)
\end{equation*}

In a flat space in Cartesian coordinates the basis vectors do not
change and so the last term is zero.  But in a curved space, or even
flat space in non-Cartesian coordinates, they may.  However, the new
vector must be expressible as a linear combination of the original
basis vectors  The components are called \emph{connection
coefficients} and are denoted as $\Gamma^\rho_{\nu\mu}$ so

\begin{align}
\label{eq:covariant_derivative}
\nabla_\mu (x^\nu \vec{e}_\mu) &= 
(\partial x^\nu) \vec{e}_\mu + 
x^\nu \Gamma^\rho_{\nu\mu} \vec{e}_\rho \\
&= (\partial x^\nu + x^\rho \Gamma^\nu_{\mu\rho}) \vec{e}_\nu \\
\nabla_\mu x^\nu &= \partial x^\nu + x^\rho \Gamma^\nu_{\mu\rho}
\end{align}

In general relativity the connection is usually assumed to be
\emph{torsion-free}, that is

\begin{equation}
\label{eq:torsion}
 \Gamma^\nu_{\mu\rho} =  \Gamma^\nu_{\rho\mu}
\end{equation}

which thus far has been borne out by experiment.  However, it is
possible to construct theories where this condition does not hold.

By considering the covariant derivative of a scalar constructed from a
one-form acting on a vector, $\nabla (x^\nu \omega_nu)$, it can be shown
that

\begin{equation*}
\nabla_\mu \omega_\nu = \partial_\mu \omega_\nu - 
\Gamma^\rho_{\mu\nu} \omega_\rho
\end{equation*}

The covariant derivative of a ${m \choose n}$ tensor generalizes this
and has a partial derivative term, $m$ positive therms in $\Gamma$ and
$n$ negative terms in $\Gamma$.

\subsection{Parallel Transport}
\label{ssec:parallel}

Covariant differentiation provides a way to ``move a vector without
changing it.''  We can \emph{parallel transport} a vector $v^\mu$
infinitesimally along a curve whose tangent vector is $u^\nu$ by
requiring

\begin{equation*}
u^\nu \nabla_\nu v^\mu = 0
\end{equation*}

As an example of such transport, consider an arrow on the equator of
the Earth pointing towards the north pole.  This arrow can be carried
halfway around the equator without rotating it, so it ends up on the
other side of the globe, still pointing north.  If the vector is then
parallel transported northward to the pole and then continued until it
returns to its starting point it will return pointing south.  Although
the vector was never rotated locally it has returned rotated.  This is
an indication that the underlying space is curved.

Of particular interest is the case where a vector is
parallel-transported along itself

\begin{equation*}
0 = v^\mu \nabla_\mu v^\nu 
= v^\mu (\partial_\mu v^\nu + \Gamma^\nu_{\mu\rho} v^\rho)
\end{equation*}

Now consider a curve $x(\lambda)$ such that $v$ is the tangent to this
curve, $v^\mu = d x^\mu/d\lambda$, then

\begin{align}
\label{eq:geodesic}
\frac{d x^\mu}{d\lambda}
  \frac{\partial }{\partial x^\mu}
  \frac{d x^\nu}{d\lambda}  
+ \Gamma^\nu_{\mu\rho} 
\frac{d x^\mu}{d\lambda}
\frac{d x^\rho}{d\lambda} &= 0 \nonumber \\
\frac{d^2 x^\nu}{d\lambda^2}
+ \Gamma^\nu_{\mu\rho} 
\frac{d x^\mu}{d\lambda}
\frac{d x^\rho}{d\lambda} &= 0 \nonumber \\
\end{align}

This is the \emph{geodesic equation}, solutions to which are
\emph{geodesics}.  The same equation can be derived by extremizing
the path length, $\sqrt{g_{\mu\nu} dx^\mu dx^\nu}$.

In general relativity test masses acting under the influence of
gravity and no other forces follow geodesics.  

\subsection{The Christoffel Symbols}

If we now require that scalars do not change under parallel transport
we have, for arbitrary vectors fields $u^\alpha, v^\beta$ and $x^\mu$

\begin{align*}
0 &= x^\mu \nabla_\mu (g_{\alpha\beta} u^\alpha v^\beta) \\
&= x^\mu (\nabla_\mu g_{\alpha\beta}) u^\alpha v^\beta
+ g^{\alpha\beta} (x^\mu \nabla_\mu u^\alpha) v^\beta
+ g^{\alpha\beta} u^\alpha (x^\mu \nabla_\mu v^\beta)
\end{align*}

We can now specialize such that $u^\alpha, v^\beta$ are constant
and so the last two terms vanish, which implies the  \emph{metric
compatibility} condition:

\begin{equation}
\label{eq:metric_compatibility}
\nabla_\mu g_{\alpha\beta} = 0
\end{equation}

Equations~\ref{eq:metric_compatibility} and~\ref{eq:torsion} together
fix the connection coefficients in terms of the metric:

\begin{equation}
\Gamma^{\rho}{\mu\nu}
= \frac{1}{2} g^{\rho\sigma}\left[
\partial_\nu g_{\mu\sigma}
+ \partial_\mu g_{\nu\sigma}
- \partial_\sigma g_{\mu\nu}
\right]
\end{equation}


Combining this with the previous section we see that the motion of a
particle is completely specified once we know the metric.

\subsection{The Riemann Tensor}

We now generalize the example given in section~\ref{ssec:parallel}, and
ask how a vector $A^\mu$ changes as it is parallel-transported around
an infinitesimal parallelogram with sides defined by the vectors
$B^\mu$ and $C^\nu$.  Recalling that vectors and directional
derivatives are the same thing, it can be shown that this is
equivalent to asking how covariant derivatives fail to commute.  The
result must be linear in the vectors and so we may write

\begin{equation}
\label{eq:riemann_def}
\left[\nabla_\mu \nabla_\nu - \nabla_\nu \nabla_\mu\right] A^\rho
= R^\rho_{\sigma\mu\nu} A^\sigma
\end{equation}

which defines the \emph{Riemann tensor} $R^\rho_{\sigma\mu\nu}$.  A
number of properties follow from this definition (which are either
obvious or may be proven by substituting the definition of the
covariant derivative, eqn.~\ref{eq:covariant_derivative}).

First, the symmetry properties

\begin{equation}
\label{eq:symmetries}
R_{\rho\sigma\mu\nu}
= -R_{\sigma\rho\mu\nu}
= -R_{\rho\sigma\nu\mu}
= R_{\mu\nu\rho\sigma}
\end{equation}

which in turn imply

\begin{align}
R^\rho_{[\sigma\mu\nu]} = 0
\end{align}

Second, the Bianchi identity,

\begin{equation}
\label{eq:bianchi}
R_{\rho\sigma\mu\nu;\alpha}
+R_{\rho\sigma\nu\alpha;\mu}
+R_{\rho\sigma\alpha\mu;\nu} = 0
\end{equation}

We may now generalize equation~\ref{eq:riemann_def} and ask how an
arbitrary tensor changes after being parallel-transported 
around a loop.  It can be shown that

\begin{equation}
\label{eq:higher_order_riemann}
\left[\nabla_\mu \nabla_\nu - \nabla_\nu \nabla_\mu\right] 
B^{\rho_1 \rho_2 \ldots \rho_n}
= - R^{\rho_1}_{\sigma \mu\nu} B^{\sigma \rho_2 \ldots \rho_n}
- R^{\rho_2}_{\sigma \mu\nu} B^{\rho_1 \sigma \ldots \rho_n}
- \ldots -
- R^{\rho_n}_{\sigma \mu\nu} B^{\rho_1 \rho_2 \ldots \sigma }
\end{equation}

which may be proved by expanding

\begin{equation*}
\left[\nabla_\mu \nabla_\nu - \nabla_\nu \nabla_\mu\right] 
(\vec{e}_\rho \otimes \vec{e}_\sigma)
\end{equation*}

and then proceeding by induction.  

The symmetries of the Riemann tensor imply that there is, up to sign,
only one non-trivial contraction

\begin{equation}
R_{\mu\nu} = R^\sigma_{\mu\sigma\nu}
\end{equation}

which defines the \emph{Ricci tensor}.  This may be contracted again

\begin{equation}
R = R^\mu_\mu
\end{equation}

to obtain the \emph{Ricci scalar}.

Contracting the Bianchi identity twice gives

\begin{equation*}
g^{\sigma\nu}
\left(\nabla_\alpha R_{\sigma\nu}
+ \nabla^\rho R_{\rho\sigma\nu\alpha}
+ \nabla_\nu R^\mu_{\sigma\alpha\mu}\right) = 0
\end{equation*}

Using the symmetries of the Riemann tensor (eqn.~\ref{eq:symmetries})
this can be written

\begin{equation*}
\nabla_\alpha R
- \nabla^\rho R_{\rho\alpha}
- \nabla^\sigma R_{\sigma\alpha} = 0
\end{equation*}

Relabeling the dummy indices and using metric compatibility gives

\begin{equation*}
\nabla^\rho \left(g_{\rho\alpha} R - 2 R_{\rho\alpha} \right) = 0
\end{equation*}

This motivates the definition of the \emph{Einstein Tensor} as

\begin{equation}
\label{eq:einstein_tensor}
G_{\mu\nu} = R_{\mu\nu} - \frac{1}{2} g_{\mu\nu} R
\end{equation}

the previous result implies this is divergentless

\begin{equation*}
\nabla^\nu G_{\mu\nu} = 0
\end{equation*}

Note that $G$ is also symmetric, $G_{\mu\nu} = G_{\nu\mu}$.

We now relate this to physics by noting that the matter and energy
content of a region is described by the stress-energy tensor
$T_{\mu\nu}$ where each component is ``the flow of $\mu$ momentum in the
$\nu$ direction.''  For example, the $0,0$ component is energy density
and the $0,i$ components are the $i^\mathrm{th}$ components of
momentum.

Conservation of energy requires that the difference in momentum
across each face of a cube be balanced by a change of energy,
within the cube,

\begin{equation*}
\partial_t \rho = \partial_i p^i
\end{equation*}

In terms of the stress-energy tensor this becomes

\begin{equation*}
0 = - \nabla^0 T_{00} \nabla^i T_{0i} = 0
= \nabla^\nu T_{0 \nu}
\end{equation*}

However the time direction is not uniquely specified as a change of
coordinates will mix space and time components, so this must
generalize to 

\begin{equation*}
\nabla^\nu T_{\mu\nu} = 0
\end{equation*}

That is, $T$ is also divergentless, like $G$, and like $G$ it is also
symmetric.  It is therefore reasonable to suggest the ansatz

\begin{equation*}
G_{\mu\nu} \propto T_{\mu\nu}
\end{equation*}

Requiring agreement with Newton's law of gravity in the appropriate
low-energy limit ($T_{00} \gg$ all other components) fixes the
constant of proportionality and gives us \emph{Einstein's field
equation}

\begin{equation}
\label{eq:einsteins_equation}
G_{\mu\nu} = 8\pi T_{\mu\nu}
\end{equation}


Note that $G_{\mu\nu}$ is entirely determined by the metric.
Equation~\ref{eq:einsteins_equation} may therefore be thought of as a
set of coupled, non-linear differential equations for $g$.

\section{Gravitational radiation}
\label{sec:gravitational_radiation}

We now move to the prediction of gravitational waves.  We begin with
Einstein's equation in empty space,

\begin{equation*}
G_{\mu\nu} = R_{\mu\nu} - \frac{1}{2} g_{\mu\nu} R = 0
\end{equation*}

By taking the trace and substituting into~\ref{eq:einsteins_equation}
it can be shown that this implies that in empty space $R_{\mu\nu} =
0$.

Using the Bianchi identity and symmetries of the Riemann tensor gives,
in empty space,

\begin{equation}
\label{eq:divergence_in_empty_space}
R_{\beta\delta;\gamma}  -R_{\beta\gamma;\delta} = 0
\end{equation}

We next consider the application of the wave operator to the Riemann
tensor.  From the Bianchi identity (eqn.~\ref{eq:bianchi}) this becomes

\begin{equation*}
\label{eq:wave_expanded}
g^{\mu\nu} R_{\alpha\beta\gamma\delta;\mu\nu}
= - g^{\mu\nu}
\left[R_{\alpha\beta\delta\mu;\gamma\nu}
+ R_{\alpha\beta\mu\gamma;\delta\nu} \right]
\end{equation*}

Consider the first term on the right-hand side:

\begin{align*}
g^{\mu\nu} R_{\alpha\beta\delta\mu;\gamma\nu}
&= g^{\mu\nu} R_{\alpha\beta\delta\mu;\nu\gamma}
+ g^{\mu\nu} R_{\alpha\beta\delta\mu;\gamma\nu}
- g^{\mu\nu} R_{\alpha\beta\delta\mu;\nu\gamma} \\
&= g^{\mu\nu} R_{\alpha\beta\delta\mu;\nu\gamma}
+ g^{\mu\nu} 
\left[\nabla_\nu,\nabla_\gamma\right] R_{\alpha\beta\delta\mu}
\end{align*}

The first term vanishes by
equation~\ref{eq:divergence_in_empty_space}.  The second term involves
products of the Riemman tensor by~\ref{eq:higher_order_riemann}.  The
second term on the right in equation~\ref{eq:wave_expanded} has the
same form.

We now specialize to the case where the Riemann tensor is small, so
that terms involving multiple factors can be neglected.  This is
equivalent to considering the Riemann tensor as a field on a flat
background.  This gives

\begin{equation}
\label{eq:riemann_wave}
g^{\mu\nu}
R_{\alpha\beta\gamma\delta;\mu\nu}
=
\Box R_{\alpha\beta\gamma\delta;\mu\nu}
= 0
\end{equation}

That is, each component of the Riemann tensor independently 
satisfies the vacuum wave equation.  We can immediately write the
solution:

\begin{equation}
R^\alpha_{\beta\gamma\delta} = 
\textrm{Re}\, A^\alpha_{\beta\gamma\delta} \exp(i k_\mu x^\mu)
\end{equation}

where $A$ is a set of amplitudes and $k^\mu$ is the wave vector.  In a
chosen coordinate system it has components $(\omega, k_x, k_y, k_z)$
where $\omega$ is the angular frequency and the spacial $k$ components
are wavelengths in each direction.  It can be shown that 

\begin{align*}
\nabla_{\vec k} \vec{k} &= 0 \\
k_\mu k^\mu &= 0 \\
\end{align*}

which together imply that gravitational waves travel along geodesics 
at the speed of light.


\section{Effect of gravitational waves}
\label{sec:effects_of_waves}

We now derive the effect of gravitational waves on matter.  Consider
two particles moving along world-lines $A^\mu$ and $B^\mu$.  Choose
coordinates so that $A$ remains fixed at the origin, $A^\mu =
(1,0,0,0)$.  We may further specialize our coordinates such that at
the origin $g_{\mu\nu} = \eta_{\mu\nu}$.  It can be shown that we may
also require that the first derivatives of the metric vanish at this
point.  We may not, however, make the second derivatives vanish in
general.  This corresponds to the fact that the Riemann tensor is
defined in terms of second derivatives.  We call the coordinate system
thus constructed a \emph{Local Lorentz Frame}.

We now define the separation between the two particles as 

\begin{equation*}
\xi^\mu = B^\mu - A^\mu
\end{equation*}

We fix $\xi$ to be perpendicular to $A$, so that $\xi^0 = 0$

If space is curved it can readily be seen that $\xi$ will not remain
constant.  For example, if the particles are initially at rest some
distance from the surface of the Earth they will both move towards 
the center of the Earth and $\xi$ will decrease.  It can be shown that 
$\xi$ obeys the equation of \emph{geodesic deviation},

\begin{equation}
\label{eq:geodesic_deviation}
\frac{d^2}{dt^2} \xi^\rho = -R^\rho_{\mu\nu\sigma} A^\mu \xi^\nu A^\sigma
=-R^\rho_{0 \nu 0} \xi^\nu 
\end{equation}

Using the condition that $\xi$ has no time component reduces this to 

\begin{equation}
\frac{d^2}{dt^2} \xi^i = -R^\rho_{\mu\nu\sigma} A^\mu \xi^\nu A^\sigma
=-R^i_{0 j 0} \xi^j
\end{equation}

That is, the change in separation between two
infinitesimally-separated test masses at rest with respect to each
other in an arbitrary gravitational field is entirely specified by
$R^i_{0 j 0}$.  From the symmetries of the Reimann tensor this is
symmetric in $i$ and $j$, and hence appears to have 6 independent
components.  However, it can be shown that these can entirely be
specified by two values, which without loss of generality we take to
be $R^x_{0 x 0}$.  $R^x_{0 y 0}$.

We now recall that in empty space $R_{\mu\nu} = 0$, which implies that
$R^y_{0 y 0} = - R^x_{0 x 0}$.  We summarize this by saying that R is
\emph{traceless}.  

We now specialize to the case of gravitational waves, so that the
Riemann tensor satisfies equation~\ref{eq:riemann_wave} and we choose
coordinates such that the wave is traveling in the $z$ direction.
Using the fact that the speed of light is 1 in dimensionless units the
solution can then be written

\begin{equation*}
R_{i 0 j 0} = A_{ij}(t - z)
\end{equation*}

where we have lowered the first index to simplify notation.

Now, using the fact that $\partial_x R_{i0j0} = \partial_y R_{i0j0}
= 0$ and integrating the Bianchi identity we can show that
$R^x_{0 y 0} = R^y_{0 x 0}$ and that all other components vanish.

It can also be shown that in addition to being traceless $R$ is
\emph{transverse}, $k^j R_{i0j0} = 0$.  We denote these two facts by
adding the superscript $TT$, and define the gravitational wave field as

\begin{equation}
\label{eq:wave_field}
-\frac{1}{2} \frac{\partial^2 h_{ij}^{TT}}{\partial t^2}
\equiv R^{TT}_{i0j0}
\end{equation}

We now decompose the separation vector $\xi$ into the initial
separation and a time-dependant perturbation, $\xi = \xi_0 + \delta
\xi$.  In terms of this equation~\ref{eq:geodesic_deviation} becomes

\begin{equation}
\label{eq:geodesic_deviation_delta}
\frac{d^2}{dt^2} \delta \xi^i = -R^{0 i 0 j} \xi_0^j
\end{equation}

where we drop the initial portion from the left-hand side because it
is constant, and we drop the perturbation from the right hand side
because it is much smaller than the initial portion.  Comparing
eqn.~\ref{eq:wave_field} and eqn.~\ref{eq:geodesic_deviation_delta}
we obtain the equation for the effect of a gravitational wave on
free-falling test masses:

\begin{equation}
\label{eq:wave_effect}
\delta \xi^i = \frac{1}{2} h^{TT}_{ij} \xi^j
\end{equation}

We note in passing that this is the same result obtained in other
treatments by expanding the metric in terms of the flat-space metric
plus a perturbation, $g_{\mu\nu} = \eta_{\mu\nu} + h_{\mu\nu}$,
substituting into the Einstein equation and expanding to first order
in $h$, and then choosing a gauge in which $h$ is transverse and
traceless.

Now, define

\begin{align*}
h_+ &\equiv h_{xx} = - h_{yy} \\
h_\times &\equiv h_{xy} = h_{yx}
\end{align*}

which we refer to as the \emph{plus} ($+$) and \emph{cross} ($\times$)
polarizations, respectively.  Consider the case where $h_\times = 0$.
If particle $B$ is initially on the $x$ axis then 

\begin{align*}
\delta \xi^x &= \frac{1}{2} h^{TT}_{xx} \xi^x \\
\delta \xi^y &= \frac{1}{2} h^{TT}_{xy} \xi^y \\
&= 0
\end{align*}

The particle remains on the $x$ axis.  For an oscillatory wave the
distance between the two particles likewise oscillates.  We can
describe this as an induced \emph{strain}, $\Delta L/L$ where $L$ is
the initial separation.  If $B$ is initially on the $y$ axis

\begin{align*}
\delta \xi^x &= \frac{1}{2} h^{TT}_{xx} \xi^x \\
&= 0 \\
\delta \xi^y &= \frac{1}{2} h^{TT}_{yy} \xi^y \\
&= - \frac{1}{2} h^{TT}_{xx} \xi^y
\end{align*}

The particle remains on the $y$ axis and oscillates out of phase with
a corresponding particle on the $x$ axis.  The net effect is that,
after a quarter cycle, a set of masses initially in a circle are moved
into an ellipse flattened along one axis and stretched along the other
such that the area remains constant.  After another quarter cycle they
return to a circle, in the next quarter cycle they are in an ellipse
with the axes flipped, and so on.

It is similarly straightforward to show that for a wave
cross-polarized wave the eigendirections are on the lines $x=\pm y$.
The effects are the same as for the $+$ polarization, rotated 45
degrees.

\section{Conclusions}

In this chapter we reviewed the basic properties of gravitational
waves and methods used to model such waves from the inspiral and
merger of systems of compact binaries.  In the next chapter we discuss
the principles behind an ongoing experiment to directly observe
gravitational radiation.


% Todo:
% Discuss the Weyl tensor, if needed for NR
