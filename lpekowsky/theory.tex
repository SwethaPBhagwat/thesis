
\section{General Relativity}
\subsection{Elements of differential geometry}

An n-dimensional ($C^\infty$) manifold $\mathcal{M}$ is a set plus a
set of {\emph charts}, invertible maps $\{\phi_i\}$ which maps open
subsets of $\mathcal{M}$ to open subsets of $\mathcal{R}^n$ such that

\iffalse
\begin{list}
\item For all points $p \in \mathcal{M}$ there exists an $\phi_i$ 
such that $p$ is in the domain of $\phi_i$.
\item The composition $\phi_i \circ \phi_j^{-1}$ on the 
intersections of the domains of $\phi_i$ and $\phi_j$ is a
($C^\infty$) function from $\mathcal{R}^n \to \mathcal{R}^n$.
\end{list} 
\fi

Two natural structures are curves, which are maps from
$\mathcal{R}\to\mathcal{M}$, and scalar function which are maps from
$\mathcal{M}\to\mathcal{R}$.  Compositing a function $f$ with a curve
$\gamma(\lambda)$ gives a map from $\mathcal{R} \rightarrow
\mathcal{R}$ which may be differentiated in the usual way at a point
$p$


\begin{equation*}
\frac{d}{d \lambda} fi \big|_p = 
  \frac{d}{d\lambda} (f \circ \gamma) \big|_p
\end{equation*}


Expanding this in terms of a chart whose domain includes $p$ and then
applying the chain rule
 
\begin{align*}
\frac{d}{d \lambda} fi \big|_p &= 
 \frac{d}{d\lambda} ( (f \circ \phi^{-1}) \circ (\phi \circ
\lambda) ) \\
&= \frac{d(\phi^{-1} \circ \gamma)^\mu}{d\lambda} 
\frac{\partial (f \circ \phi^{-1}) }{\partial x^\mu} \big|_p \\
&= \frac{dx^\mu}{d\lambda} \partial_\mu f \big|_p
\end{align*}

where $x^\mu$ are the coordinates on $\mathcal{R}^n$.  Finally, since
the function $f$ is arbitrary we can define

\begin{equation}
\label{eq:tangent_vector}
\frac{d}{d\lambda} = \frac{dx^\mu}{d\lambda} \partial_\mu
\end{equation}

Geometrically taking the derivative gives the tangent vector to the
curve at the point $p$.  Equation~\ref{eq:tangent_vector} identifies
the set of such vectors with the set if directional derivatives,
taking the partial derivatives along the coordinates as the basis.
Henceforth this basis will be denoted both $\partial_\mu$ and
$\vec{e}_\mu$.

Note that the tangent to a curve is defined at the point $p$.  Each
point in the manifold posses its own space of tangent vectors defined
by equivalence classes of curves passing through it.  These spaces are 
distinct, which will be important in what follows.

We next define \emph{one-forms} as linear maps from vectors to
$\mathcal{R}$.  The set of one-forms at a point can be shown to form a
vector space, a natural basis for which can be obtained by requiring

\begin{equation*}
\vec{e}_\mu \tilde{\omega}^\nu = \delta_\mu^\nu
\end{equation*}

and the components of an arbitrary form $\omega$ in this basis may be
found by applying the form to the basis vectors

\begin{align*}
\omega(\vec{e}_\nu)
&= \tilde{\omega}^\mu \omega_\mu (\vec{e}_\nu) \\
&= \omega_\mu \tilde{\omega}^\mu (\vec{e}_\nu) \\
&= \omega_\mu \delta^\nu_\mu \\
&= \omega_\nu\\
\end{align*}

We can then build up arbitrary ${m \choose n}$ tensors as linear maps
from tensor products of $m$ vectors and $n$ one-forms to
$\mathcal{R}$.  The components of a tensor $T$ in some coordinates may
found by applying it to combinations of the basis vectors and 1-forms:

% T^..._... = 

\subsection{The metric tensor}

A particularly import tensor in General Relativity is the metric, a
${2 \choose 0}$ tensor that is symmetric ($g_{\mu\nu} = g_{\nu\mu}$
and non-degenerate (the determinant of $g$ taken as a matrix 
$|g_{\mu\nu}| \neq 0$.  The latter features makes it possible to
define the inverse metric $g^{\mu\nu}$ as

\begin{equation*}
g^{\mu_\rho} g_{\rho_\nu} = \delta^\mu_\nu
\end{equation*}

Given a vector $x^\mu$ the object $g_{\mu\nu} x^\mu$ maps another
vector to a real number, and is therefore a one-form.  The metric
therefore maps between the space of one-forms and the space of vectors
at each point.  Most importantly, the metric defines a notion of
distance on the manifold.  Infinitesimally

\begin{equation}
ds^2 = g_{\mu\nu} dx^\mu dx^\nu
\end{equation}



\subsection{Covariant derivatives}

Since vectors are only defined at a point we need additional structure
to define derivatives of vectors, as there is no natural way to
compare vectors that live in different spaces.  We seek an operator
$\nabla$ with the following properties

\iffalse
\begin{list}
\item Maps ${m \choose n}$ tensors to ${m \choose {n+1}}$ tensors.
This is so we may consider the directional derivative of a tensor $T$
along a vector $x$ as $x^\mu \nabla_\mu T$.
\item Reduces to partial differentiation when applies to a scalar
field.
\item Linear.
\item Satisfies the Leibniz rule, $\nabla(a b) = a\nabla b + b \nabla a$.
\end{list}
\fi

Such an operator applied to a vector field gives

\begin{equation*}
\nabla_\mu (x^\nu \vec{e}_\mu)
= (\partial x^\nu) \vec{e}_\mu + x^\nu (\nabla_\nu \vec{e}_\mu)
\end{equation*}

In a flat space in Cartesian coordinates the basis vectors do not
change and so the last term is zero.  But in a curved space, or even
flat space in non-Cartesian coordinates, they may.  However, the new
vector must be expressible as a linear combination of the original
basis vectors  The components are called {\emph connection
coefficients} and are denoted as $\Gamma^\rho_{\nu\mu}$ so

\begin{align}
\label{eq:covariant_derivative}
\nabla_\mu (x^\nu \vec{e}_\mu) &= 
(\partial x^\nu) \vec{e}_\mu + 
x^\nu \Gamma^\rho_{\nu\mu} \vec{e}_\rho \\
&= (\partial x^\nu + x^\rho \Gamma^\nu_{\mu\rho}) \vec{e}_\nu \\
\nabla_\mu x^\nu &= \partial x^\nu + x^\rho \Gamma^\nu_{\mu\rho}
\end{align}

In general relativity it is usually assumed that

\begin{equation}
\label{eq:torsion}
 \Gamma^\nu_{\mu\rho} =  \Gamma^\nu_{\rho\mu}
\end{equation}

which thus far has been borne out by experiment.  However, it is
possible to construct theories where this condition (``torsion-free'')
does not hold.

By considering the covariant derivative of a scalar constructed from a
one-form acting on a vector, $\nabla (x^\nu \omega_nu)$, it can be shown
that

\begin{equation*}
\nabla_\mu \omega_\nu = \partial_\mu \omega_\nu - 
\Gamma^\rho_{\mu\nu} \omega_\rho
\end{equation*}

The covariant derivative of a ${m \choose n}$ tensor generalizes this
and has a partial derivative term, $m$ positive therms in $\Gamma$ and
$n$ negative terms in $\Gamma$.

\section{Parallel Transport}
\label{sec:parallel}

Covariant differentiation provides a way to ``move a vector without
changing it.''  We can {\emph parallel transport} a vector $v^\mu$
infinitesimally along a curve whose tangent vector is $u^\nu$ by
requiring

\begin{equation*}
u^\nu \nabla_\nu v^\mu = 0
\end{equation*}

As an example of such transport, consider an arrow on the equator of
the Earth pointing towards the north pole.  This arrow can be carried
halfway around the equator without rotating it, so it ends up on the
other side of the globe, still pointing north.  If the vector is then
parallel transported northward to the pole and then continued until it
returns to its starting point it will return pointing south.  Although
the vector was never rotated locally it has returned rotated.  This is
an indication that the underlying space is curved.

Of particular interest is the case where a vector is
parallel-transported along itself

\begin{equation*}
0 = v^\mu \nabla_mu v^\nu 
= v^\mu (\partial_\mu v^\nu + \Gamma^\nu_{\mu\rho} v^\rho)
\end{equation*}

Now consider a curve $x(\lambda)$ such that $v$ is the tangent to this
curve, $v^\mu = d x^\mu/d\lambda$, then

\begin{equation}
\label{eq:geodesic}
\frac{d x^\mu}{d\lambda}
  \frac{\partial }{\partial x^\mu}
  \frac{d x^\nu}{d\lambda}  
+ \Gamma^\nu_{\mu\rho} 
\frac{d x^\mu}{d\lambda}
\frac{d x^\rho}{d\lambda} = 0
\end{equation}

\subsection{The Christoffel Symbols}

If we now require that scalars do not change under parallel transport
we have, for arbitrary vectors $u^\alpha, v^\beta$ and $x^\mu$

\begin{align*}
0 &= x^\mu \nabla_\mu (g_{\alpha\beta} u^\alpha v^\beta) \\
&= x^\mu (\nabla_\mu g_{\alpha\beta}) u^\alpha v^\beta
+ g^{\alpha\beta} (x^\mu \nabla_\mu u^\alpha) v^\beta
+ g^{\alpha\beta} u^\alpha (x^\mu \nabla_\mu v^\beta)
\end{align*}

We can now specialize such that $u^\alpha, v^\beta$ are constant
vectors and so the last two terms vanish, which implies that

\begin{equation}
\label{eq:metric_compatibility}
\nabla_\mu g_{\alpha\beta} = 0
\end{equation}

Equations~\ref{eq:metric_compatibility} and~\ref{eq:torsion} together
fix the connection coefficients in terms of the metric:

\begin{equation}
\Gamma^{\rho}{\mu\nu}
= \frac{1}{2} g^{\rho\sigma}\left[
\partial_\nu g_{\mu\sigma}
+ \partial_\mu g_{\nu\sigma}
- \partial_\sigma g_{\mu\nu}
\right]
\end{equation}


\subsection{The Riemann Tensor}

We now generalize the example given in section~\ref{sec:parallel},
and ask how a vector $A^\mu$ changes as it is parallel-transported
around an infinitesimal parallelogram with sides defined by the
vectors $A^\mu$ and $B^\nu$.  It can be shown that this is equivalent
to asking how covariant derivatives fail to commute.  The result must
be linear in the vectors and so we may write

\begin{equation}
\label{eq:riemann_def}
\left[\nabla_\mu \nabla_\nu - \nabla_\nu \nabla_\mu\right] A^\rho
= R^\rho_{\sigma\mu\nu} A^\sigma
\end{equation}

which defines the {\emph Riemann tensor} $R$.  A number of properties
follow from this definition (which are either obvious or may be proven
by substituting the definition of the covariant derivative
(eqn.~\ref{eq:covariant_derivative}).

First, the symmetry properties

\begin{equation}
\label{eq:symmetries}
R_{\rho\sigma\mu\nu}
= -R_{\sigma\rho\mu\nu}
= -R_{\rho\sigma\nu\mu}
= R_{\mu\nu\rho\sigma}
\end{equation}

which in turn imply

\begin{align}
R^\rho_{[\sigma\mu\nu]} = 0
\end{align}

Second, the Bianchi identity,

\begin{equation}
\label{eq:bianchi}
R_{\rho\sigma\mu\nu;\alpha}
+R_{\rho\sigma\nu\alpha;\mu}
+R_{\rho\sigma\alpha\mu;\nu} = 0
\end{equation}

We may now generalize equation~\ref{eq:riemann_def} and ask how an
arbitrary tensor changes after being parallel-transported 
around a loop

\begin{equation}
\label{eq:higher_order_riemann}
\left[\nabla_\mu \nabla_\nu - \nabla_\nu \nabla_\mu\right] 
B^{\rho_1 \rho_2 \ldots \rho_n}
= - R^{\rho_1}_{\sigma \mu\nu} B^{\sigma \rho_2 \ldots \rho_n}
- R^{\rho_2}_{\sigma \mu\nu} B^{\rho_1 \sigma \ldots \rho_n}
- \ldots -
- R^{\rho_n}_{\sigma \mu\nu} B^{\rho_1 \rho_2 \ldots \sigma }
\end{equation}

This result is sensible; it is linear in $B$, has two free upper
indices, and reduces to the correct expression when $B$ is a vector.
This result may be proved by expanding 

\begin{equation*}
\left[\nabla_\mu \nabla_\nu - \nabla_\nu \nabla_\mu\right] 
(\vec{e}_\rho \otimes \vec{e}_\sigma)
\end{equation*}

and then proceeding by induction.  

The symmetries of the Riemann tensor imply that there is, up to sign,
only one non-trivial contraction

\begin{equation}
R_{\mu\nu} = R^\sigma_{\mu\sigma\nu}
\end{equation}

which defines the {\emph Ricci tensor}.  This may be contracted again

\begin{equation}
R = R^\mu_\mu
\end{equation}

to obtain the \emph{Ricci scalar}.


Contract the Bianchi identity twice

\begin{align*}
0 &=
g^{\rho\mu} g^{\sigma\nu}
\left(R_{\rho\sigma\mu\nu;\alpha}
+R_{\rho\sigma\nu\alpha;\mu}
+R_{\rho\sigma\alpha\mu;\nu}\right) \\
&= g^{\rho\mu} g^{\sigma\nu}
\left(\nabla_\alpha R_{\rho\sigma\mu\nu}
+ \nabla_\mu R_{\rho\sigma\nu\alpha}
+ \nabla_\nu R_{\rho\sigma\alpha\mu}\right) \\
&= g^{\sigma\nu}
\left(\nabla_\alpha R_{\sigma\nu}
+ \nabla^\rho R_{\rho\sigma\nu\alpha}
+ \nabla_\nu R^\mu_{\sigma\alpha\mu}\right) \\
\end{align*}

Using the symmetries of the Riemann tensor (eqn.~\ref{eq:symmetries})
this can be written

\begin{align*}
0 &= g^{\sigma\nu}
\left(\nabla_\alpha R_{\sigma\nu}
- \nabla^\rho R_{\sigma\rho\nu\alpha}
- \nabla_\nu R^\mu_{\sigma\mu\alpha}\right) \\
&= g^{\sigma\nu}
\left(\nabla_\alpha R_{\sigma\nu}
- \nabla^\rho R_{\sigma\rho\nu\alpha}
- \nabla_\nu R_{\sigma\alpha}\right) \\
&= \nabla_\alpha R
- \nabla^\rho R_{\rho\alpha}
- \nabla^\sigma R_{\sigma\alpha} \\
\end{align*}

Relabeling the dummy indices gives

\begin{equation*}
0 = \nabla_\alpha R - 2 \nabla^\rho R_{\rho\alpha}
\end{equation*}

Then by metric compatibility

\begin{align*}
0 &= g_{\rho\alpha} \nabla^\rho R - 2 \nabla^\rho R_{\rho\alpha} \\
&= \nabla^\rho (g_{\rho\alpha} R) - 2 \nabla^\rho R_{\rho\alpha} \\
&= \nabla^\rho \left(g_{\rho\alpha} R - 2 R_{\rho\alpha} \right)
\end{align*}

This motivates the definition of the \emph{Einstein Tensor} as

\begin{equation}
\label{eq:einstein_tensor}
G_{\mu\nu} = R_{\mu\nu} - \frac{1}{2} g_{\mu\nu} R
\end{equation}

the previous implies this is divergentless

\begin{equation*}
\nabla^\nu G_{\mu\nu} = 0
\end{equation*}

Note that $G$ is also symmetric, $G_{\mu\nu} = G_{\nu\mu}$.


The matter and energy content of a region is described by the stress-
energy tensor $T_\mu\nu$ where each component is ``the flow of $\mu$
momentum in the $\nu$ direction.''  For example, the $0,0$ component
is energy density and the $0,i$ components are the $i^\mathrm{th}$.

Conservation of energy requires that the difference in momentum
across each face of a cube be balanced by a change of energy,
within the cube,

\begin{equation*}
\partial_t \rho = \partial_i p^i
\end{equation*}

In terms of the stress-energy tensor this becomes

\begin{equation*}
0 = - \nabla^0 T_{00} \nabla^i T_{0i} = 0
= \nabla^\nu T_{0 \nu}
\end{equation*}

However the time direction is not uniquely specified, so this 
must generalize to

\begin{equation*}
\nabla^\nu T_{\mu\nu} = 0
\end{equation*}

That is, $T$ is also divergentless, like $G$, and like $G$ it is also
symmetric.  It is therefore reasonable to suggest the ansatz

\begin{equation*}
G_{\mu\nu} \propto T_{\mu\nu}
\end{equation*}

Requiring agreement with Newton's law of gravity in the appropriate
low-energy limit ($T_{00} \gg$ all other components) fixes the
constant of proportionality and gives us \emph{Einstein's field
equation}

\begin{equation}
G_{\mu\nu} = 8\pi T_{\mu\nu}
\end{equation}


\section{Gravitational radiation}

We begin with Einstein's equation in empty space,

\begin{equation*}
G_{\mu\nu} = R_{\mu\nu} - \frac{1}{2} g_{\mu\nu} R = 0
\end{equation*}

Taking the trace

\begin{equation*}
g^{\mu\nu} R_{\mu\nu} - \frac{1}{2} g^{\mu\nu} g_{\mu\nu} R 
= R - 2 R = -R = 0
\end{equation*}

Substituting this result back into Einstein's equation shows that in
empty space $R_{\mu\nu} = 0$.

We also have, in empty space, using the Bianchi identity and
symmetries of the Riemann tensor:

\begin{align}
\label{eq:divergence_in_empty_space}
R^\mu_{\beta\gamma\delta;\mu} &= 
-R^\mu_{\beta\delta\mu;\gamma}  
-R^\mu_{\beta\mu\gamma;\delta} \\ 
&= R^\mu_{\beta\mu\delta;\gamma}  
-R^\mu_{\beta\mu\gamma;\delta} \\
&= R_{\beta\delta;\gamma}  
-R_{\beta\gamma;\delta} \\
&= 0
\end{align}

We next consider the application of the wave operator to the Riemann
tensor.  From the Bianchi identity (eqn.~\ref{eq:bianchi}) this becomes

\begin{equation*}
\label{eq:wave_expanded}
g^{\mu\nu} R_{\alpha\beta\gamma\delta;\mu\nu}
= - g^{\mu\nu}
\left[R_{\alpha\beta\delta\mu;\gamma\nu}
+ R_{\alpha\beta\mu\gamma;\delta\nu} \right]
\end{equation*}

Consider the first term on the right-hand side:

\begin{align*}
g^{\mu\nu} R_{\alpha\beta\delta\mu;\gamma\nu}
&= g^{\mu\nu} R_{\alpha\beta\delta\mu;\nu\gamma}
+ g^{\mu\nu} R_{\alpha\beta\delta\mu;\gamma\nu}
- g^{\mu\nu} R_{\alpha\beta\delta\mu;\nu\gamma} \\
&= g^{\mu\nu} R_{\alpha\beta\delta\mu;\nu\gamma}
+ g^{\mu\nu} 
\left[\nabla_\nu,\nabla_\gamma\right] R_{\alpha\beta\delta\mu}
\end{align*}

The first term vanishes by
equation~\ref{eq:divergence_in_empty_space}.  The second term involves
products of $R$ by~\ref{eq:higher_order_riemann}.  The second term on
the right in equation~\ref{eq:wave_expanded} has the same form.

We now specialize to the case where $R$ is small so terms
involving multiple factors can be neglected.  This is equivalent to
considering $R$ as a field on a flat background.  This gives

\begin{equation}
\label{eq:riemann_wave}
g^{\mu\nu}
R_{\alpha\beta\gamma\delta;\mu\nu}
=
\Box R_{\alpha\beta\gamma\delta;\mu\nu}
= 0
\end{equation}

That is, each component of the Riemann tensor independently 
satisfies the vacuum wave equation.

\section{The LIGO Gravitational Wave Detectors}
\label{sec:ligo_detectors}

% Todo:
% Discuss the Weyl tensor, if needed for NR

