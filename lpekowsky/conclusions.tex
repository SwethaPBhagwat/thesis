We have compared the post-Newtonian waveforms currently used to
search for gravitational waves against a waveform produced by 
a full numeric simulation.  Based on these studies we have identified
several factors which can improve the overlaps between the two, and
hence between analytic waveforms and the signals we expect to see.

We have tested these optimizations in the first NINJA project, which
suggests that the improvement in overlaps carries through to improved
SNR in a full search.  This in turn implies that by implementing these
changes we may be able to enhance the significance of signals in LIGO
and Virgo.

We will further test these ideas in NINJA-2, which is currently
on-going.  In order to maximize the utility of NINJA-2 and correct
for some of the limitations of NINJA-1 we have performed extensive
validations of the hybrid pN/NR waveforms.  A NINJA-2 data set is now
available, and the unmodified CBC pipelines have been run.  The
results contain unexpected features which will need to be understood
before we can use these results as a baseline against which to test
our modifications.  NINJA-2 will also enable many other studies.  In
particular we plan to use it to determine the optimum total mass at
which the low mass and high mass searches should transition.

We have also shown how the tools and methods of the CBC search can be
applied to detchar and used to identify noisy times in the detectors
and remove them from analysis.  The results from these detchar studies
also proved to be tremendously useful in performing follow up studies
of a candidate gravitational wave event, although the event ultimately
turned out to be a blind injection.

With the end of S6 the initial LIGO era has ended.  The
interferometers are currently undergoing major upgrades in preparation
for advanced LIGO, to start around 2015.  At the design sensitivity
advanced LIGO will have ten times the range of initial LIGO.  This new
era will present new challenges to data analysts, however the basic
structure of the current templated, matched filter search will
continue to be used.   It will therefore be useful to continue
optimizing the search against the initial LIGO noise curve and initial
LIGO data, as NINJA-2 will do.  In addition it is likely future NINJA
projects will begin to use simulated advanced LIGO noise.  Similarly,
daily ihope will continue to be run in advanced LIGO, and perhaps even
earlier, during engineering runs as the new instruments are being
tested.  We expect that continued studies with numeric waveforms and
the use of CBC-specific detector characterization will contribute
towards making the potential of LIGO as a new window to the Universe a
reality.


