% $Id$

One of the most interesting outstanding problems in astronomy is the
\emph{dark matter} problem. Dark matter is so called because it has eluded
detection through its emission or absorbsion of electomagnetic radiation. Our
knowlege of its existance comes from the gravitational interaction of dark
matter with luminous matter in the universe. There are two major theories
proposed to explain the origin of dark matter. The first is that they are
weakly interacting massive particles (WIMPs). WIMPs are a concequency of
supersymmertry and are outside the scope of this thesis; we refer the
interested reader to \cite{XXX}.  In this thesis, we are concered with the
proposal that a some component of the dark matter consists of massive
astrophysical compact halo objects (MACHOs), in particular the possibility
that MACHOs may be primordial black holes (PBHs) formed in the early universe.
In this chapter, we review the evidence for dark matter in the universe and
describe how it may be possible to detect a proposed population of PBHs in the
galactic halo.

\section{Dark Matter In The Galactic Halo}

We stated above that dark matter is detected only by its gravitational
interaction with luminous matter. The origin of the dark matter problem comes
from the study of galactic rotation curves; measurements of the velocities of
luminous, baryonic matter in the disks of spiral galaxies as a function of
galactic radius.  Let us consider a simple rotational model for the disk of a
spiral galaxy.  Consider a star with mass $m_s$ orbiting at radius $r$ in the
outer part of the galaxy's disk. Newtonian dynanics tells us that if the mass
inside radius $r$ is $m_g$ then
\begin{equation}
\frac{Gm_g m_s}{r^2} = \frac{m_s v_s^2}{r}
\label{eq:newtongalaxy}
\end{equation}
where $v_s$ is the velocity of the star and $G$ is the gravitational constant. 
If we assume that as we increse $r$, the change in the $m_g$ is negligable.
This is a reasonable assumption towards the edge of the disk, since the mass
is concentrated towards the center of the galaxy.  We can see from equation
\ref{eq:newtongalaxy} that we would expect that the velocity of stars at the
edge of the galactic disk would fall off as 
\begin{equation}
v_s \propto \frac{1}{\sqrt{r}}
\end{equation}
and so a typical galactic rotation curve would fall off as $r^{-1/2}$.
Galactic rotation curves can be measuring using the doppler shift of the
$21$~cm hydrogen line and has been for several galaxies. It is found that the
rotation curves to not fall off as expected. Instead they are flat out to the
edge of the visible matter in the disk, as shown in figure~\ref{f:rotcurves}. 
This surprising result suggests that most of the matter in galaxies does not
emit light but is gravitationally coupled to the visible matter. Rotation
curves suggest that around 80\%--90\% of the matter in spiral galaxies is in
the form of dark matter.

Since we have, so far, been unable to observe the dark matter component of the
galaxy, we must infer its distribution and density from the indirect
obesrvations and numerical simulations of galaxy formation. Consider the
evolution of a spatial distribution of baryonic matter. Baryonic matter is the
typical luminous matter that we see composed of neutrons, protons, electrons
and other baryons. If the distribution is initially spherical and rotating
with some angular momentum, $L$, then over time the matter will lose energy
through inelastic collisions. Since the angular momentum of the system must be
conserved, howeverm the initial distribution will collapse to a rotating disk.
This toy model is typical of the formation of galaxtic disks from baryonic
matter. On the other hand, if the initially spherical distribution is composed
of dark matter, the only collisions that the population will undergo are
elastic, because the dark matter is weakly interacting. As a result of this,
if the dark matter is initially distributed uniformly with and isotropic, it
will maintain this distribution as it evolves.

We may huess that the dark matter halos are at least as old os the visible
matter as they are much more massive. Since the dark matter is gravitationally
bound to the visible matter in the disk, it is reasonable to assume that the
visible disk and dark halo are in thermal equilibrium with some characteristic
mean square velocity. Since we do not expect a spherical dark matter halo to
collapse to a disk, the simplest possible assumption is that the dark halo is
a spherical, isothermal distribution of dark matter. We may ask what the
dark matter density is in the neighbourhood of the earth. If we assume that
the density of the dark matter is $\rho(r)$ then for a spherical halo the mass
within a thin shell of width $dr$ is
\begin{equation}
dM(r) = 4\pi r^2 \rho(r)\, dr.
\label{eq:simplehalodensity}
\end{equation}
Using Newtonian dynamics, the velocity, $v$, of a particle of mass $m$ at
radius $r$ is
\begin{equation}
\begin{split}
\frac{GM(r)m}{r^2} &= \frac{mv^2}{r} \\
v^2 &= \frac{GM(r)}{r}.
\end{split}
\end{equation}
The galactic rotation curves tell us that the velocity is independent of the
radius, so
\begin{equation}
M(r) = \frac{v^2r}{G}.
\end{equation}
Differentiating this with respect to $r$ and substituting the result into
equation \ref{eq:simplehalodensity}, we obtain
\begin{equation}
\frac{dM(r)}{dr} = \frac{v^2}{G} = 4\pi r^2\rho(r)
\end{equation}
which gives
\begin{equation}
\rho(r) = \frac{v^2}{4\pi r^2 G}.
\label{eq:simplehalodensity2}
\end{equation}
Since the dark and visible matter are in thermal equilibrium, we may use the
measrued rotational velocity of local stars about the galactic center as the
velocity of the dark matter. The earth is approximately $8$~kPc from the
galactic center and the rotational velocity of objects at this radius is
$v\sim 200\mathrm{km.s}^-1$. Using these values in equation
(\ref{eq:simplehalodensity2}), we obtain a value of
\begin{equation}
\rho(r_E) = 7.6 \times 10^{-25}\, \mathrm{g}.\mathrm{cm}^{-3}.
\end{equation}
Typical estimates for the halo density range from $2\times
10^{-25}\,\mathrm{g}.\mathrm{cm}^{-3}$ to
$10^{-24}\,\mathrm{g}.\mathrm{cm}^{-3}$\cite{XXX}. This is approximately
$0.01\,M_\odot.\mathrm{pc}^{-3}$.

\section{Gravitational Microlensing}

\section{Gravitational Waves Binary Black Hole MACHOs}

Motivation for search: dark matter problem in the galaxy, microlensing
results, Nakamura proposal that MACHOs may be (B)BHs. Predicted rate is
$5\times10^{-2}\times2^{\pm 1}$ events/yr/galaxy is higher than BNS rate.
Waveforms well modelled, same pipeline as S2 BNS. Brief description of sciruns
with reference to instument paper and S2 BNS paper.

\section{Binary Black Hole MACHO Population Models}

Review Ioka et al paper on BBH formation in the early universe. Density of
local dark matter. Review population models used and make some plots of MACHO
distribution.

We start from the (flattened) spherical halo model
