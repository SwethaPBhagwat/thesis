% $Id$

One of the most interesting problems in astrophysics is that of \emph{dark
matter}. Dark matter is material in the universe that interacts
gravitationally with luminous matter, but has so far eluded direct
observation. 


There have been several  the nature of dark matter,
chief among these are WIMPs and MACHOs. WIMPs are \emph{weakly interacting
massive particles}, supersymmertic particles produced as a relic of the big
bang, and are outside the scope of this thesis\footnote{We refer the
interested reader to \cite{Griest:1995gs} for a review of the nature of dark
matter.}. Here we are concerned with the proposeal that a component of the
dark matter is in the form of \emph{massive astrophysical compact halo
objects} or MACHOs\cite{Griest:1990vu}. Possible candidates for MACHOs are
brown dwarfs, Jupiters or black hole remnants from an early population of
stars.

of WIMPs or MACHOs (or some combination of the two).  WIMPs, , are outside the scope of this
thesis
In this chapter, we review the evidence for a form of dark matter
known as massive astrophysical compact halo objects (MACHOs)\cite{x}


Dark matter is so called because it has eluded detection through its
emission or absorbsion of electomagnetic radiation. Our knowlege of its
existance comes from the gravitational interaction of dark matter with
luminous matter in the universe. There are two major theories proposed to
explain the origin of dark matter. The first is that they are weakly
interacting massive particles (WIMPs). WIMPs are a concequency of
supersymmertry and are outside the scope of this thesis; we refer the
interested reader to \cite{XXX}.  In this thesis, we are concered with the
proposal that a some component of the dark matter consists of massive
astrophysical compact halo objects (MACHOs), in particular the possibility
that MACHOs may be primordial black holes (PBHs) formed in the early universe.
In this chapter, we review the evidence for dark matter in the universe and
describe how it may be possible to detect a proposed population of PBHs in the
galactic halo.

\section{Dark Matter In The Galactic Halo}

Dark matter is detected by its gravitational interaction with luminous matter.
Strong evidence for the presence of dark matter in the universe comes from the
study of galactic rotation curves; measurements of the velocities of luminous
matter in the disks of spiral galaxies as a function of galactic radius.  Let
us consider a simple rotational model for the disk of a spiral galaxy.
Consider a star with mass $m_s$ orbiting at radius $r$ in the outer part of
the galaxy's disk. Newtonian dynanics tells us that if the mass inside radius
$r$ is $m_g$ then
\begin{equation}
\frac{Gm_g m_s}{r^2} = \frac{m_s v_s^2}{r}
\label{eq:newtongalaxy}
\end{equation}
where $v_s$ is the velocity of the star and $G$ is the gravitational constant. 
If we assume that as we increse $r$, the change in the $m_g$ is negligable.
This is a reasonable assumption towards the edge of the disk, since the mass
is concentrated towards the center of the galaxy.  We can see from equation
\ref{eq:newtongalaxy} that we would expect that the velocity of stars at the
edge of the galactic disk would fall off as 
\begin{equation}
v_s \propto \frac{1}{\sqrt{r}}
\end{equation}
and so a typical galactic rotation curve would fall off as $r^{-1/2}$.
Galactic rotation curves, measuring using the doppler shift of the
$21$~cm hydrogen line, have been measured for several galaxies. It is found
that the rotation curves to not fall off as expected. Instead they are flat
out to the edge of the visible matter in the disk, as shown in
figure~\ref{f:rotcurves}.  This surprising result suggests that most of the
matter in galaxies does not emit light but is gravitationally coupled to the
visible matter. Rotation curves suggest that around 80\%--90\% of the matter
in spiral galaxies is in the form of dark matter\cite{Sancisi:1987}.

Since we have been unable to observe the dark matter component of the galaxy,
we must infer its distribution and density from the indirect obesrvations and
numerical simulations of galaxy formation. Consider the evolution of a spatial
distribution of baryonic matter. Baryonic matter is the typical luminous
matter that we see composed of neutrons, protons, electrons and other baryons.
If the distribution is initially spherical and rotating with some angular
momentum, $L$, then over time the matter will lose energy through inelastic
collisions. Since the angular momentum of the system must be conserved,
howeverm the initial distribution will collapse to a rotating disk.  This toy
model is typical of the formation of galaxtic disks from baryonic matter. On
the other hand, if the initially spherical distribution is composed of dark
matter, the only collisions that the population will undergo are elastic,
because the dark matter is weakly interacting. As a result of this, if the
dark matter is initially distributed uniformly with and isotropic, it will
maintain this distribution as it evolves.

We may guess that the dark matter halos are at least as old os the visible
matter as they are much more massive. Since the dark matter is gravitationally
bound to the visible matter in the disk, it is reasonable to assume that the
visible disk and dark halo are in thermal equilibrium with some characteristic
mean square velocity. Since we do not expect a spherical dark matter halo to
collapse to a disk, the simplest possible assumption is that the dark halo is
a spherical, isothermal distribution of dark matter. We may ask what the
dark matter density is in the neighbourhood of the earth. If we assume that
the density of the dark matter is $\rho(r)$ then for a spherical halo the mass
within a thin shell of width $dr$ is
\begin{equation}
dM(r) = 4\pi r^2 \rho(r)\, dr.
\label{eq:simplehalodensity}
\end{equation}
Using Newtonian dynamics, the velocity, $v$, of a particle of mass $m$ at
radius $r$ is
\begin{equation}
\begin{split}
\frac{GM(r)m}{r^2} &= \frac{mv^2}{r} \\
v^2 &= \frac{GM(r)}{r}.
\end{split}
\end{equation}
The galactic rotation curves tell us that the velocity is independent of the
radius, so
\begin{equation}
M(r) = \frac{v^2r}{G}.
\end{equation}
Differentiating this with respect to $r$ and substituting the result into
equation \ref{eq:simplehalodensity}, we obtain
\begin{equation}
\frac{dM(r)}{dr} = \frac{v^2}{G} = 4\pi r^2\rho(r)
\end{equation}
which gives
\begin{equation}
\rho(r) = \frac{v^2}{4\pi r^2 G}.
\label{eq:simplehalodensity2}
\end{equation}
Since the dark and visible matter are in thermal equilibrium, we may use the
measrued rotational velocity of local stars about the galactic center as the
velocity of the dark matter. The earth is approximately $8$~kPc from the
galactic center and the rotational velocity of objects at this radius is
$v\sim 200\mathrm{km\,s}^-1$. Using these values in equation
(\ref{eq:simplehalodensity2}), we obtain a value of
\begin{equation}
\rho(r_E) = 7.6 \times 10^{-25}\, \mathrm{g}\,\mathrm{cm}^{-3}.
\end{equation}
Clearly equation (\ref{eq:simplehalodensity2}) cannot be true at all radii, as
it suggests that $\rho \rightarrow \infty$ as $r \rightarrow \infty$. It is
observered that rotation velocities fall to zero as $r\rightarrow 0$. The data
at small $r$ is consistent with the dark matter having a constant \emph{core
density}, $\rho_c$ within a \emph{core radius}, $a$. The halo density then
becomes
\begin{equation}
\rho(r) = \frac{\rho_c}{1 + \left(\frac{r}{a}\right)^2}.
\label{eq:simplehalodensity3}
\end{equation}.
The values of $\rho_c$ and $a$ are obtained from fitting measured galactic
rotation curves to equation (\ref{eq:simplehalodensity3}) using data near the
galactic center. 

There is, in fact, no evidence to suggest that halos are
exactly spherical; the halo density may be flattened\cite{Rix:1996}. For a
flattened halo the dark matter density becomes
\begin{equation}
\rho(R,z) = \frac{\rho_c r^2_c}{a^2 + R^2 + z^2/q^2}
\label{eq:simplehalodensity4}
\end{equation}
where $R$ and $z$ are galactocentric cylindrical coordinates and $q$ is a
parameter that describes the flattening of the halo. Careful modelling of the
Galaxy\cite{1995ApJ...449L.123G}, suggests that the local halo density is
\begin{equation}
\rho_(r_E) = 9.2_{-3.1}^{+3.8} \times 10^{-25}\,  \mathrm{g}\,\mathrm{cm}^{-3}
\end{equation}
or approximately $0.01\,M_\odot.\,\mathrm{pc}^{-3}$.

\section{Gravitational Microlensing}

\section{Gravitational Waves Binary Black Hole MACHOs}

Motivation for search: dark matter problem in the galaxy, microlensing
results, Nakamura proposal that MACHOs may be (B)BHs. Predicted rate is
$5\times10^{-2}\times2^{\pm 1}$ events/yr/galaxy is higher than BNS rate.
Waveforms well modelled, same pipeline as S2 BNS. Brief description of sciruns
with reference to instument paper and S2 BNS paper.

\section{BBHMACHO Population Model}

The goal of this thesis is to search for the gravitational waves from binary
black hole MACHOs described in the previous section. In the absence of a
detection, however, we wish to place an \emph{upper limit} on the rate of
BBHMACHO inspirals in the galaxy. we can then compare to the theortical rate
of $5\times 10^{-2}\times2^{\pm 1} \mathrm{yr}^{-1}\,\mathrm{MWEG}^{-1}$
decribed in the previous section. We will see later that, in order to
determine an upper limit on the rate, we need to simulate the spatial
distibution of BBHMACHOs in the halo. We assume that the distribution of
BBHMACHOs in galactocentric cylindrical coordinates, $R,\theta,z$, follows the
halo density given by equation (\ref{eq:simplehalodensity4}). That is
\begin{equation}
\rho(r) \propto \frac{1}{a^2 + R^2 + z^2/q^2}
\end{equation}
where $a$ is the halo core radius and $q$ is the halo flatening parameter. If
wemake the coordinate change $z/q \rightarrow z$, then we may obtain a
probability density function, $f(R,\theta,z)$ for the spatial distribuation of
the MACHOs, given by
\begin{equation}
f(R,\theta,z)\,R dR d\theta dz \propto \frac{1}{a^2 + R^2 + z^2/q^2}\,R dR d\theta dz.
\end{equation}
Recall that for a probability density function, $f(x)\,dx$, the cumulative
distribution, $F(X)$, is given by
\begin{equation}
F(X) = \int_{-\infty}^X f(x)\, dx.
\end{equation}
Now $0 \le F(X) \le 1$


To normalize this PDF, we note that
\begin{equation}
\int_0^{R_{\mathrm{max}}}\int_0^{2\pi}\int{-\infty}^{\infty} \frac{1}{a^2 + R^2 + z^2/q^2}\,R\, dR\, d\theta\, dz = 1
\label{eq:normpdf}
\end{equation}
where $R_{\mathrm{max}}$ is the size of the halo.  To evaluate the integral on
the left hand side, we make the coordinate change
\begin{equation}
\begin{split}
R &= r \cos \varphi \\
z &= r \sin \varphi \\
\end{split}
\end{equation}
where
\begin{equation}
\begin{split}
\tan \varphi &= \frac{z}{R} \\
r^2 &= R^2 + z^2 \\
\end{split}
\end{equation}
and so equation (\ref{eq:normpdf}) becomes
\begin{multline}
\int_0^{R_\mathrm{max}}\int_0^{2\pi}\int_{-\infty}^{\infty} \frac{1}{a^2 + R^2 + z^2/q^2}\,R\, dR\, d\theta\, dz \\
= \int_0^{R_{\mathrm{max}}}\int_0^{2\pi}\int_{-\frac{\pi}{2}}^{\frac{\pi}{2}} \frac{r\cos\varphi}{a^2+r^2}\,r\, dr \, d\theta\, d\varphi \\
= \int_0^{2\pi}\, d\theta 
\int_{-1}^{1} d\sin\varphi \int_0^{R_{\mathrm{max}}} \frac{r^2}{a^2+r^2}\,r dr \\
= \left[\theta\right]_0^{2\pi} \left[u\right]_{-1}^1 \left[r -
a\arctan\left(\frac{r}{a}\right)\right]_0^{R_{\mathrm{max}}} = 1.
\end{multline}
Notice that in the 






