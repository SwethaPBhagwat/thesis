The direct detection of gravitational waves promises to usher in a new era of astronomy. The \ac{GW} spectrum represents an entirely new window on the universe, independent of, and complimentary to, electromagnetic (EM) radiation. Gravitational waves can be used to directly probe objects unobservable by EM telescopes; e.g., the properties of black holes, the equation of state of neutron stars, and the state of the universe prior to the emission of the cosmic microwave background. Joint GW and EM observations offer more possibilities, such as understanding the progenitors of short-hard gamma-ray bursts (GRBs) and measuring the expansion of the universe. The GW spectrum would also give us insight into the physics of strong field gravity and numerical solutions of the Einstein equations, as well as provide a test for alternative theories of gravity \cite{SathyaSchutz}.

The U.S. \ac{LIGO} and the French-Italian Virgo interferometer are seeking to make the first direct detections of gravitational waves. \ac{LIGO} has just completed its sixth science run (S6) and is preparing for the Advanced LIGO (aLIGO) era, which will begin in 2014. Virgo, currently in its third science run (VSR3), will be upgraded on the same schedule as aLIGO. This thesis focuses on the search for compact binary coalescences (CBCs) with a total mass $< 35\,\Msun$ and a component mass $\mathrm{\geq 1M_\odot}$ using LIGO and Virgo data. This search includes binary neutron stars (BNS), binary black holes (BBH), and binaries containing a neutron star and a black hole (NSBH).

Coalescing compact binaries with a total mass $< 35\,\Msun$ have a number of properties that make them promising candidates for detection by LIGO and Virgo. As the binaries' components spiral into each other, they emit gravitational waves that pass through the sensitive band of the LIGO and Virgo detectors. These waves can be modeled using a combination of post-Newtonian approximations and numerical solutions. Knowing the morphology of the waveform provides a number of advantages, as it allows us to use match filtering and signal-based vetoes. The detectors have stationary, Gaussian noise that is intrinsic to the detector, as well as transient ``glitches" which come from environmental and instrumental sources. Match filtering is the best method to find weak signals in the stationary noise since the signal-to-noise ratio (SNR) grows with the square root of the number of cycles \cite{SathyaSchutz}. For non-Gaussian glitches, signal-based vetoes provide a powerful way to separate noise from candidates. Additionally, by utilizing all the detectors in the LIGO-Virgo network, we can perform coincidence tests to ensure that triggers from different observatories are consistent.

In this thesis we detail how searches for ``low mass" \acp{CBC} are carried out and we give some recent results from this search. In Chapter \ref{ch:theory} we review the theory behind gravitational waves, showing how they arise in general relativity, and the basics of the \ac{LIGO}/Virgo interferometers. In Chapter \ref{ch:pipeline_principles} we review some of the principles of matched filtering, and how various statistical tests are used to search the data. Next, in Chapter \ref{ch:far}, we detail how false alarm rates are calculated for \ac{GW} triggers. In Chapter \ref{ch:ihope_pipeline} we step through the pipeline used to search for \acp{CBC}, which makes use of the methods presented in the Chapters \ref{ch:pipeline_principles} and \ref{ch:far}. Chapter \ref{ch:s5_results} shows a paper published by the \ac{LSC} giving the results from a $6$ month analysis of \ac{S5} data. We next give a detailed examination of the analysis of \ac{S6}, including tuning choices made and preliminary results from that search. In Chapter \ref{ch:ligo_south} we present an analysis of the \ac{S5} and \ac{S6} data that supports that we can detect \acp{GW} at expected rates in Advanced \ac{LIGO}. Finally, in Chapter \ref{ch:future_developments} we present some future directions for the pipeline.

%All of these techniques are needed as separating noise from signal can be difficult. Even after applying signal-based vetoes and coincidence tests, non-Gaussian glitches can mimic gravitational waves, thereby causing ``false alarms." Such glitches are detrimental to our search; they increase background noise, decreasing the detectors' sensitivities. The best way to suppress these false alarms is to classify them so that they may be vetoed or, better yet, so that the instruments may be fixed to remove them.

%LIGO's fifth science run (S5), which ran from November 2005 to October 2007, was broken up into three blocks for analysis: the first calendar year (the ``S5 first year search") \cite{S51yr}, the six months between November 2006 and May 2007 (the ``12-18 month search") \cite{12to18}, and the joint LIGO-Virgo period from May to November 2007 (the ``LVC search") \cite{LVCsearch}. I joined LIGO in January of 2008 and began by helping Drew Keppel of Caltech (now at AEI-Hannover) with the analysis of the first year data. I then took a leading role in analyzing one of the months from the 12-18 month search. This involved using LIGO data to search for direct detections of CBCs and collaborating with the other analysts to publish results. I also developed code to compute false alarm rates (FARs) for triggers output by the CBC group's pipeline. This ``post-processing pipeline" was used by the 12-18 month analysts, and served as a basis for further development in the LVC search.

%S5 was a major milestone for LIGO. Although no gravitational waves were detected, the detectors reached their design sensitivity for the first time and the CBC group placed upper limits on the rates of coalescences of BNS, BBH, and NSBH systems. In the process, we identified several areas for improvement. A major concern was the length of time between data taking and when the CBC analysis was complete: it was not until two years after the end of S5 that final results were obtained. Decreasing this latency would allow us to fine-tune the detectors for CBC searches, thereby increasing our sensitivity and the probability of detecting a gravitational wave.

