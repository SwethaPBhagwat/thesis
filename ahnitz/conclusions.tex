

Finally, in Ch.~\ref{ch:bns_dev} we have presented a new pipeline specifically targeted for the detection of gravitational-waves from binary neutron star sources in LIGO data. Using the single stage search pipeline we investigated the configuration choices used for PSD estimation, SNR thresholds, low frequency cutoff, and $\chi^2$ bins used within the ranking statistic. To assess the sensitivity, we develop a method to measure the false alarm rate of possible signals, and introduce the concept of both the inclusive and exclusive FAR measures. We find that for S6 data, the choices for low frequency cutoff at 40Hz, and the SNR threshold at 5.5, as used in prior S6/VSR2,3 searches for BNS sources, were appropriate. Additionally, we find that for a two detector search conducted using the Hanford and Livingston observatories, decreasing the SNR threshold below 5.3 will not result in any gain in search sensitivity using the conservative inclusive IFAR. In sec~\ref{sec:psd} and sec~\ref{sec:chisq} we show significant improvements in search sensitivity for BNS sources by retuning the number of PSD samples per estimate, and the number of bins used in the signal consistance test, respectively. We find an additional $25\%$ increase in the detection rate of BNS systems when using the retuned BNS search over a BNS search that uses the configuration proposed in Ch.~\ref{ch:single_stage}, which showed a $10\%$ increase over the S6/VSR2,3 configuration. We also find that using an aligned spin template bank marginally decreases the sensitivity 
to BNS mergers for conservative estimates of the BNS populations when comparing to a bank of stictly non-spinning templates. As these tuning significantly differ from those used in the wider lowmass search performed in S6/VSR2,3, we propose that a focused, non-spinning search for binary neutron stars be conducted for aLIGO and AdV.
