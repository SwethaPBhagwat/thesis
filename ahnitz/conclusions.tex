In this work we have investigated the effects of neglecting spin when
searching for binary neutron star systems in aLIGO and AdV. We have found
that, if component spins in binary neutron star systems are as large as 0.4,
then neutron star spin cannot be neglected, and there is a non-trivial loss in
signal-to-noise ratio even if the maximum spin is restricted to be less than
0.05. We have shown that the
geometric algorithm for placing and aligned spin template bank works for
aligned spin systems and have demonstrated that it does significantly better
for generic, precessing BNS systems than the traditional non-spinning bank.
However, for the BNS aligned spin $\chi_i < 0.4$ parameter space the aligned
spin bank requires approximately five times as many templates as the
non-spinning bank. This increased number of templates will increase the
computational cost of the search and increase the number of background events,
so needs to be balanced against the potential gain in being able to cover a
larger region of parameter space. A further advantage of this method is the ease
with which it can be incorporated into existing or future search
pipelines, which include the use of signal-based vetoes~\cite{Allen:2004gu}
and coincidence algorithms~\cite{Robinson:2008}. In future work we will
investigate how this template bank performs in data from the aLIGO and AdV
detectors which includes non-Gaussian and non-stationary noise features.
Finally we note that the method proposed in this work should be applicable
wherever the TaylorF2 waveforms closely represent actual gravitational
waveforms. In a future work we will investigate how well this method performs
in the binary black hole and neutron-star, black-hole regions of the parameter space.
Wherever the TaylorF2 approximation begins to break down, a stochastic
bank placement may still be the most viable option.


We have found that there is significant disagreement between \ac{NSBH}
waveforms modelled with the TaylorT2, TaylorT4, and SEOBNRv1 approximants. 
This will pose problems for the construction of optimal NSBH detection searches, 
potentially reducing the event rate, 
and may cause significant biases in the parameter measurement of detected signals.
The discrepancies are not accounted for by the differences between
frequency and time domain waveforms and start at fairly low ($v \sim 0.2$) orbital velocities.
Since the discrepancies in the approximants result from how the \ac{PN} expansions of the energy and flux
are combined and truncated, we conclude
that the calculation of higher order \ac{PN} terms is required to increase the
faithfulness of these approximants, and more importantly, to improve the
ability to detect \ac{NSBH} coalescences. The
discrepancies between approximants are significantly smaller when the spin of
the black hole is close to zero, which further motivates the calculation of the
\ac{PN} terms associated with the spin of the objects beyond those known
completely up to 2.5\ac{PN} order and partially up to 3.5\ac{PN}.
Therefore, additional work is needed to verify
the validity of waveform models used for \ac{NSBH} searches.
We also note that we have
only compared different waveform families under the assumption that the spins
of the component objects are (anti-)aligned with the orbital angular momentum
of the system.  It is expected that generic \ac{NSBH} systems will not be limited to


We have demonstrated the use of a new pipeline to search for gravitational
waves from compact object binaries in LIGO data.  The results of our study are
summarized in Fig.~\ref{fig:conc} which compares the sensitivity of the search
pipeline used in S6/VSR2,3 (analysis 1 of Table~\ref{table:search}) with the
most sensitive pipeline proposed here (analysis 8 of Table~\ref{table:search})
which uses a shared fixed 3.5pN template bank in both detectors generated using 
a harmonic mean power spectral density, and the exact-match coincidence test.  
We see that these improvements result in a gain of $\sim 10\%$ in the sensitive
volume of the search at a false-alarm rate of $10^{-3}$ per year. 
The new pipeline uses a simpler, single-stage workflow that allows us to
estimate false-alarm rates to $\sim 10^{-4}$ per year using one week of data. With our improved
implementation of the $\chi^2$ signal-based veto, we demonstrate that the new
pipeline has the same computational cost as the two-stage workflow used in the S6/VSR2,3
analysis. We propose that this workflow be used as a basis for offline
searches for gravitational waves from compact-object binary sources in aLIGO
and AdV.

We note that a new class of search pipeline was prototyped in
S6/VSR2,3~\cite{Abbott:2011ys} that produces triggers in low-latency for rapid
follow-up by electromagnetic observatories. These pipelines are under
active development for aLIGO and AdV~\cite{Adams:MBTA,Cannon:2011vi}.
Low-latency searches differ from the pipeline presented here as they are constrained to
only use information available in the past and trade computational cost for
speed of producing detection candidates. However, since they are 
based on coincident matched filtering, our results can also be
used to inform the development of low-latency searches. For example, we would expect
that the harmonic mean (using recent past detector data) would provide the best
power spectral density estimation for the construction of template banks used
in the singular value decomposition proposed in Ref.~\cite{Cannon:2011vi}.
Similarly, we expect that exact-match coincidence would provide the best
coincidence method for the low-latency pipelines.

Finally, we note that Figs.~\ref{fig:ethinca_hist} and \ref{fig:exact_hist}
show that, although the distribution of triggers in the S6 search using the
ellipsoidal test is very close to that of Gaussian noise this is not the case
for exact-match. This suggests that additional tuning is possible to increase
the sensitivity of the search. Investigation of improved tuning could explore
the optimal length of time for a single bank, further tuning of the
coincidence test, improvements to power spectral density estimation used in
the matched filter, improved signal-based vetoes and optimization of the
combined detection statistic. Further tuning beyond what is presented here
will be the subject of future studies. 
aligned spins, but may instead be more isotropically oriented.
This could lead to an additional source of discrepancy between our models and
the true signal, which would result in an additional loss in the detection rate
of sources.


In this work we have explored the effect that the angular momentum of the black
hole will have on searches for neutron-star black-hole binaries with
\ac{aLIGO}. The black hole's angular momentum will affect the phase evolution
of the emitted gravitational-wave signal, and, if the angular momentum is
misaligned with the orbital plane, will cause the system to precess. We have
found that if these effects are neglected in the filter waveforms used to
search for \ac{NSBH} binaries it will result in a loss in detection rate of
$31-36\%$ when searching for \ac{NSBH} systems with masses uniformly 
distributed in the range 
$(3-15,1-3)M_{\odot}$. When restricting the masses to 
$(9.5-10.5,1.35-1.45)M_{\odot}$ we find that the loss in detection rate is
$33 - 37\%$. The error in these measurements is due to uncertainty in 
the \ac{PN} waveform models used to simulate \ac{NSBH} gravitational-wave 
signals.

We have shown that an aligned spin template bank offers a
$16\%-30\%$ improvement in the detection rate of neutron-star black-hole
mergers when compared to a non-spinning template bank when searching for 
\ac{NSBH} systems with masses in the range $(3-15,1-3)M_{\odot}$. However, when
searching for \ac{NSBH} systems with masses restricted to the range 
$(9.5-10.5,1.35-1.45)M_{\odot}$ we find the improvement is reduced to $5-17\%$.
Some systems are not recovered well with this new bank of filters. These systems
are ones where the black-hole spin is misaligned with the orbit and the waveform
is significantly modified due to precession of the orbital plane. This happens
most often when $m_{BH} / m_{NS}$ and the spin magnitude are both large. In
\cite{Brown:2012gs} the authors predict where in the parameter space to expect
\ac{NSBH} systems that will not be recovered well by non-precessing template
banks. These predictions were
given in terms of the angles between the orbital plane, the black hole's angular
momentum and the line-of-sight to an observer. These predictions agree with the
results that we obtain in this work. In \cite{Ajith:2012mn} the authors claim
that an aligned-spin template bank will be effectual for detecting precessing
\ac{NSBH} systems. In this work, we find that with an aligned-spin template
bank $17-23\%$ of \ac{NSBH} systems will be missed compared to an ideal search
with
exactly matching filter waveforms. In reality this ideal search could never
be performed as it would require an infinite number of filter waveforms.
Template banks are usually constructed to allow for no more than a 3\% loss in
\ac{SNR}, therefore we expect to lose up to $10\%$ of systems even if the
template bank fully covers the signal parameter space. We therefore conclude
that searches using precessing waveforms as templates could potentially
increase the detection rate of \ac{NSBH} signals, but not by more than $\sim
20\%$. Performing such a search would, however, remove an observational bias
against systems where precessional effects are most prevalent in the
gravitational-wave signal.

These figures are also affected by the 
parameter distribution chosen for the \ac{NSBH} systems. Here we chose a 
distribution that is uniform in mass, uniform in spin magnitudes, isotropic in 
spin orientations and isotropic in orientation parameters and sky location. We 
have however, explored how the ability to detect precessig \ac{NSBH} signals 
varies as a function of the masses and spins as seen in Figures 
\ref{fig:aspinavFF} and \ref{fig:arseofsauron}. 

When searching for \ac{NSBH} systems in \ac{aLIGO} one has to
consider the non-Gaussianity of the background noise, which we have not done in
this work. A non-Gaussian noise
artifact can produce \acp{SNR} that are considerably larger than those expected
from Gaussian noise fluctutations. To deal with this, numerous
consistency tests are used in the analyses to separate gravitational wave
signals from instrumental noise artifacts \cite{Babak:2012zx}. It is possible
that the detection rate could be further reduced from the values we quote in
this work if some signals \emph{fail} these consistency tests and are
mis-classified as non-Gaussian noise transients.
However, these signal consistency tests should only act to remove, or reduce the
significance of, events that already have low fitting factors and therefore do
not match well with the search templates. 
Another important consideration is that of the
number of templates used in the bank. To achieve higher fitting factors will
require more template waveforms, covering a larger signal space, which will 
allow more freedom in matching the
background noise and will mean that the \ac{SNR} of the loudest background
triggers will increase. Therefore signals will need slightly higher \acp{SNR} to
achieve the same false alarm probability. However, a factor of 10 increase in
the number of \emph{independent} templates will only increase the expected
\ac{SNR} of the loudest background event by less than $5\%$, if 
Gaussian noise is assumed. Therefore, while we
are careful to note these considerations, we do not believe they will have a
large impact on the numbers we quote above and leave a detailed investigation of
such effects to future work.

In this work we have restricted ourselves to 
considering post-Newtonian, inspiral-only signal waveforms and consider only the 
case of two point particles. This was done as there is not currently any widely 
available waveform model that includes both the full evolution of a \ac{NSBH} 
coalescence \emph{and} includes precessional effects over the full parameter 
space that we consider. When such a model is available it may be that 
tidal forces and the merger component of the waveform may affect our 
conclusions. We believe that such effects will be limited as the
black hole mass is $< 15 M_{\odot}$ in our simulations, however it would be 
informative to repeat our simulations when a full \ac{NSBH} waveform model is 
available.


We have presented a new pipeline specifically targeted for the detection
of gravitational-waves from binary neutron star sources in LIGO data. 
Using the single stage search pipeline we investigated the configuration
choices used for PSD estimation, SNR thresholds, low frequency cutoff, 
and $\chi^2$ bins used within the ranking statistic. 
To assess the sensitivity, we develop a method to measure the false 
alarm rate of possible signals, and introduce the concept of both the
inclusive and exclusive FAR measures. We find that for S6 data, the 
choices for low frequency cutoff at 40Hz, and the SNR threshold at 5.5,
as used in prior S6/VSR2,3 searches for BNS sources, were appropriate.
Additionally, we find that for a two detector search conducted using
the Hanford and Livingston observatories, decreasing the SNR threshold
below 5.3 will not result in any gain in search sensitivity using the conservative 
inclusive IFAR. In sec~\ref{sec:psd} and sec~\ref{sec:chisq} we show significant
improvements in search sensitivity for BNS sources by retuning the number of PSD
samples per estimate, and the number of bins used in the signal consistance test,
respectively. We find an additional $25\%$ increase in the detection rate
of BNS systems when using the retuned BNS search over a BNS search that
uses the configuration proposed in Ch.~\ref{ch:single_stage},
which showed a $10\%$ increase over the S6/VSR2,3 configuration.
 We also find that using an aligned spin template bank marginally decreases the sensitivity 
to BNS mergers for conservative estimates of the BNS populations when
comparing to a bank of stictly non-spinning templates. As these tuning significantly
differ from those used in the wider lowmass search performed in S6/VSR2,3, 
we propose that a focused, non-spinning search for binary neutron
 stars be conducted for aLIGO and AdV.
