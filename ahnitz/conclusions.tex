We have investigated the effects of neglecting spin when
searching for binary neutron star systems in aLIGO and AdV. We have found
that if component spins in binary neutron star systems are as large as 0.4
then neutron star spin cannot be neglected, and there is a non-trivial loss in
signal-to-noise ratio even if the maximum spin is restricted to be less than
0.05. We have shown that the geometric algorithm for placing and
aligned spin template bank works for
aligned spin systems and have demonstrated that it does significantly better
for generic, precessing BNS systems than the traditional non-spinning bank.
However, for the BNS aligned spin $\chi_i < 0.4$ parameter space the aligned
spin bank requires approximately five times as many templates as the
non-spinning bank. This increased number of templates will increase the
computational cost of the search and increase the number of background events,
so needs to be balanced against the potential gain in being able to cover a
larger region of parameter space. A further advantage of this method is the ease
with which it can be incorporated into existing or future search
pipelines, which include the use of signal-based vetoes~\cite{Allen:2004gu}
and coincidence algorithms~\cite{Robinson:2008}.

We have found that there is significant disagreement between \ac{NSBH}
waveforms modelled with the TaylorT2, TaylorT4, and SEOBNRv1 approximants. 
This will pose problems for the construction of optimal NSBH detection searches, 
potentially reducing the event rate, 
and may cause significant biases in the parameter measurement of detected signals.
The discrepancies are not accounted for by the differences between
frequency and time domain waveforms and start at fairly low ($v \sim 0.2$) orbital velocities.
Since the discrepancies in the approximants result from how the \ac{PN} expansions of the energy and flux
are combined and truncated, we conclude
that the calculation of higher order \ac{PN} terms is required to increase the
faithfulness of these approximants, and more importantly, to improve the
ability to detect \ac{NSBH} coalescences. The
discrepancies between approximants are significantly smaller when the spin of
the black hole is close to zero, which further motivates the calculation of the
\ac{PN} terms associated with the spin of the objects beyond those known
completely up to 2.5\ac{PN} order and partially up to 3.5\ac{PN}.

We have explored the effect that the angular momentum of the black
hole will have on searches for neutron-star black-hole binaries with
\ac{aLIGO}. The black hole's angular momentum will affect the phase evolution
of the emitted gravitational-wave signal, and, if the angular momentum is
misaligned with the orbital plane, will cause the system to precess. We have
found that if these effects are neglected in the filter waveforms used to
search for \ac{NSBH} binaries it will result in a loss in detection rate of
$31-36\%$ when searching for \ac{NSBH} systems with masses uniformly 
distributed in the range 
$(3-15,1-3)M_{\odot}$. When restricting the masses to 
$(9.5-10.5,1.35-1.45)M_{\odot}$ we find that the loss in detection rate is
$33 - 37\%$. The error in these measurements is due to uncertainty in 
the \ac{PN} waveform models used to simulate \ac{NSBH} gravitational-wave 
signals. We have found that an aligned spin template bank offers a
$16\%-30\%$ improvement in the detection rate of neutron-star black-hole
mergers when compared to a non-spinning template bank when searching for 
\ac{NSBH} systems with masses in the range $(3-15,1-3)M_{\odot}$. However, when
searching for \ac{NSBH} systems with masses restricted to the range 
$(9.5-10.5,1.35-1.45)M_{\odot}$ we find the improvement is reduced to $5-17\%$.
Some systems are not recovered well with this new bank of filters. These systems
are ones where the black-hole spin is misaligned with the orbit and the waveform
is significantly modified due to precession of the orbital plane. This happens
most often when $m_{BH} / m_{NS}$ and the spin magnitude are both large. Note, 
that these results are for an idealized search that neglects the effects
of non-Gaussian noise.

We have demonstrated the use of a new pipeline to search for gravitational
waves from compact object binaries in LIGO data. We find that the sensitivity of the search
pipeline used in S6/VSR2,3 is $\sim 10\%$ less sensitive at a false-alarm rate of $10^{-3}$ per year than the most sensitive
pipeline proposed, which uses a shared fixed 3.5pN template bank in both detectors generated using 
a harmonic mean power spectral density, and the exact-match coincidence test.  
The new pipeline uses a simpler, single-stage workflow that allows us to
estimate false-alarm rates to $\sim 10^{-4}$ per year using one week of data. With our improved
implementation of the $\chi^2$ signal-based veto, we demonstrate that the new
pipeline has the same computational cost as the two-stage workflow used in the S6/VSR2,3
analysis. We propose that this workflow be used as a basis for offline
searches for gravitational waves from compact-object binary sources in aLIGO
and AdV. Finally, we note that although the distribution of triggers 
in the S6 search using the ellipsoidal test is very close to that of Gaussian noise this is not the case
for exact-match. This suggests that additional tuning is possible to increase
the sensitivity of the search. 

We have presented a new pipeline specifically tuned for the detection
of gravitational-waves from binary neutron star sources in LIGO data. 
Using the single-stage search pipeline we investigated the configuration
choices used for PSD estimation, SNR thresholds, low frequency cutoff, 
and $\chi^2$ bins used within the ranking statistic. 
To assess the sensitivity, we develop a method to measure the false 
alarm rate of possible signals, and introduce the concept of both the
inclusive and exclusive FAR measures. We find that for S6 data, the 
choices for low frequency cutoff at 40Hz, and the SNR threshold at 5.5,
as used in prior S6/VSR2,3 searches for BNS sources, were appropriate.
Additionally, we find that for a two detector search conducted using
the Hanford and Livingston observatories, decreasing the SNR threshold
below 5.3 will not result in any gain in search sensitivity using the conservative 
inclusive IFAR. We show significant
improvements in search sensitivity for BNS sources by retuning the number of PSD
samples per estimate, and the number of bins used in the signal consistence test,
respectively. We find an additional $25\%$ increase in the detection rate
of BNS systems when using the retuned BNS search over a BNS search that
uses the initially proposed single-stage pipeline,
which had already demonstrated a $10\%$ improvement over the S6/VSR2,3 configuration.
We also find that using an aligned spin template bank marginally decreases the sensitivity 
to BNS mergers for conservative estimates of the BNS populations when
comparing to a bank of stictly non-spinning templates. As these tuning significantly
differ from those used in the wider lowmass search performed in S6/VSR2,3, 
we propose that a focused, non-spinning search for binary neutron
stars be conducted for aLIGO and AdV.
