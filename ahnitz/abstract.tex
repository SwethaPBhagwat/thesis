%%%%%%%%%%%%%%%%%%%%%%%%%%%%%%%%%%%%%%%%%%%%%%
%%%%%%% ABSTRACT
%%%%%%%%%%%%%%%%%%%%%%%%%%%%%%%%%%%%%%%%%%%%%%

% bns spin abstract section
The detection of gravitational waves from binary neutron stars
is a major goal of the gravitational-wave observatories Advanced LIGO and
Advanced Virgo. Previous searches for binary neutron stars with LIGO and Virgo
neglected the component stars' angular momentum (spin).  We demonstrate that
neglecting spin in matched-filter searches causes advanced detectors
to lose more than 3\% of the possible signal-to-noise ratio for 59\% (6\%) of
sources, assuming that neutron star dimensionless spins, $c\mathbf{J}/GM^2$, are uniformly distributed
with magnitudes between $0$ and $0.4$ $(0.05)$ and that the neutron stars
have isotropically distributed spin orientations.
We present a new method for constructing template banks for gravitational
wave searches for systems with spin. We present a new metric in a parameter
space in which the template placement metric is globally flat.
This new method can create template banks of signals with
non-zero spins that are (anti-)aligned with the orbital angular momentum.  We show that this search loses more than
3\% of the maximium signal-to-noise for only 9\% (0.2\%) of BNS sources with dimensionless spins between $0$ and $0.4$ $(0.05)$ and isotropic spin orientations. Use of this
template bank will prevent selection bias in gravitational-wave searches and
allow a more accurate exploration of the distribution of spins in binary
neutron stars.

%nsbh abstract section
Gravitational waves radiated by the coalescence of compact-object binaries
containing a neutron star and a black hole are one of the most interesting
sources for the ground-based gravitational-wave observatories Advanced LIGO and
Advanced Virgo. Advanced LIGO will be sensitive to the inspiral of a $1.4\, M_\odot$
neutron star into a $10\,M_\odot$ black hole to a maximum distance of $\sim 900$~Mpc. 
Achieving this sensitivity and extracting the physics imprinted in
observed signals requires accurate modeling of the binary to construct
template waveforms. In a neutron star--black hole binary, the black hole may
have significant angular momentum (spin), which affects the phase evolution of
the emitted gravitational waves. We investigate the ability of currently available
post-Newtonian templates to model the gravitational waves emitted during the
inspiral phase of neutron star--black hole binaries. We restrict to the case where the spin of the
black hole is aligned with the orbital angular momentum and compare several
post-Newtonian approximants. We examine
restricted amplitude post-Newtonian waveforms that are accurate to
third-and-a-half post-Newtonian order in the orbital dynamics and complete to second-and-a-half post-Newtonian order
in the spin dynamics. We also consider post-Newtonian waveforms that include the recently derived third-and-a-half
post-Newtonian order spin-orbit correction and the third post-Newtonian order spin-orbit tail correction. 
We compare these post-Newtonian approximants to the effective-one-body waveforms for spin-aligned binaries.
For all of these waveform families, we find that
 there is a large disagreement between
different waveform approximants starting at low to moderate black hole spins,
particularly for binaries where the spin is anti-aligned with the orbital
angular momentum. The match between the TaylorT4 and TaylorF2 approximants is $\sim 0.8$ for a binary with $m_{BH}/m_{NS} \sim 4$ and 
$\chi_{BH} = cJ_{BH}/Gm^2_{BH} \sim 0.4$.
We show that the divergence between the gravitational waveforms begins in the early
inspiral at $v \sim 0.2$ for $\chi_{BH} \sim 0.4$.  Post-Newtonian spin corrections beyond those currently
known will be required for optimal detection searches and to measure the
parameters of neutron star--black hole binaries. The strong dependence of 
the gravitational-wave signal on the spin dynamics will make it possible to extract significant
astrophysical information from detected systems with Advanced LIGO and
Advanced Virgo.

%nsbh effectualness / precession section
The first direct detection of neutron-star--black-hole binaries will
likely be made with gravitational-wave observatories. Advanced LIGO and
Advanced Virgo will be able to observe neutron-star--black-hole mergers at a
maximum distance of 900Mpc. To acheive this sensitivity, gravitational-wave
searches will rely on using a bank of filter waveforms that
accurately model the expected gravitational-wave signal. The emitted signal
will depend on the masses of the black hole and the neutron star and also the
angular momentum of both components. The angular momentum
of the black hole is expected to be comparable to the orbital angular momentum
when the system is emitting gravitational waves in Advanced LIGO's and Advanced
Virgo's sensitive band.
This angular momentum will affect the dynamics of the inspiralling system and
alter the phase evolution of the emitted gravitational-wave signal. In
addition, if the black hole's angular momentum is not aligned with the orbital
angular momentum it will cause the orbital plane of the system to precess.
In this work we demonstrate that if the effect of the black
hole's angular momentum is neglected in the waveform models used in
gravitational-wave searches, the detection rate of $(10+1.4)M_{\odot}$
neutron-star--black-hole
systems would be reduced by $33 - 37\%$. The error in this measurement is due
to uncertainty in the Post-Newtonian approximations that are used to model the
gravitational-wave signal of neutron-star--black-hole inspiralling binaries. We
describe a new method for creating a bank of filter waveforms where the black
hole has non-zero angular momentum that is aligned with the orbital angular
momentum. With this bank we find that the detection rate of $(10+1.4)M_{\odot}$
neutron-star--black-hole systems would be reduced by $26-33\%$. Systems that
will not be detected are ones where the precession of the orbital plane causes
the gravitational-wave signal to match poorly with non-precessing filter
waveforms. We identify the regions of parameter space where such systems occur
and suggest methods for searching for highly precessing
neutron-star--black-hole binaries.

The second generation of ground-based gravitational-wave
detectors will begin taking data in September 2015. Sensitive and
computationally-efficient data analysis methods will be required to maximize
what we learn from their observations. In this paper, we describe improvements
made to the offline analysis pipeline searching for gravitational waves from
stellar-mass compact binary coalescences, and assess how these improvements
affect search sensitivity. Starting with the two-stage \texttt{ihope} pipeline
used in S5, S6 and VSR1-3 and using two weeks of S6/VSR3 data as test periods,
we first demonstrate a pipeline with a simpler workflow. This
\emph{single-stage pipeline} performs matched filtering and coincidence
testing only once. This simplification allows us to reach much lower
false-alarm rates for loud candidate events. We then describe an optimized
$\chi^2$ test which minimizes computational cost. Next, we compare methods of
generating template banks, demonstrating that a fixed bank may be used for
extended stretches of time. Fixing the bank reduces the cost and complexity,
compared to the previous method of regenerating a template bank every 2048 s
of analyzed data. Creating a fixed bank shared by all detectors also allows us
to apply a more stringent coincidence test, whose performance we quantify.
With these improvements, we find a 10\% increase in sensitive volume
with a negligible change in computational cost.
