%%%%%%%%%%%%%%%%%%%%%%%%%%%%%%%%%%%%%%%%%%%%%%
%%%%%%% ABSTRACT
%%%%%%%%%%%%%%%%%%%%%%%%%%%%%%%%%%%%%%%%%%%%%%

% bns spin abstract section
The detection of gravitational waves from binary neutron star
is a major goal of the gravitational-wave observatories Advanced LIGO and
Advanced Virgo. In addition, the first direct 
detection of neutron-star--black-hole binaries will
likely be made with gravitational-wave observatories.
Previous searches for binary neutron star and neutron-star--black-hole mergers
with LIGO and Virgo
neglected the component stars' angular momentum (spin). Advanced LIGO and
Advanced Virgo will be able to observe neutron-star--black-hole mergers at a
maximum distance of 900Mpc. To acheive this sensitivity, gravitational-wave
searches will rely on using a bank of filter waveforms that
accurately model the expected gravitational-wave signal. The emitted signal
will depend on the masses and angular momentum of both components. 

We demonstrate in chapter~\ref{ch:bns_spin} that neglecting spin in matched-filter searches causes advanced detectors
to lose more than 3\% of the possible signal-to-noise ratio for 59\% (6\%) of
sources, assuming that neutron star dimensionless spins, $c\mathbf{J}/GM^2$, are uniformly distributed
with magnitudes between $0$ and $0.4$ $(0.05)$ and that the neutron stars
have isotropically distributed spin orientations.
We present a new method for constructing template banks for gravitational
wave searches for systems with spin. We present a new metric in a parameter
space in which the template placement metric is globally flat.
This new method can create template banks of signals with
non-zero spins that are (anti-)aligned with the orbital angular momentum.  We show that this search loses more than
3\% of the maximium signal-to-noise for only 9\% (0.2\%) of BNS sources with dimensionless spins between $0$ and $0.4$ $(0.05)$ and isotropic spin orientations. Use of this
template bank will prevent selection bias in gravitational-wave searches and
allow a more accurate exploration of the distribution of spins in binary
neutron stars.

We investigate in chapter~\ref{ch:nsbh_faith} the ability of currently available
post-Newtonian templates to model the gravitational waves emitted during the
inspiral phase of neutron star--black hole binaries. We restrict to the case where the spin of the
black hole is aligned with the orbital angular momentum and compare several
post-Newtonian approximants. We examine
restricted amplitude post-Newtonian waveforms that are accurate to
third-and-a-half post-Newtonian order in the orbital dynamics and complete to second-and-a-half post-Newtonian order
in the spin dynamics. We also consider post-Newtonian waveforms that include the recently derived third-and-a-half
post-Newtonian order spin-orbit correction and the third post-Newtonian order spin-orbit tail correction. 
We compare these post-Newtonian approximants to the effective-one-body waveforms for spin-aligned binaries.
For all of these waveform families, we find that
 there is a large disagreement between
different waveform approximants starting at low to moderate black hole spins,
particularly for binaries where the spin is anti-aligned with the orbital
angular momentum. The match between the TaylorT4 and TaylorF2 approximants is $\sim 0.8$ for a binary with $m_{BH}/m_{NS} \sim 4$ and 
$\chi_{BH} = cJ_{BH}/Gm^2_{BH} \sim 0.4$.
We show that the divergence between the gravitational waveforms begins in the early
inspiral at $v \sim 0.2$ for $\chi_{BH} \sim 0.4$.  Post-Newtonian spin corrections beyond those currently
known will be required for optimal detection searches and to measure the
parameters of neutron star--black hole binaries. The strong dependence of 
the gravitational-wave signal on the spin dynamics will make it possible to extract significant
astrophysical information from detected systems with Advanced LIGO and
Advanced Virgo.

In chapter~\ref{ch:nsbh_prec} we demonstrate that if the effect of the black
hole's angular momentum is neglected in the waveform models used in
gravitational-wave searches, the detection rate of $(10+1.4)M_{\odot}$
neutron-star--black-hole
systems would be reduced by $33 - 37\%$. The error in this measurement is due
to uncertainty in the Post-Newtonian approximations that are used to model the
gravitational-wave signal of neutron-star--black-hole inspiralling binaries. We
describe a new method for creating a bank of filter waveforms where the black
hole has non-zero angular momentum that is aligned with the orbital angular
momentum. With this bank we find that the detection rate of $(10+1.4)M_{\odot}$
neutron-star--black-hole systems would be reduced by $26-33\%$. Systems that
will not be detected are ones where the precession of the orbital plane causes
the gravitational-wave signal to match poorly with non-precessing filter
waveforms. We identify the regions of parameter space where such systems occur
and suggest methods for searching for highly precessing
neutron-star--black-hole binaries.

In chapter~\ref{ch:single_stage} we describe improvements
made to the offline analysis pipeline searching for gravitational waves from
stellar-mass compact binary coalescences, and assess how these improvements
affect search sensitivity. Starting with the two-stage \texttt{ihope} pipeline
used in S5, S6 and VSR1-3 and using two weeks of S6/VSR3 data as test periods,
we first demonstrate a pipeline with a simpler workflow. This
\emph{single-stage pipeline} performs matched filtering and coincidence
testing only once. This simplification allows us to reach much lower
false-alarm rates for loud candidate events. We then describe an optimized
$\chi^2$ test which minimizes computational cost. Next, we compare methods of
generating template banks, demonstrating that a fixed bank may be used for
extended stretches of time. Fixing the bank reduces the cost and complexity,
compared to the previous method of regenerating a template bank every 2048 s
of analyzed data. Creating a fixed bank shared by all detectors also allows us
to apply a more stringent coincidence test, whose performance we quantify.
With these improvements, we find a 10\% increase in sensitive volume
with a negligible change in computational cost. We follow up with additional
compuatational improvements to the matched-filtering aglorithm in chapter~\ref{ch:opt}.

Finally, in chapter~\ref{ch:single_stage} we demonstrate an analysis pipeline
that is focused on the detection of binary neutron star mergers. Using the 
improved single stage pipeline, we use three weeks of S6/VSR3 data to test 
further improvements to the pipeline. We describe a method for calculating
the significance of candidate events, and measure propabilities under the
assumption of both including and excluding foreground events from a background.
We investigate alternate configurations of the filtering process, including changes
to the spectrum estimation and signal-consistency test. We find that
the new configuration is able to achieve a 25\% increase in sensivitive volume. Lastly,
we investigate using an aligned spin template bank, and show that for conservative 
estimates of BNS populations a non-spinning template bank has marginally superior 
sensitivity.



