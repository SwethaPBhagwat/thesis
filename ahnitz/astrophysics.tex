\section{Introduction}

The coalescence of compact binary systems are a promising source of gravitaional-wave detections from the next generation observatories, Advanced LIGO (aLIGO) and Advanced Virgo (AdV). Binary neutron star systems are likely to be one of the first sources observed by these observatories. Advanced LIGO will begin its first observing run (O1) in the fall of 2015, and will reach design sensitivity by 2018-19. Detection rate estimates for aLIGO and AdV sugest that BNS sources will be one of the most numerous source detected, with plausible
rates of $\sim 10/\mathrm{yr}$~\cite{Abadie:2010cf}. The detection of multiple BNS systems will allow us to explore the processes of stellar evolution and measure the properties of a nuetron star, including information about the nuclear equation of state\cite{lacky}.

%Current electromagnetic observations suggest that the neutron star mass distribution peaks at $1.35 \Msun$--$1.5 \Msun$ with a narrow width~\cite{Kiziltan:2010ct}, although neutron stars in globular clusters seem to have a considerably wider mass distribution~\cite{Kiziltan:2010ct}. There is also evidence that a neutron star in one system has a mass as high as $\sim 3 \Msun$~\cite{Freire:2007jd}. The dimensionless spin magnitude $\chi = cJ/Gm^2$ for neutron stars is constrained by possible neutron star equations of state to a maximum of 0.7~\cite{Lo:2010bj}.  The fastest observed pulsar has a spin period of 1.4 ms~\cite{Hessels:2006ze}, corresponding to a $\chi \sim 0.4$, and the most rapidly spinning observed neutron star in a binary, J0737--3039A, has a spin of only $\chi \sim 0.05$.  

The gravitional wave from an inspiralling binary system can be separated into an inspiral portion, where the wave is slowly increasing in amplitude and frequency, a merger, and the post-merger signal. For systems with a total mass is less than $\sim 12 M_\odot$, and where the angular momenta of the compact objects is low, as is the case with BNS systems, it has been shown that post-Newtonian approximants, which model only the inspiral portion of the waveform, and are currently available at up to 3.5PN order, can provide an accurate model of the gravitational waveform for the purposes of detection.

Since we have a well-modelled gravitational wave signal, searches for gravitational waves from CBC sources use template-based matched filtering~\cite{Allen:2004gu}. The data from a detector is correlated against of bank of known template waveforms. The set of templates is chosen so that any signal will lose no more than $3\%$ of the optimal signal-to-noise ratio that would be obtained by an exactly matching template in the absence of noise. Geometric metric-based methods for placing both non-spinning and aligned spin templates have been shown to be effective for binary neutron stars~\cite{brown:2012qf}.

In addition to possible signals, and Gaussian noise, detector data contains non-Gaussian noise transients, which can generate large, spurious SNR triggers. To mitigate the effect of these noise transients, signal-consistency tests are used to create a weighted detection statistic. In addition, a signal must been seen in multiple detectors with consistent parameters. It has been shown that using a bank of templates shared between all detectors, and requiring a signal to be observed in the same template accross the detector network, improves the overall search sensitivity~\cite{samantha}. 

\section{Populations of BNS Sources}

\section{Population of NSNBH Sources}

\section{Modelling BNS/NSBH Waveforms}
