\def\Msun{M_\odot}
\acrodef{aLIGO}[aLIGO]{Advanced Laser Interferometer Gravitational-wave Observatory}
\acrodef{AdV}[AdV]{Advanced Virgo}
\acrodef{LIGO}[LIGO]{Laser Interferometer Gravitational-wave Observatory}
\acrodef{CBC}[CBC]{compact binary coalescence}
\acrodef{S6}[S6]{LIGO's sixth science run}
\acrodef{VSR23}[VSR2 and VSR3]{Virgo's second and third science runs}
\acrodef{EM}[EM]{electromagnetic}
\acrodef{NS}[NS]{neutron star}
\acrodef{BH}[BH]{black hole}
\acrodef{BNS}[BNS]{binary neutron star}
\acrodef{NSWD}[NSWD]{neutron star-white dwarf}
\acrodef{NSBH}[NSBH]{neutron star and a black hole}
\acrodef{GRB}[GRB]{gamma-ray burst}
\acrodef{S5}[S5]{LIGO's fifth science run}
\acrodef{S4}[S4]{LIGO's fourth science run}
\acrodef{VSR1}[VSR1]{Virgo's first science run}

\acrodef{PSD}[PSD]{power spectral density}
\acrodef{VSR3}[VSR3]{Virgo's third science run}
\acrodef{BBH}[BBH]{binary black holes}
\acrodef{SNR}[SNR]{signal-to-noise ratio}
\acrodef{SPA}[SPA]{stationary-phase approximation}
\acrodef{LHO}[LHO]{LIGO Hanford Observatory}
\acrodef{LLO}[LLO]{LIGO Livingston Observatory}
\acrodef{LSC}[LSC]{LIGO Scientific Collaboration}
\acrodef{PN}[PN]{post-Newtonian}
\acrodef{DQ}[DQ]{data quality}
\acrodef{IFO}[IFO]{interferometer}
\acrodef{DTF}[DTF]{detection template families}
\acrodef{FAR}[FAR]{false alarm rate}
\acrodef{FAP}[FAP]{false alarm probability}
\acrodef{PTF}[PTF]{physical template family}
\acrodef{ADE}[ADE]{advanced detector era}
\acrodef{FFT}[FFT]{Fast Fourier Transformation}
\acrodef{GPU}[GPU]{graphical processing unit}
\acrodef{ISCO}[ISCO]{inner-most stable circular orbit}
\acrodef{MECO}[MECO]{minimum energy circular orbit}

\section{Population of BNS Sources}
\label{astro:bns}

Electromagnetic observations suggest that the \ac{NS} mass
distribution in \ac{BNS} peaks at $1.35 \Msun - 1.5 \Msun$ with a narrow
width~\cite{Kiziltan:2010ct}, although \acp{NS} in globular clusters seem to
have a considerably wider mass distribution~\cite{Kiziltan:2010ct}.  There is
also evidence that a neutron star in one system has a mass as high as $\sim 3
\Msun$~\cite{Freire:2007jd}. 

If the neutron stars are rotating, 
%the gravitational-waves phasing is affected
%by spin-orbit and spin-spin coupling in the
%binary~\cite{Kidder:1992fr,Kidder:1995zr}.
coupling between the neutron stars' spin $\bm{S}_{1,2}$ and the
orbital angular momentum $\bm{L}$ of the binary will affect the dynamics of BNS
mergers~\cite{Kidder:1992fr,Apostolatos:1994mx,Kidder:1995zr,Blanchet:2006gy}.  
We measure the neutron stars' spin using the dimensionless parameter
$\bm{\chi}_{1,2} = {\bm{S}_{1,2}}/{m_{1,2}^2}$. The maximum spin value for a wide class of neutron star equations of state is
$\chi \equiv \left| \bm{\chi} \right| \sim 0.7$~\cite{Lo:2010bj}. 
However, the spins of neutron stars in BNS
systems is likely to be smaller than this limit. The spin period at the birth
of a neutron star is thought to be in the range
$10$--$140$~ms~\cite{Lorimer:2008se,Mandel:2009nx}. During the evolution of
the binary, accretion may increase the spin of one of the
stars~\cite{Bildsten:1997vw}, however neutron stars are unlikely to have
periods less than 1~ms~\cite{Chakrabarty:2008gz}, corresponding to a
dimensionless spin of $\chi \sim 0.4$.  The period of the fastest known pulsar
in a double neutron star system, J0737--3039A, is
$22.70$~ms~\cite{Burgay:2003jj}, corresponding to a spin of only $\chi \sim
0.05$.

\section{Population of NSBH Sources}


Although no \ac{NSBH} binaries have been directly 
observed, both \acp{NS} and \acp{BH} have been observed in other binary systems, and these
observations can be used to make inferences about the mass and spin
distributions that might be expected in \ac{NSBH} binaries. Several \ac{BNS} systems and
\ac{NSWD} systems have been observed by detecting their electromagnetic
signatures. The observational data for \acp{BH} is more limited
than for \acp{NS}.

Black holes observed in X-ray binaries can be used to estimate the \ac{BH}
mass distribution, though it is difficult to disentangle the individual masses
and inclination angle with only electromagnetic observations~\cite{Ozel:2010su}.
Using a population of $\sim20$ low-mass X-ray binary systems with estimated 
masses, two separate works found that a \ac{BH} mass distribution of $7.8 \pm 
1.2 M_{\odot}$ fits the observed data well~\cite{Ozel:2010su,Farr:2010tu}. 
There is evidence that there is a ``mass gap'' between $3M_{\odot}$ and 
$5M_{\odot}$ where \acp{BH} will not form~\cite{Ozel:2010su,Farr:2010tu}, 
although this may be due to observational bias~\cite{Kreidberg:2012ud}. When 
high-mass X-ray binary systems are considered the mass distribution increases 
to $9.2012 \pm 3 M_{\odot}$, although a Gaussian model is a poor fit for these 
systems~\cite{Farr:2010tu}. Evidence exists for a stellar mass black hole with 
mass $> 20 M_{\odot}$ in the IC 10 X-1 x-ray 
binary~\cite{Prestwich:2007mj,Silverman:2008ss}. Observations of black hole spin have found spin values that span the minimum and
maximum possible values for Kerr black holes~\cite{Miller:2009cw}, therefore we conservatively
assume a uniform black-hole spin distribution between 0 and 1. 

\section{Modelling Gravitational Waveforms}
Achieving aLIGO's optimal sensitivity to
\ac{NSBH} binaries and exploring their physics 
requires accurate modeling of the gravitational waves emitted 
over many hundreds of orbits as the signal sweeps through the detector's
sensitive band. For \ac{BNS} systems the mass
ratio between the two neutron stars is small and the angular momenta of the
neutron stars (the neutron stars' spins) is low. In this case, the emitted waves are
well modeled by \ac{PN}
theory~\cite{Blanchet:2006zz,Buonanno:2009zt,Brown:2012qf}. 
However, \ac{NSBH} binaries can have significantly larger mass ratios and the spin of
the black hole can be much larger than that of a neutron star. The combined
effects of mass ratio and spin present challenges in constructing accurate gravitational waveform models for
\ac{NSBH} systems, compared to \ac{BNS} systems.  For the studies in this paper, we choose to use a uniform 
range of 3 to 15 solar masses for the black holes in our \ac{NSBH} signal 
population. This is partly motivated by the considerations above, and partly by 
our concern of the validity of inspiral-only, point particle waveform models 
for high-mass \ac{NSBH} systems.

\subsection{Constructing post-Newtonian Waveforms}
\label{sec:waveforms}

One way we can model the gravitational waveforms of BNS and NSBH
binaries is by constructing waveforms using the
\ac{PN} approximations of the binary's equation of motion and
gravitational radiation.  To obtain the gravitational-wave phase from these
quantities, we assume that the binary evolves adiabatically through a series
of quasi-circular orbits. This is a reasonable approximation as gravitational
radiation is expected to circularize the orbits of isolated
binaries~\cite{Peters:1964zz}.  In this limit, the equations of motion reduce
to series expansions of the center-of-mass energy $E(v)$ and gravitational-wave
flux $\mathcal{F}(v)$, which are expanded as a power series in the orbital
velocity $v$ around $v = 0$. They are given as
%
\begin{align}
%
E(v) &= E_{\mathrm{N}} v^2 \left(1+\sum_{n=2}^{6}E_i v^i\right), \\
%
F(v) &= F_{\mathrm{N}} v^{10} \left(1+\sum_{n=2}^{7}F_i v^i\right),
%
\end{align}
%
where the coefficients $\{E_\mathrm{N}, E_i, F_\mathrm{N}, F_i\}$ are
defined in Appendix~\ref{app:EF}.  For terms not involving the spin of the
objects, the energy is known to order $v^6$, while the flux is known to $v^7$,
referred to as $3.0$PN and $3.5$PN, respectively.  At order $3.0$PN, the flux
contains terms proportional to both $v^6$ and $v^6 \log v$; which are regarded
to be of the same order. Complete terms involving the spins of the objects first
appear as spin-orbit couplings at 1.5\ac{PN} order, with spin-spin couplings
entering at 2\ac{PN} order, and next-to-leading order spin-orbit couplings
known at 2.5\ac{PN} order. 

We use the assumption that these systems are evolving independently to relate
the \ac{PN} energy and gravitational-wave flux equations, i.e. the loss of energy of the
system is given by the gravitational-wave flux
%
\begin{equation}
%
\frac{dE}{dt} = - \mathcal{F}.
%
\end{equation}
%
This can be re-arranged to give an expression for the time evolution of the
orbital velocity,
%
\begin{equation}\label{eq:dvdt}
%
\frac{d v}{dt} = - \frac{\mathcal{F}(v)}{E'(v)},
%
\end{equation}
%
where $E'(v) = dE/dv$. The orbital evolution can be transformed to the gravitational
waveform by matching the near-zone gravitational potentials to the wave
zone. The amplitude of gravitational waves approximated in this way are given by the
\ac{PN} expansion of the amplitude. This gives different amplitudes for
different modes of the orbital frequency. The dominant gravitational-wave frequency $f$ is
given by twice the orbital frequency, which is related to the orbital velocity
by $v = (\pi M f)^{1/3}$. The orbital phase is therefore given by 
%
\begin{equation}\label{eq:dphidt}
%
\frac{d\phi}{dt} = \frac{v^3}{M},
%
\end{equation}
%
and the phase of the dominant gravitational-wave mode is twice the orbital phase.
Here, we will only expand the  gravitational-wave  amplitude to Newtonian
order (0\ac{PN}), which, when combined with the phase, is referred to as a
restricted \ac{PN} waveform.

Solutions $v(t)$ and $\phi(t)$ to Eqs.~(\ref{eq:dvdt}) and (\ref{eq:dphidt})
can be used to construct the plus and cross polarizations and the observed
gravitational waveform.  For restricted waveforms, these are:
%
\begin{eqnarray}
%
h_+(t) &=& - \frac{2\,M\,\eta}{D_L}\,v^2\,(1 + \cos^2 \theta)\,\cos 2 \phi(t)\
, \\
%
h_\times(t) &=& - \frac{2\,M\,\eta}{D_L}\,v^2\,2 \cos \theta\,\sin 2 \phi(t)\ ,
\\
%
h(t) &=& F_+ \, h_+(t) + F_\times \, h_\times(t)\ .
%
\end{eqnarray}
%
Here $F_+$ and $F_\times$ are the antenna pattern functions of the detector,
$D_L$ is the luminosity distance between the binary and observer, and $\theta$
is the inclination angle between the orbital angular momentum of the binary and
the direction of gravitational-wave propagation: $\cos \theta = \hat{L} \cdot \hat{N}$. Thus, a
non-precessing, restricted PN waveform is fully specified by 
$v(t)$ and $\phi(t)$ (or equivalently $t(v)$ and $\phi(v)$).

We now have the ingredients necessary to produce the TaylorT2 and TaylorT4
families of approximants, which we describe in the following sections. 

%%%%%%%%%%%
\subsection{TaylorT4}

The TaylorT4 approximant, introduced in~\cite{Buonanno:2002fy}, is formed by
numerically solving the differential equation
%
\begin{equation}\label{eq:t4}
%
\frac{dv}{dt} = \left[ \frac{-\mathcal{F}(v)}{E'(v)} \right]_{k} = A_{k}(v).
%
\end{equation}
%
The notation $\left[ Q \right]_{k}$ indicates that the quantity $Q$ is to be
truncated at $v^k$ order. Terms containing pieces logarithmic in
$v$ are considered to contribute at the order given by the non-logarithmic
part. Thus waveforms expanded to 3.5\ac{PN} order in the phase would be
truncated at $k = 7$.  We use $A_k$ as shorthand for the truncated quantity that
is used as the expression for $dv/dt$.

The resulting differential equation, given explicitly in Appendix~\ref{app:T4},
is non-linear and therefore must be solved numerically. The result is a
function $v(t)$. The phase can then be calculated by
%
\begin{equation}
%
\frac{d\phi}{dt} = \frac{v(t)^3}{M}.
%
\end{equation}
%
The phase is integrated from a fiducial starting frequency up to the \ac{MECO},
which is defined by
%
\begin{equation}
%
\frac{dE(v)}{dv} = 0.
%
\end{equation}
%
The \ac{MECO} frequency is where we consider the adiabatic approximation to have broken down. Note
that the \ac{MECO} frequency is dependent on not only the masses but also the
spins of the objects; specifically, systems where the objects' spins are aligned
with the orbital angular momentum will have a higher \ac{MECO} frequency.
When the partial spin-related terms at 3.0\ac{PN} and 3.5\ac{PN} are included, however,
there are regions of the \ac{NSBH} parameter space for which the MECO condition is never satisfied. For these cases,
we impose that the rate of increase in frequency must not decrease (i.e. we stop if $dv/dt \leq 0$), and that the 
characteristic velocity of the binary is less than $c$ (i.e. we stop if $v \geq 1$). 
We terminate the waveforms as soon as any of these stopping conditions are met.

%To be used as a template or injection, the differential equations
%must be solved numerically and an FFT performed to transform to the frequency
%domain. Both of these steps are computationally expensive.

%%%%%%%%%%%
\subsection{TaylorT2}
\label{subsec:t2}

In contrast to the TaylorT4 approximant, the TaylorT2 approximant is
constructed by expanding $t$ in terms of $v$ and truncating the expression to
consistent \ac{PN} order. We first construct the quantity
%
\begin{equation}\label{eq:t2}
%
\frac{dt}{dv} = \left[ \frac{E'(v)}{-\mathcal{F}(v)} \right]_{k} = B_{k}(v).
%
\end{equation}
%
This can be combined with the integral of Eq.~\eqref{eq:dphidt} and solved in
closed form as a perturbative expansion in $v$,
%
\begin{equation}\label{eq:phaset2}
%
\phi(v) = \int \frac{v^3}{M} B_{k}(v) dv.
%
\end{equation}
%
The explicit result of this integral is given in Ap.~\ref{app:T2}.
Similar to TaylorT4, the phase is generally calculated up to the \ac{MECO}
frequency. However, for some points of parameter space, this formulation can
result in a frequency that is not monotonic below the \ac{MECO} frequency.
As with TaylorT4, we stop the waveform evolution with $dv/dt \leq 0$ or $v \geq 1$.


A related approximant can be computed directly in the frequency domain by using
the stationary phase approximation~\cite{Droz:1999qx,Blanchet:2006zz}.  This
approximant is called TaylorF2 and can be expressed as an analytic expression
of the form
%
\begin{equation}
%
\phi(f) = A(f) e^{ i \psi(f) },
%
\end{equation}
%
where the phase takes the form
%
\begin{equation}
%
\psi(f) = \sum_{i = 0}^{7} \sum_{j = 0}^{1} \lambda_{i, j} f^{(i-5)/3} \log^j
f.
%
\end{equation}
%
The full expressions for the amplitude and phase are given in
Ap.~\ref{app:F2}.  Because the stationary phase approximation is generally
valid, the TaylorT2 and TaylorF2 approximants are nearly
indistinguishable~\cite{Droz:1999qx}. An advantage of the TaylorF2 approximant
comes from the fact that it can be analytically calculated in the frequency
domain.  In practice, waveforms that are generated in the frequency domain
without the use of integration are less computationally costly, and so searches
for  gravitational waves from inspiraling binary systems have been performed using the
TaylorF2 approximant~\cite{Blanchet:1996pi, Droz:1999qx, Blanchet:2006zz,
Abbott:2003pj, Abbott:2005pf, Abbott:2005pe, Abbott:2005qm, Abbott:2007xi,
Abbott:2009zi, Abbott:2009qj, Abadie:2010yba, Abadie:2011nz, Abbott:2007rh,
Abadie:2010uf, Briggs:2012ce}.

% Low Mass:
% S1:		Abbott:2003pj
% S2MACHO:	Abbott:2005pf
% S2BNS:	Abbott:2005pe
% S2LIGOTAMA:	Abbott:2005qm
% S3-S4:	Abbott:2007xi
% S5yr1:	Abbott:2009zi
% S5yr2:	Abbott:2009qj
% S5VSR1:	Abadie:2010yba
% S6:		Abadie:2011nz
%
% GRBs:
% 070201: Abbott:2007rh
% S5: Abadie:2010uf
% S6: Briggs:2012ce

%%%%%%%%%%%
\subsection{SEOBNRv1}

An additional approximant we use in this work is the spinning effective one-body model
(SEOBNRv1), presented in Ref.~\cite{Taracchini:2012ig}.  This approximant
incorporates the results of black hole perturbation theory, the self-force
formalism and \ac{PN} results. The model has been calibrated to numerical
relativity simulations, including simulations where the objects' spins were
(anti-) aligned with the orbital angular momentum and had magnitudes of $\chi
\pm 0.4$.  In order to compare these waveforms more fairly with the \ac{PN}
approximants that only model the inspiral, we truncate this model before the
merger section of the waveform. Although SEOBNRv1 is limited to $\chi \leq 0.6$,
further models have been developed since this work was completed that extend
the model to allow generic spins \cite{Taracchini:2013rva}.
