%%%%%%%%%%%%%%%%%%%%%%%%%%%%%%%%%%%%%%%%%%%%%%%%%%%%%%%%%%%%%%%%%%%%%%%%%%%%%%%
%%% Describe the wondrous conclusions of the thesis.
%%%%%%%%%%%%%%%%%%%%%%%%%%%%%%%%%%%%%%%%%%%%%%%%%%%%%%%%%%%%%%%%%%%%%%%%%%%%%%%

With the advent of Advanced LIGO and Advanced Virgo detectors, 
gravitational wave searches are expected to be able to see up to $10$
times further out in the universe, leading to a thousandfold increase in 
the volume of the accessible universe. Assuming a uniformly 
distributed population of compact object binaries in co-moving volume, 
we expect to see up to a three orders of magnitude increase in the detection 
rate. Gravitational wave searches make use of theoretical knowledge of
binary dynamics and employ modeled waveforms as filter templates.
With the increase in sensitivity, the resolution of the detectors 
for small errors in modeled waveforms also increases. In this dissertation,
we primarily focus on selecting and developing optimal waveform filters
for Advanced LIGO searches, as well as validating modeled search algorithms
using accurate simulated signals embedded in emulated detector noise.

For binaries with masses $m_1,m_2\leq 25M_\odot$, we compare the 
conventionally used post-Newtonian waveforms with the more recently 
developed Effective-One-Body (EOB) ones~\cite{BuonannoEOBv2Main}. 
As this EOB model is calibrated
against high-accuracy numerical simulations of non-spinning binary 
black holes, it is demonstrably an accurate model for {\it comparable}
mass-ratio binaries. However, it is computationally more expensive
than the post-Newtonian approximants. 
We investigate the region of the parameter
space of non-spinnning binaries where the accuracy of post-Newtonian
approximants is sufficient and we can win with computational cost, as
well as the region where EOB waveforms would be required. We also
study the impact of ignoring sub-dominant waveform multipoles in 
searches, and find that doing so decreases our ability to detect binaries
which have their orbital angular momentum highly inclined to the 
line of sight connecting them to terrestrial detectors.

More recently, very high accuracy fully numerical simulations of 
binary black holes have been performed, solving Einstein field equations
in full generality. While the methodology for performing such simulations
is being advanced at an accelerated pace, they are still expensive to 
perform for long durations of the coalescence process. However, hybrid
waveforms can be constructed where post-Newtonian approximants model the 
weak-field slow-motion portion and numerical relativity simulations 
model the strong-field fast-motion portion. We demonstrate that, within the 
limits of current numerical relativity technology, 
it is possible to use hybrid waveforms
in gravitational wave searches. Moreover, we show that this is possible in 
the entire region of the non-spinning binary parameter space where
post-Newtonian approximants are insufficient for Advanced LIGO searches.

Apart from having applications in enhancing the accuracy of theoretical 
and phenomenological waveform models, numerical simulations can also be used
to validate the gravitational wave search methods. We do precisely this within
the purview of the NINJA-2 project. Several numerical relativity groups contributed
post-Newtonian-hybridized simulations to the project. These were subsequently
injected in emulated advanced detector noise. We demonstrate the ability of
existing search algorithms to successfully {\it detect} these simulations
embedded within emulated noise. This is different from the NINJA-1 project
on a few counts, one of them being the nature of the emulated noise. In the 
NINJA-2 project, initial LIGO data with its non-Gaussian transient noise was
recolored to the expected sensitivity of the Advanced LIGO-Virgo detectors, as
opposed to colored Gaussian noise that was used in NINJA-1. 
Therefore this project provided a more robust test of our search methods, and 
provided a benchmark against which future search developments could be compared.


While the above concerns primarily comparable mass-ratio binaries, we 
also develop a waveform model for intermediate mass-ratio ones with 
$m_1/m_2 \in [10, 100]$. 
Intermediate mass-ratio systems, containing intermediate mass and stellar
mass black holes will also be relatively more massive than stellar mass binaries.
This would shift the frequency of the emitted gravitational radiation to 
lower values, and their late-inspiral and merger would occur in the most
sensitive frequency band of Advanced detectors. This makes the modeling 
of the later portion of their waveforms crucial to their detection. 
%
First-order conservative self-force corrections have been derived for a
test-particle moving in the background of a supermassive Schwarzschild 
black hole. Using the form of these calculations, we formulate a 
prescription to model the early and late inspiral
of such binaries. Then, using the implicit rotation source picture,
due to Baker et al~\cite{Baker:2008}, we develop a model for the plunge and merger
where the smaller object is no longer moving in quasi-circular orbits
and is very close to its partner. We then complete the description 
by stitching the quasi-normal ringdown waveform emitted by the black hole formed from 
the merger of the two initial holes. Therefore, we complete a model that
captures the entire coalescence process for intermediate mass-ratios.

To summarize, for {\it comparable} mass ratio binaries, we show that a combination
of post-Newtonian and post-Newtonian--Numerical-Relativity hybrid waveforms
would be sufficient for gravitational wave searches. This is true for the 
entire non-spinning binary black hole parameter space, up to arbitrarily high
masses. We also successfully validate gravitational wave search algorithms 
that have been used in the most recent LIGO-Virgo searches, using accurate 
numerical simulations injected in emulated detector noise. 
For {\it intermediate} mass ratios, we develop an accurate waveform 
model that captures the binary dynamics from the weak-field slow-motion
regime to the strong-field regime up to the merger of both compact objects. 
Therefore the work presented in this dissertation is an effort towards
arriving at optimal search filters for non-spinning binary black holes 
which are prospective sources detectable by the second-generation terrestrial
gravitational wave detectors; as well as towards validating existing search 
algorithms using an improved testing methodology.












