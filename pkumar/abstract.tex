%%%%%%%%%%%%%%%%%%%%%%%%%%%%%%%%%%%%%%%%%%%%%%
%%%%%%% ABSTRACT
%%%%%%%%%%%%%%%%%%%%%%%%%%%%%%%%%%%%%%%%%%%%%%

A direct observable consequence of General Relativity is gravitational
radiation. A direct detection of gravitational waves would not only be a much
awaited discovery, but also open up the possibility of doing astronomy and 
observing exotic astrophysical phenomena in a different light. There will 
also be an opportunity to test the theory of General Relativity in a strong 
field regime, which no past tests have probed. There are 
concerted efforts in the present era towards this aim. The major ground
based observatories include the LIGO detectors in the United States, Virgo
detector in Italy, GEO600 in Germany, the planned KAGRA detector in Japan
and the LIGO-India project in India. The second generation detectors are 
expected to begin observation next year, with the hope of reaching design
sensitivity in $2019$. 

Binary systems of compact astrophysical objects such as black holes and 
neutron stars emit gravitational waves as they move in orbit. With the emitted
radiation, they lose energy and momentum resulting in the shrinking of their 
orbit, leading eventually to a merger. These are also the sources of prime
importance for the ground-based detectors operating in the $10-2000$~Hz 
frequency band. As the gravitational waves become very weak by the time they 
reach us, instrumental noise usually overwhelms them in the detector data
making it a challenge to search for them. Contemporary and past gravitational
wave searches fold in the theoretical knowledge of expected signals by using
modeled waveforms to filter the data. This technique is called 
matched-filtering, and is highly sensitive to the accuracy of pre-known 
waveform templates. 

There has been considerable effort in the recent decades to develop systematic
approximations to the Einstein field equations of General Relativity. Most 
of such schemes use the assumption that the two bodies are moving slowly and 
are well separated. As the binary shrinks, these waveform {\it models} become 
increasingly inaccurate. More recently, numerical relativists have been 
able to solve the field equations of gravity numerically to high accuracy.
With the advent of this field, we have gained detailed knowledge of the last
stages of the binary motion before the black holes and/or neutron stars merge.
However, due to the pressing computational cost of numerical simulations, these
are restricted to the very last stages of binary coalescence. Some theoretical
techniques are able to interpolate between the post Newtonian regime and the 
numerical relativity regime, in an attempt to capture the entire inspiral and
merger process.

In this thesis we probe the question of accuracy of waveform models for 
comparable mass binaries. Computational cost is expected to be a significant
for advanced LIGO searches. We determine the domain of applicability of the
computationally inexpensive closed form models, and the same for the more 
expensive calibrated semi-analytic ones. We further explore the option of 
using {\it hybrid} waveforms constructed by numerically stitching post 
Newtonian and numerically simulated ones. Beyond matched-filtering, there 
is extensive processing of the filter output before a detection candidate can
be confirmed. We also utilize recent results from numerical relativity to study
the ability of LIGO searches to make detections in (recolored) detector noise.
Lastly, we develop two waveform models using the self-force formalism, one 
that covers the inspiral phase and the other that captures the entire 
coalescence process. This formalism was developed to model extreme mass-ratio
binaries, and we successfully extend it to intermediate mass-ratios.
