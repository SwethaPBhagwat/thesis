%%%%%%%%%%%%%%%%%%%%%%%%%%%%%%%%%%%%%%%%%%%%%%
%%%%%%% ABSTRACT
%%%%%%%%%%%%%%%%%%%%%%%%%%%%%%%%%%%%%%%%%%%%%%

A direct observable consequence of General Relativity is gravitational
radiation. The much anticipated direct detection of gravitational waves 
would open up a whole new spectrum for doing astronomy, and enable us to 
observe astrophysical phenomena that are yet unreachable by conventional 
electromagnetic astronomy. 
Gravitational-wave observations will also serve as strong field tests of
General Relativity. There is a concerted effort in the present era to 
detect gravitational waves, which includes the LIGO detectors in the United
States, the Virgo detector in Italy, GEO600 in Germany, the planned KAGRA
detector in Japan and the LIGO-India (IndiGo) project. The second generation
terrestrial detectors are expected to begin observation in $2015$, and will
reach their design sensitivity by $2018-19$.

Binary systems of compact astrophysical objects such as black holes and 
neutron stars emit gravitational waves as they orbit. The emitted 
waves carry away energy and angular momentum from the binary, causing the binary
orbit to shrink till it merges. Stellar-mass compact binaries are the primary
targets for gravitational wave detection with the second generation detectors, 
as they merge within the sensitive frequency band of the detectors (i.e.~$10-2000$~Hz).
Gravitational wave signals are buried within the instrument noise in detector
data, making it a challenge to search for them. 
Past gravitational wave searches with LIGO and Virgo detectors operated by 
matched-filtering the data with theoretically modeled waveform templates. 
The matched-filter is sensitive to the accuracy of waveform templates, and it 
is therefore crucial to have faithful compact binary coalescence models.

There has been steady progress in obtaining approximate solutions to Einstein's
equations for the two body problem. Several perturbative schemes have been 
developed to model the emitted gravitational waveforms. Most of
the perturbative approaches, however, assume that the binary motion is 
non-relativistic and orbit radius is large. As a result, the predicted 
waveforms become increasingly inaccurate as the binary approaches merger. 
An accurate description of compact binary mergers has become accessible
through Numerical Relativity simulations. Following the recent breakthroughs
in the field, numerical relativists have performed high accuracy simulations
probing different regions of the binary parameter space. 
% There has been considerable effort in the recent decades to develop systematic
% approximations to the Einstein field equations of General Relativity. Most 
% of such schemes use the assumption that the two bodies are moving slowly and 
% are well separated. As the binary shrinks, these waveform {\it models} become 
% increasingly inaccurate. More recently, numerical relativists have been 
% able to solve the field equations of gravity numerically to high accuracy.
% With the advent of this field, we have gained detailed knowledge of the last
% stages of the binary motion before the black holes and/or neutron stars merge.
However, due to the computational cost of numerical simulations, these currently
span only the very last stages of binary coalescence. In addition, there is ongoing
research on theoretical techniques that interpolate between the 
post-Newtonian regime and the Numerical Relativity regime, in an attempt to 
capture the entire inspiral and merger process.

In this thesis we study the accuracy of waveform models for comparable mass binaries.
% Computational cost is expected to be a significant for advanced LIGO searches. 
We determine the domain of applicability of the computationally inexpensive closed 
form models, and the same for the semi-analytic models that have been calibrated
to Numerical Relativity simulations (and are computationally more expensive).
We further explore the option of 
using {\it hybrid} waveforms constructed by numerically stitching post 
Newtonian and Numerical Relativity waveforms. Beyond matched-filtering, there 
is extensive processing of the filter output before a detection candidate can
be confirmed. We utilize recent results from Numerical Relativity to study
the ability of LIGO searches to make detections, using (recolored) detector data.
Lastly, we develop a waveform model, using recent self-force results, that 
coptures the complete binary coalescence process. The self-force formalism was
developed to model extreme mass-ratio binaries, and we successfully extend it 
to intermediate mass-ratios.
