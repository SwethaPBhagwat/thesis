\documentclass[a4paper,12pt]{article}
% \documentclass[a4paper,10pt]{scrartcl}
\addtolength{\topmargin}{-1.5in}
\advance\oddsidemargin-0.80in
\advance\evensidemargin-1.5cm
\textheight10.5in
\textwidth6.75in

\usepackage[utf8]{inputenc}
\usepackage{amsmath}
\usepackage{amssymb}

\title{DL11349/Kumar -- Response to the report of the Referee}
\author{Prayush Kumar, Ilana MacDonald, Duncan A. Brown,\\
Harald P. Pfeiffer, Kipp Cannon, Michael Boyle, \\
Lawrence E.~Kidder, Abdul H.~Mrou\'{e}, Mark A.~Scheel, \\
B\'{e}la Szil\'{a}gyi and An\i l Zengino\u{g}lu}
\date{\today}

\pdfinfo{%
  /Title    (DL11349/Kumar -- Responses to the report of the Referee.)
  /Author   (Prayush Kumar)
  /Creator  ()
  /Producer ()
  /Subject  ()
  /Keywords ()
}

\begin{document}
\maketitle

We would like to thank the referee for their careful reading of the 
manuscript, and their suggestions. All of their remarks have been 
incorporated in the manuscript, and were helpful in clarifying the 
information presented in the article.
\newline

\textbf{Report of the Referee -- DL11349/Kumar}

\textit{The authors study the feasibility of searching for gravitational waves
from the inspiral and merger of binary black hole systems using
template banks constructed from numerical relativity waveforms, and
hybrid post-Newtonian/numerical relativity waveforms. This approach
potentially avoids the need for model waveforms which must be tuned
against numerical relativity, and could therefore streamline the
process by which new simulations are employed in searches.}

\textit{The authors conclude that, despite the discreetness in mass ratio
imposed by the nature and computation expense of NR simulations, a
fairly small set of existing simulations is sufficient to construct an
effectual bank over a significant range of masses and mass ratios.
This result is interesting and important to the gravitational wave
science community. The authors then extend their work to consider the
length and accuracy requirements on future numerical simulations in
order to ensure coverage down to 12 solar masses, at which point pure
post-Newtonian waveforms have been found to be effectual. This
extension will be of interest to both those searching for
gravitational waves and numerical relativists. I recommend this
manuscript be published in PRD.}

\textit{I have only a few minor remarks:}
\newline

\textbf{Comments:}
\vspace{8pt}

{\bf 1.} \textit{"We find that the EOBNRv2 manifold is a reasonable approximation
for the hybrid manifold." I would suggest alluding to this result
earlier, as while reading the methods I had wondered if the use of
EOBNRv2 to model signals would introduce any uncertainty into the
results.}
\vspace{8pt}

This was mentioned in Sec.III, where we describe the method for
quantification of template bank effectualness (above Eq. 18). I have
added the following text in the Introduction to clarify this point
earlier:

\vspace{5pt}
``The bank placement algorithm uses the EOB
model from Ref.~[53] (EOBNRv2). As this model was cal-
ibrated against NR for most of these mass-ratios, we expect
the manifold of EOBNRv2 to be a reasonable approxima-
tion for the NR manifold. In Sec.~V, we demonstrate that
this approximation holds well for NR-PN hybrids as well. "
\vspace{8pt}

{\bf 2.} \textit{"... domain pseudospectral code to solve Einsteins equations." $\rightarrow$ Einstein's}
\vspace{8pt}

This has been changed as suggested.
\vspace{12pt}

{\bf 3.} \textit{In Table I it would be useful to include a columns giving the
lowest mass for which the waveform starts at 15Hz.}
\vspace{8pt}

Table~I of the manuscript has been updated, adding a column with the 
binary masses for which the respective simulations start at $15$~Hz, 
as shown here in Table~\ref{tableNR}
\begin{table}
\centering
\begin{tabular}{| c | c | c | c |}
\hline
$\hspace{10pt}\eta\hspace{10pt}$ & \hspace{15pt} q\hspace{15pt} & Length (in orbits) & Minimum total mass ($M_\odot$)\\ \hline
0.25 & 1 & 33 & 49 \\
0.2222 & 2 & 15 & 76 \\
0.1875 & 3 & 18 & 82 \\
0.1600 & 4 & 15 & 87 \\
0.1224 & 6 & 20 & 83 \\
0.0988 & 8 & 25 & 83 \\
%0.0884 & 9.2 & ?? \\
\hline
\end{tabular}
\caption{See text for details}
\label{tableNR}
\end{table}

% q \omega_0
% 1 0.025 [http://arxiv.org/pdf/1210.3007v2.pdf]
% 8 ? 
% 2 0.0176711 [http://arxiv.org/pdf/1206.3015v2.pdf]
% 3 0.0189994
% 4 0.0203077
% 6 0.01935244
\vspace{12pt}

{\bf 4.} \textit{"And the coverage fraction of the bank is 99\%." Coverage fraction
was not defined.}
\vspace{8pt}

The description of the termination condition for the stochastic 
bank construction has been rephrased as: 
"The process is repeated till the fraction of 
proposals being accepted falls below $\sim 10^{-4}$, and $\gtrsim 99\%$
of the parameter space is covered effectually."
\vspace{12pt}

{\bf 5.} \textit{"... pushing their mass-ratios to the two neighboring mass-ratio
from $\mathcal{S}_q$" I suspect it does not matter very much, but was the pushing
done along lines of constant total mass or chirp mass?}
\vspace{8pt}

The uncovered points are pushed to the nearest allowed mass-ratio values
along lines of constant chirp mass. This has been clarified in the paper as: 
"... pushing their mass-ratios to the two neighboring mass-ratios from
$\mathcal{S}_q$ along lines of constant chirp mass"

\vspace{12pt}

{\bf 6.} \textit{ "The detectors will be relatively very sensitive to a relatively
short frequency band." This reads a little awkwardly.}
\vspace{8pt}

This has been rephrased as: "The detectors will be most sensitive in a 
comparatively narrow frequency band."
\vspace{12pt}

{\bf 7.} \textit{"The spacing between neighboring templates is given by requiring
that the overlap between them be 97%. We take the union of these banks
as the final two-dimensional bank." Again I suspect it does not matter
much, but how was the first point along each q line chosen?}
\vspace{8pt}

The first point, i.e. the lowest total mass template, was chosen to
be the one for which the hybrid mismatch is 3\%. As the hybrid mismatches
grow at lower masses, it is reasonable to choose the 
first point where the mismatches are almost at the threshold.
This has been clarified as:

\vspace{5pt}
``The template with the lowest total mass is chosen by requiring the
 hybrid mismatch to be $3\%$ at that point. The spacing between 
 neighboring templates is given by requiring that the overlap between
 them be $97\%$. We take the union of these banks as the final 
 two-dimensional bank."


% \newline

\end{document}
