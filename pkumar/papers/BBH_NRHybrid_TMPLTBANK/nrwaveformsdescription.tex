
The numerical relativity waveforms used in this paper were produced
with the SpEC code~\cite{spec}, a multi-domain pseudospectral code to solve
Einstein’s equations. SpEC uses Generalized Harmonic coordinates,
spectral methods, and a flexible domain decomposition, all of which
contribute to it being one of the most accurate and efficient codes
for computing the gravitational waves from binary black hole
systems. High accuracy numerical simulations of the late-inspiral,
merger and ringdown  of coalescing binary black-holes have been
recently performed for mass-ratios $q\equiv m_1/m_2\in\{1,2,3,4,6,8\}$
~\cite{Buchman:2012dw,Scheel:2008rj,NRPNComparisonBoyleetal,Mroue:2012kv}.

The equal-mass,
non-spinning waveform covers 33 inspiral orbits and was first discussed
in~\cite{MacDonald:2012mp,Mroue:2012kv}. 
This waveform was obtained with numerical techniques similar to those 
of~\cite{Buchman:2012dw}. The unequal-mass waveforms of mass ratios 
$2, 3, 4,$ and $6$ were presented in detail in Ref.~\cite{Buchman:2012dw}.
The simulation with mass ratio $6$ covers about 20 orbits and the
simulations with mass ratios 2, 3, and 4 are somewhat shorter and
cover about 15 orbits. The unequal mass waveform with mass ratio 8 was
presented as part of the large waveform catalog
in~\cite{Mroue:2013xna,Mroue:2012kv}. It is approximately 25 orbits in
length. 
% As it becomes possible to simulate BBH evolution for longer times, in our 
% template bank construction we assume that simulations that span $\gtrsim 20$
% orbits, including the inspiral, merger and ringdown cycles, are (will be) 
% available by the time Advanced LIGO begins observation runs. 
We summarize the NR simulations used in this study in 
Table~\ref{table:etalist4}.

\begin{table}
\begin{tabular}{| c | c | c |}
\hline
$\eta$ & q & Length (in orbits)\\ \hline
0.25 & 1 & 33 \\
0.2222 & 2 & 15 \\
0.1875 & 3 & 18 \\
0.1600 & 4 & 15 \\
0.1224 & 6 & 20 \\
0.0988 & 8 & 25 \\
%0.0884 & 9.2 & ?? \\
\hline
\end{tabular}
\caption{SpEC BBH simulations used in this study.  Given are symmetric mass-ratio $\eta$, mass-ratio $q=m_1/m_2$, and the length in orbits of the simulation.}
\label{table:etalist4}
\end{table}
