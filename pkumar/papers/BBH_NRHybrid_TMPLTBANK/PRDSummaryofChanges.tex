\documentclass[a4paper,12pt]{article}
% \documentclass[a4paper,10pt]{scrartcl}
\addtolength{\topmargin}{-1.5in}
\advance\oddsidemargin-0.80in
\advance\evensidemargin-1.5cm
\textheight10.5in
\textwidth6.75in

\usepackage[utf8]{inputenc}
\usepackage{amsmath}
\usepackage{amssymb}

\title{DL11349/Kumar -- Summary of changes}
\author{Prayush Kumar, Ilana MacDonald, Duncan A. Brown,\\
Harald P. Pfeiffer, Kipp Cannon, Michael Boyle, \\
Lawrence E.~Kidder, Abdul H.~Mrou\'{e}, Mark A.~Scheel, \\
B\'{e}la Szil\'{a}gyi and An\i l Zengino\u{g}lu}
\date{\today}

\pdfinfo{%
  /Title    (DL11349/Kumar -- Summary of changes)
  /Author   (Prayush Kumar)
  /Creator  ()
  /Producer ()
  /Subject  ()
  /Keywords ()
}

\begin{document}
\maketitle

We summarize the changes that have been made to the manuscript, in response
to the referee's comments and suggestions. We would like to the thank the 
referee for carefully reading through the manuscript and their constructive
suggestions.
\newline

\textbf{As suggested by the referee:}
\newline


{\bf 1.} 
We use the manifold of Effective-One-Body (EOBNRv2) waveforms for 
the construction of template banks. In the first version of the 
manuscript, this was justified in Section~III, where we describe 
the method for quantification of template bank effectualness 
(above Eq. 18). To clarify this reasoning earlier to the reader, 
the following text has been added to the Introduction:

{\it ``The bank placement algorithm uses the EOB
model from Ref.~[53] (EOBNRv2). As this model was calibrated 
against NR for most of these mass-ratios, we expect
the manifold of EOBNRv2 to be a reasonable approximation 
for the NR manifold. In Sec.~V, we demonstrate that
this approximation holds well for NR-PN hybrids as well."}
\vspace{8pt}

{\bf 2.} 
In Section~II~A, a typographical error has been rectified, changing

% \vspace{5pt}
\textit{``... Einsteins equations."} to {\it ``... Einstein's equations."}
\vspace{8pt}

{\bf 3.}
A column has been added to Table~I, which includes information about 
the starting frequency of each of the listed Numerical Relativity simulation.
This information is presented in the form of the value of the binary mass
for which the gravitational-waves extracted from the (respective) simulation
start at $15$~Hz. This particular frequency is significant because it 
is expected to be the lower bound on the sensitive frequency band of
the Advanced LIGO-Virgo detectors.
\vspace{8pt}

{\bf 4.} 
In Section~IV, we describe the stochastic construction of template 
banks. As the fraction of accepted proposals decreases, the fraction 
of the total parameter space, that is covered at the desired density 
increases. Both asymptotically approach $0$ and $1$, respectively, 
as the bank construction algorithm converges. We terminate 
the Monte Carlo sampling process when the fraction of the parameter 
space covered is $\gtrsim 0.99$. In the first version of the manuscript, 
this was described as: \textit{``And the coverage fraction of the bank is
99\%."} This has been re-phrased as: 
{\it ``The process is repeated till the fraction of 
proposals being accepted falls below $\sim 10^{-4}$, and $\gtrsim 99\%$
of the parameter space is covered effectually."}
\vspace{8pt}

{\bf 5.}
In Section~IV, we have clarified that while completing the coverage
of the stochastically constructed bank, we push the points sampled in 
the under-covered regions of the parameter space to the closest allowed
values of the mass-ratio -- along lines of constant chirp mass.
We have changed \textit{``\dots pushing their mass-ratios to the two 
neighboring mass-ratio from $\mathcal{S}_q$"} to 
{\it ``\dots pushing their mass-ratios to the two neighboring mass-ratios
from $\mathcal{S}_q$ along lines of constant chirp mass".}

\vspace{8pt}

{\bf 6.}
In Section~V, the following statement about the sensitivity of the 
Advanced LIGO-Virgo detectors, 
\textit{``The detectors will be relatively very sensitive to a relatively
short frequency band."}, has been rephrased as: {\it ``The detectors will
be most sensitive in a comparatively narrow frequency band."}
\vspace{8pt}

{\bf 7.}
In Section~V, we describe two independent methods of constructing
hybrid waveform template banks. For the second method, the 
spacing between neighboring templates is obtained by requiring a 
fixed value of inner product between them. We have added a description 
of how the first template (for each unique value of mass-ratio) is chosen.
We replaced the following text, 
\textit{``The spacing between neighboring templates is given by requiring
that the overlap between them be 97\%. We take the union of these banks
as the final two-dimensional bank."}, with
{\it ``The template with the lowest total mass is chosen by requiring the
 hybrid mismatch to be $3\%$ at that point. The spacing between 
 neighboring templates is given by requiring that the overlap between
 them be $97\%$. We take the union of these banks as the final 
 two-dimensional bank."}
\newline
 \vspace{8pt}

\textbf{Other changes:}
\newline

{\bf 1.}
The second paragraph in Section~V (describing Figure~5)
gives the orbital frequencies 
at which the post-Newtonian waveforms were stitched to the Numerical 
Relativity waveforms to obtain the hybrids that we use in this work. 
Also given is the number of orbits -- {\it before merger} -- that these 
frequencies correspond to. We discovered that, due to the inclusion of 
ringdown cycles, the numbers of orbits stated were
artificially elevated. These have been re-computed, and corrected in
the manuscript, making sure that the number of orbits stated in the 
updated version 
include only the inspiral and merger portions of the waveform
(and not the {\it ringdown}).
The following text, {\it ``In terms of number of orbits before merger,
this is 31.9 orbits for $q=1$, 17.8 orbits for $q=2$,
16.9 orbits for $q=3$, 18.4 orbits for $q=4$, 21.6 orbits for $q=6$,
and 25.1 orbits for $q=8$."}, has been changed to {\it ``In terms of
number of orbits before merger, this is 26.9 orbits for $q=1$, 13.6 
orbits for $q=2$, 12.6 orbits for $q=3$, 14.3 orbits for $q=4$, 
17.8 orbits for $q=6$, and 21.4 orbits for $q=8$."}. The updated 
information is consistent with Table~I.
\vspace{8pt}

% function of total mass. In terms of orbital frequency, these are
% matched at $M\omega_m=0.025$ for $q=1$, $M\omega_m=0.038$ for $q=2$,
% and $M\omega_m=0.042$ for $q=3,4,6,8$. In terms of number of orbits
% -before merger, this is 26.9 orbits for $q=1$, 13.6 orbits for $q=2$,
% -12.6 orbits for $q=3$, 14.3 orbits for $q=4$, 17.8 orbits for $q=6$,
% -and 21.4 orbits for $q=8$. The dotted line indicates a mismatch of
% +before merger, this is 31.9 orbits for $q=1$, 17.8 orbits for $q=2$,
% +16.9 orbits for $q=3$, 18.4 orbits for $q=4$, 21.6 orbits for $q=6$,
% +and 25.1 orbits for $q=8$. 
{\bf 2.}
In Figure~5 and its inset figure, the black circles indicate the 
lowest binary total masses for which the Numerical Relativity
waveforms start at $15$~Hz. We discovered a problem with the formula
used to compute these masses (only for the ``NR-only" points). 
This figure has been updated with the correct values for all 
mass-ratios. The updated information is consistent with Table~I.


% \newline

\end{document}
