
%\red{\em Describe how we derive an error-bound on a PN-NR hybrid waveform.}

The largest source of error in hybrid gravitational waveforms is
caused by the higher-order unknown terms in the PN component. These
cause different PN approximants to diverge as the binary black holes approach
merger. In order to ascertain what this error is, we compare many
types of PN waveforms and find the maximum mismatch between hybrids
which use different PN approximants, in this case, Taylor T1, T2, T3,
and T4. We assume this to be close to the actual PN error, as discussed in 
Sec.~\ref{s2:quantifyingerrors}.

More specifically, we take four hybrid waveforms $h_\text{Tn}$, where $n = [1,2,3,4]$, and find their mismatches as defined in Eq.~\ref{eq:mismatch}:
\begin{equation}
\mathcal{M}_\text{max}(\eta,M) = \underset{(i,j)}{\mx}\,\underset{M''}{\mn}\,\,  \mathcal{M}\left(h^{\mathrm{T}i}(\eta,M), h^{\mathrm{T}j}(\eta,M'')\right) 
\end{equation}
$\mathcal{M}_\text{max}$ is what we call the hybridization error.

Because NR waveforms have a limited number of orbits, to obtain results for hybrids with lower matching frequencies, we replace the NR part of the hybrids with EOBNRv2 waveforms as described in Sec.~\ref{s2:EOBwaveforms}. Since only the PN waveforms are changing in these hybrid comparisons, using EOB hybrids gives the same results as using NR hybrids.
