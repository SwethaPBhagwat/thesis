

Post-Newtonian (PN) theory is a perturbative approach to describing the
motion of a compact object binary, during the slow-motion and weak-field 
regime, i.e. the inspiral phase. The conserved energy of a binary in orbit,
$E$, has been calculated to 3PN order in literature~\citep{Jaranowski:1997ky,
Jaranowski:1999ye,Jaranowski:1999qd,Damour:2001bu,Blanchet:2003gy,
Damour:2000ni,Blanchet:2002mb}.
Using the adiabatic approximation, we treat the course of inspiral as a series
of radially shrinking circular orbits. This is valid during the inspiral when
the angular velocity of the binary evolves more slowly than the orbital 
time-scale. The radial separation shrinks as the binary loses energy to 
gravitational radiation that propagates outwards from the system. 
% For a binary with individual component masses $m_1$ and $m_2$ and total mass
%$M=m_1+m_2$, the conserved energy accurate to 3PN
% can be written as \citep{Jaranowski:1997ky,Jaranowski:1999ye,Jaranowski:1999qd,Damour:2001bu,Blanchet:2003gy,Damour:2000ni,Blanchet:2002mb}
% \begin{equation}
% \begin{split}\label{eq:E3PN}
% E_3(v)=&-\dfrac{1}{2}\eta v^2 \left[1- \l(\dfrac{3}{4}+\dfrac{1}{12}\eta\r)v^2 - \l(\dfrac{27}{8}-\dfrac{19}{8}\eta\right.\right.\\
% +&\left.\left.\dfrac{1}{24}\eta^2 \r)v^4 - \l(\dfrac{675}{64}-\l(\dfrac{34445}{576}-\dfrac{205}{96}\pi^2\r)\eta\right.\right.\\
% +&\left.\left.\dfrac{155}{96}\eta^2 +\dfrac{35}{5184}\eta^3\r) v^6\right],
% \end{split}
% \end{equation}
% where $\eta=m_1m_2/(m_1+m_2)^2$, $v=(\pi Mf)^{1/3}$ is the characteristic velocity of the binary, and $f$ denotes the frequency of the emitted gravitational wave throughout.
The energy flux from a binary $F$ is known in PN theory to 3.5PN 
order~\cite{FluxandE3-5PN,Blanchet:2004ek,Blanchet:2005tk,Blanchet:2004bb}.
% \begin{equation}
% \begin{split}\label{eq:Ft3.5PN}
% F_{3.5}(v)=&\dfrac{32}{5}\eta^2 v^{10}\left[1 - \l(\dfrac{1247}{336}+\dfrac{35}{12}\eta\r)v^2+4\pi v^3\right.\\
% -&\left.\l(\dfrac{44711}{9072}-\dfrac{9271}{504}\eta -\dfrac{65}{18}\eta^2 \r)v^4\right.\\
% -&\left.\l(\dfrac{8191}{672}+\dfrac{583}{24}\eta\r)\pi v^5+ \l(\dfrac{6643739519}{69854400}\right.\right.\\
% +&\left.\left.\dfrac{16}{3}\pi^2 -\dfrac{1712}{105}\gamma +\l(\dfrac{41}{48}\pi^2 -\dfrac{134543}{7776}\r)\eta \right.\right.\\
% -&\left.\left.\dfrac{94403}{3024}\eta^2 -\dfrac{775}{324}\eta^3 -\dfrac{856}{105}\textrm{log}(16v^2)\r)v^6\right.\\ 
% -&\left.\l(\dfrac{16285}{504}-\dfrac{214745}{1728}\eta -\dfrac{193385}{3024}\eta^2 \r)\pi v^7\right],
% \end{split}
% \end{equation}
% where $\gamma$ is Euler's constant. In the limit $\dot{\omega}/\omega \ll 1$, 
% we can approximate the energy of the system to be the energy averaged over a 
% period. 
Combining the energy balance equation, $\D E/\D t = -F$, with Kepler's law 
gives a description of the radial and orbital phase evolution of the binary. 
We start the waveform where the GW frequency enters the sensitive frequency 
band of advanced LIGO, i.e. at $15$Hz. 
% \begin{subequations}\label{eq:PNOrbitalEvolution01}
% \begin{align}
% \dfrac{\D\phi}{\D t} - \dfrac{v^3}{M} &= 0,\label{eq:PNOrbitalEvolution01_01}\\
% \dfrac{\D v}{\D t} + \dfrac{F(v)}{M\l(\D E/\D v\r)} &= 0.\label{eq:PNOrbitalEvolution01_02}
% \end{align}
% \end{subequations}
Depending on the way the expressions for orbital energy and flux
%in Eq.~\ref{eq:E3PN} \& \ref{eq:Ft3.5PN} 
are combined to obtain the coordinate evolution for the binary,
%to solve Eq.~\ref{eq:PNOrbitalEvolution01}, 
we get different Taylor\{T1,T2,T3,T4\} time-domain approximants. Using the 
stationary phase approximation~\cite{MatthewsWalker}, frequency-domain 
equivalents of these approximants, i.e. TaylorF$n$, can be constructed. 
Past GW searches have extensively used the TaylorF2 approximant, as it has a
closed form and mitigates the computational cost of generating and numerically
fourier-transforming time-domain template~\cite{Colaboration:2011nz,Abadie:2010yb,
Abbott:2009qj,Abbott:2009tt,Messaritaki:2005wv}.
We refer the reader to Ref.~\cite{PNtheoryLivingReviewBlanchet,JolienGWPhysAst}
for an overview.
From the coordinate evolution, we obtain the emitted gravitational waveform;
approximating it by the quadrupolar multipole $h_{2,\pm 2}$ which is the 
dominant mode of the waveform.
% in previous waveform-related studies~\cite{CompTemplates2001,CompTemplates2009,
% Miller:2005qu,NRPNComparisonBoyleetal,NRPNComparisonBaker2007,Boyle:2011dy,
% MacDonald:2011ne,Brown:2012nn}.

