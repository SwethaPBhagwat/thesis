

Full numerical simulations are available for a limited number of binary 
mass combinations. We use a recently proposed EOB 
model~\cite{BuonannoEOBv2Main}, which we refer to as EOBNRv2, as a substitute
to model the signal from binaries with arbitrary component masses
in this paper. This model was calibrated to most of the numerical simulations
that we use to construct templates banks, which span the range of masses 
we consider here well. So we expect this approximation to hold. We describe the
model briefly here.

The EOB formalism maps the dynamics of a two-body system onto an effective-mass
moving in a deformed Schwarzschild-like background~\citep{EOBOriginalBuonannoDamour}.
The formalism has evolved to use Pad\'{e}-resummations of perturbative 
expansions calculated from PN theory, and allows for the introduction of higher
(unknown) order PN terms that are subsequently calibrated against NR 
simulations of BBHs 
\cite{EOBdevel01,EOBdevel02,EOBNRdevel03,DamourFluxhlm01,EOBNRdevel01}. The EOB 
model proposed recently in Ref.~\citep{BuonannoEOBv2Main} has been calibrated 
to SpEC NR waveforms for binaries of mass-ratios $q=\{1,2,3,4,6\}$, where 
$q\,\equiv \, m_1/m_2$, and is the one that we use in this paper (we will refer
to this model as EOBNRv2).

The dynamics of the binary enters in the metric coefficient of the deformed
Schwarzschild-like background, the EOB Hamiltonian \cite{EOBOriginalBuonannoDamour}, 
and the radiation-reaction force. 
% In polar coordinates $(r,\Phi)$, the EOB 
% metric is written as
% \begin{equation}\label{eq:dsEOB}
% \D s_{\eff}^2 = -A(r)\D t^2 + \dfrac{A(r)}{D(r)}\D r^2 + r^2\left(\D\Theta^2 + \sin^2\Theta \D\Phi^2\right).
% \end{equation}
These
%metric coefficients $A(r)$ and $D(r)$ 
are known to 3PN order \cite{EOBOriginalBuonannoDamour,PadeAD} from PN theory.
The 4PN \& 5PN terms were introduced in the potential $A(r)$, which was 
Pad\'{e} resummed and calibrated to NR simulations
\citep{EOBNRdevel01,EOBNRdevel02,EOBNRdevel03,EOBNRdevel04,BuonannoEOBv2Main}.
We use the resummed potential calibrated in Ref.~\cite{BuonannoEOBv2Main} 
(see Eq.~(5-9)). The geodesic dynamics of the reduced mass 
$\mu\,=\,m_1 m_2 / M$ in the deformed background 
%of Eq.~\eqref{eq:dsEOB} 
is described by
the Hamiltonian $H^{\eff}$ given by Eq.~(3) in \cite{BuonannoEOBv2Main}.
% , which can be written as \cite{EOBOriginalBuonannoDamour,PadeAD}
% \begin{equation}
% \begin{split}
% H^{\eff} =\, & \mu\hat{H}^{\eff} \\
%          =\, & \mu\sqrt{A(r) \left( 1 +  \dfrac{A(r)}{D(r)}p_r^2 + 2(4 - 3\eta)\eta \dfrac{p_r^4}{r^2} + \dfrac{p^2_{\Phi}}{r^2} \right)},
% \end{split}
% \end{equation}
% where $(p_r,p_{\Phi})$ are momenta conjugate to the $(r,\Phi)$ coordinates. 
The Hamiltonian describing the conservative dynamics of the binary
(labeled the \textit{real} Hamiltonian $H^{\real}$) is related to 
$\hat{H}^{\eff}$ as in Eq.~(4) of \cite{BuonannoEOBv2Main}.
% \begin{equation}
% H^{\real} = \mu\hat{H}^{\real} = M \sqrt{1 + 2\eta (\hat{H}^{\eff} - 1)}.
% \end{equation}
The inspiral-merger dynamics can be obtained by numerically solving the 
Hamiltonian equations of motion for $H^{\real}$, see e.g. Eq.(10)
of~\cite{BuonannoEOBv2Main}. 

The angular momentum carried away from the binary
by the outwards propagating GWs results in a radiation-reaction force
%$\hat{F}_{\Phi}$ 
that causes the orbits to shrink.
% ,
% \begin{eqnarray}
%\dfrac{\D r}{\D\hat{t}} &\equiv & \dfrac{\partial \hat{H}^{\real}}{\partial p_r} = \dfrac{A(r)}{\sqrt{D(r)}}\dfrac{\partial \hat{H}^{\real}}{\partial p_{r*}} (r, p_{r*}, p_{\Phi}) ,\\
%\dfrac{\D\Phi}{\D\hat{t}} &\equiv & \hat{\Omega} = \dfrac{\partial \hat{H}^{\real}}{\partial p_{\Phi}} (r, p_{r*}, p_{\Phi}) , \\ 
%\dfrac{\D p_{r_*}}{\D\hat{t}} &=& -\dfrac{A(r)}{\sqrt{D(r)}} \dfrac{\partial \hat{H}^{\real}}{\partial r} (r, p_{r*}, p_{\Phi}) ,\\
% \dfrac{\D p_{\Phi}}{\D\hat{t}} &=& \hat{F}_{\Phi},%(r, p_{r*}, p_{\Phi}) ;
% \end{eqnarray}
% where, $\hat{t}\equiv t/M$ is time in dimensionless units; and
% \begin{equation}
%\hat{F}_{\Phi} = -\dfrac{1}{\eta \hat{\Omega}} \dfrac{\D E}{\D t} = -\dfrac{1}{\eta v^3} \dfrac{\D E}{\D t},
% \hat{F}_{\Phi} = -\dfrac{1}{\eta v^3} \dfrac{\D E}{\D t},
% \end{equation}
% where, $v=\hat{\Omega}^{1/3}=(\pi Mf)^{1/3}$ as before, and $f$ is the 
% instantaneous frequency of the emitted GWs. 
This is due to the flux of energy from the binary, which
%flux $\D E/\D t$ 
is obtained by summing over the contribution from each term in the multipolar
decomposition of the inspiral-merger EOB waveform.
% , i.e.
% \begin{equation}\label{eq:definedEdt}
% \frac{\D E}{\D t} = \frac{\hat{\Omega}^2}{8\pi} \Sum_{l}\Sum_{m} \left|\frac{\mathcal{R}}{M} h_{lm}\right|^2,
% \end{equation}
% where $\mathcal{R}$ is the physical distance to the binary, and 
%$h_{lm}$, which are 
% the EOB waveform multipoles, 
%defined implicitly as
%  \begin{equation}\label{eq:hlmdef}
%  h_+ - \ii h_{\times} = \dfrac{M}{\mathcal{R}} \Sum^{\infty}_{l=2} \Sum^{m=l}_{m = -l} Y^{lm}_{-2}\, h_{lm},
%  \end{equation}
%  where $Y^{lm}_{-2}$ are spin $-2$ weighted spherical harmonics, 
%  and $h_{+,\times}$ are the two orthogonal polarizations of the incoming GW. 
% These multipoles 
% %$h_{lm}$ 
% are functions of the orbital phase of the binary $\propto e^{-\ii m\Phi}$.
Complete resummed expressions for these multipoles~\cite{DamourFluxhlm01} can be 
read off from Eq.(13)-(20) of Ref.~\cite{BuonannoEOBv2Main}. In this paper, 
as for PN waveforms, we model the inspiral-merger part 
%$l=2\ldots 8$ to compute the energy flux, and approximate the 
%summation in Eq.~\ref{eq:hlmdef} 
by summing over the dominant $h_{2,\pm 2}$ multipoles.

The EOB merger-ringdown part is modeled as a sum of $N$ quasi-normal-modes 
(QNMs),
% \begin{equation}
% h_{lm}^{\RD}(t) = \Sum^{N-1}_{n=0}A_{lmn}e^{-\ii\sigma_{lmn}(t-t_{lm}^{\mathrm{match}})},
% \end{equation}
where $N=8$ for EOBNRv2~\citep{EOBNRdevel01,EOBNRdevel02,EOBNRdevel04,BHRDQNMs}.
The ringdown frequencies depend on the mass and spin of the BH that is formed 
from the coalescence of the binary. The inspiral-merger and ringdown parts are
attached by matching them at the time at which the amplitude of the 
inspiral-merger waveform peaks.
%, i.e. at $t_{lm}^{\mathrm{match}}=t^{lm}_{\peak}$
~\citep{EOBNRdevel01,BuonannoEOBv2Main}. The matching procedure followed
% involves equating the complex amplitude and
% phase (and its derivative) of $h_{lm}(t)$ and $h_{lm}^{\RD}(t)$ over a small 
% interval of time ending at $t_{lm}^{\mathrm{match}}$, from which we obtain the
% complex amplitudes $A_{lmn}$. 
is explained in detail Sec.~II~C of Ref.~\citep{BuonannoEOBv2Main}.
%gives further details of this procedure. 
By combining them, we obtain the complete waveform for a BBH system.
% We combine the inspiral-merger waveform $h_{lm}(t)$ and the ringdown 
% waveform $h^{\RD}(t)$ to obtain the complete inspiral-merger-ringdown EOB 
% waveform $h^{\textrm{IMR}}(t)$, i.e.
% \begin{equation}
% h^{\textrm{IMR}}_{lm}(t) = h_{lm}(t)\Theta(t^{\mathrm{match}}_{lm}-t) + h^{\RD}(t)\Theta(t-t^{\mathrm{match}}_{lm}),
% \end{equation}
% where $\Theta(x)=1$ for $x\geq 0$, and 0 otherwise. These multipoles are then
% combined to give the two orthogonal polarizations of the gravitational 
% waveform, $h_+$ and $h_{\times}$, as in Eq.~\ref{eq:hlmdef}.
