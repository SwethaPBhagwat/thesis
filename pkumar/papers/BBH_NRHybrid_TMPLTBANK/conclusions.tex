
The upgrades currently being installed to increase the sensitivity of the
ground based interferometric gravitational-wave detectors LIGO and
Virgo~\cite{Harry:2010zz,aVIRGO} are scheduled to complete within the next
two years. The second generation detectors will have a factor of $10$ better 
sensitivity across the sensitive frequency band, with the lower frequency
limit being pushed from $40$Hz down to $\sim 10$Hz. 
They will be able to detect GWs from stellar-mass BBHs up to distances of a few
Gpc, with the expected frequency of detection between 
$0.4 - 1000\, yr^{-1}$~\cite{LSCCBCRates2010}.

Gravitational-wave detection searches for BBHs operate by matched-filtering
the detector data against a bank of modeled waveform 
templates~\cite{Babak:2012zx,Sathyaprakash:1991mt, SathyaMetric2PN,
OwenTemplateSpacing,BabaketalBankPlacement,SathyaBankPlacementTauN,
Cokelaer:2007kx}.
Early LIGO-Virgo searches employed PN waveform template banks
that spanned only the inspiral phase of the
coalescence~\cite{Colaboration:2011nz,Abadie:2010yb,Abbott:2009qj,
Abbott:2009tt,Messaritaki:2005wv}.
Recent work has shown that a similar bank of PN templates would be
effectual for the advanced detectors, to detect non-spinning BBHs with 
$m_1 + m_2 \lesssim 12M_\odot$~\cite{Brown:2012nn,CompTemplates2009}.
Searches from the observation period between $2005-07$ and $2009-10$ 
employed templates that also included the late-inspiral, merger and 
ringdown phases of binary coalescence~\cite{Abadie:2011kd,Aasi:2012rja}. 

Recent advancements in Numerical Relativity have led to high-accuracy 
simulations of the late-inspiral and mergers of BBHs. The multi-domain SpEC
code~\cite{spec} has been used to perform simulations for non-spinning 
binaries with mass-ratios
$q=1,2,3,4,6,8$~\cite{Buchman:2012dw,Mroue:2012kv,Mroue:2013inPrep}. 
Owing to their high computational complexity, the length of these
simulations varies between $15-33$ orbits. Accurate modeling of the
late-inspiral and merger phases is important for stellar mass BBHs,
as they merge at frequencies that the advanced detectors would be 
sensitive to~\cite{Brown:2012nn}. Analytic models, like those within the 
Effective-One-Body formalism, have been calibrated to the NR simulations 
to increase their accuracy during these 
phases~\cite{EOBOriginalBuonannoDamour,EOBNRdevel01,BuonannoEOBv2Main,
Taracchini:2012ig}. Other independent models have also been developed 
using information from NR simulations and their hybrids~\cite{NRAR:home,
Ajith:2007qp,Santamaria:2010yb,Huerta:2012zy}. An alternate derived 
prescription is that of NR+PN hybrid waveforms, that
are constructed by joining long PN early-inspirals and late-inspiral-merger
simulations from NR~\cite{Boyle:2011dy,MacDonald:2011ne,MacDonald:2012mp,
Ohme:2011zm,Hannam:2010ky}. 


NR has long sought to contribute template banks for gravitational-wave
searches. Due to the restrictions on the length and number of NR waveforms,
this has been conventionally pursued by calibrating intermediary
waveform models, and
using those for search templates. In this paper, we explore the alternative
of using NR waveforms and their hybrids directly in template banks.
We demonstrate the feasibility of this idea for non-spinning binaries,
and extending it to spinning binaries would be the subject of a future
work. We find that with only six non-spinning NR simulations, we can 
cover down to $m_{1,2}\gtrsim 12M_\odot$. We show that with
$26$ additional NR simulations, we can complete the non-spinning template
banks down to $M\simeq 12M_\odot$, below which existing PN waveforms 
have been shown to suffice for aLIGO. From template bank accuracy 
requirements, we are able to put a bound on the required length and 
initial GW frequencies for the new simulations. This method can therefore
be used to lay down the parameters for future simulations. 
%The ability to use hybrid waveforms
%within the software infracture of the LIGO-Virgo collaboration has been 
%demonstrated in the NINJA-2 collaboration~\cite{NINJA2:2013inPrep}. The
%template banks we present here can be directly used in aLIGO searches. 

% We propose the use of the NR waveforms themselves, or hybrids constructed
% out of them, in template banks for aLIGO GW searches. We demonstrate this 
% for the case of non-spinning BBHs, and aim to extend to spinning binaries 
% in future work. There are two benefits of using hybrid waveforms in 
% template banks. First, the intrinsic model errors in hybrid waveforms 
% arise out of the yet-unknown higher-order PN terms. These errors can be
% (conservatively) put a bound on~\cite{MacDonald:2011ne,MacDonald:2012mp}, 
% and we show that they can and should be taken into account when 
% constructing banks for stellar mass BBHs. Complete IMR waveform models
% are based on resummed or phenomenological extensions of PN calculations.
% Their free parameters are fitted to ensure the model agrees with NR 
% simulations that span a few tens of orbits before merger.
% However, their intrinsic errors in the late-inspiral regime, where the 
% PN theory begins to be inaccurate, are difficult to precisely quantify;
% especially for binary masses and spins to which the models are 
% interpolated or extrapolated to. Bounding of model errors in hybrids is
% useful for developing conservatively effectual banks. We expect this
% to be an even bigger advantage for template banks for spinning BBHs.
% Second, we are able to predict the simulations that would be required
% to complete the hybrid bank at lower masses, down to $M\sim 12M_\odot$.
% This is useful in choosing parameters for future NR simulations, which
% can be directly applied to GW searches.

First, we construct a bank for using pure-NR waveforms as templates, 
using a stochastic algorithm similar to Ref.~\cite{Harry:2009ea,
Ajith:2012mn,Manca:2009xw}. The filter templates are constrained to 
mass-ratios for which we have NR simulations available, i.e. 
$q=1,2,3,4,6,8$. 
We assume that the simulations available to us are $\geq 20$ orbits in
length. To test the bank, we simulate a population of $100,000$ BBH 
signals and filter them through the bank. The signals and templates 
are both modeled with the EOBNRv2 model~\cite{BuonannoEOBv2Main}. We 
demonstrate that this bank is indeed effectual and recovers 
$\geq 97\%$ of the optimal SNR for GWs from BBHs with mass-ratios 
$1\leq q\leq 10$ and chirp-mass 
$\mathcal{M}_c\equiv (m_1+m_2)^{-1/5}(m_1 m_2)^{3/5}$ above $27M_\odot$. 
Fig.~\ref{fig:bank001_01_match} shows this fraction at different 
simulated points over the binary mass space. With an additional 
simulation for $q=9.2$, we are able to extend the coverage to higher
mass-ratios. We show that a bank viable for NR waveform templates for
$q=1,2,3,4,6,9.2$, would recover $\geq 97\%$ of the optimal SNR
for BBHs with $10\leq q\leq 11$. The SNR recovery fraction from
such a bank is shown in Fig.~\ref{fig:bank006_01_match}.


Second, we construct effectual banks for currently available
NR-PN hybrid waveform templates. We derive a bound on waveform model
errors, which is independent of analytical models and can be used
to independently assess the errors of such models (see
Sec.~\ref{s1:quantifyingerrors} for details). This allows us to estimate
the hybrid waveform mismatches due to PN error, which are negligible
at high masses, and become significant at lower binary masses. We
take their contribution to the SNR loss into account while 
characterizing template banks. For hybrid banks, we demonstrate and 
compare two independent algorithms of template bank construction. 
First, we stochastically
place a bank grid, as for the purely-NR template bank. Second, we lay down
independent sub-banks for each mass-ratio, with a fixed overlap between
neighboring templates, and take their union as the final bank. 
To test these banks, we simulate a population of $100,000$ BBH signals
and filter them through each. We simulate the GW signals and the 
templates using the recently developed EOBNRv2 model~\cite{BuonannoEOBv2Main}. 
The fraction of the optimal SNR recovered by the two banks, before and after
accounting for the hybrid errors, are shown in the left and right panels 
of Fig.~\ref{fig:Current-hybrids-stochastic-FF} and 
Fig.~\ref{fig:Current-hybrids-FF} (respectively).
We observe that for BBHs with $m_{1,2}\geq 12M_\odot$ hybrid template 
banks will recover $\geq 96.5\%$ of the optimal SNR.  For testing the
robustness of our conclusions, we also test the banks using TaylorT4+NR
hybrid templates. The SNR recovery from a bank of these is shown in
Fig.~\ref{fig:Current-real-hybrids-FF}. We conclude that, 
the currently available NR+PN hybrid waveforms can be used as templates in 
a matched-filtering search for GWs from BBHs with $m_{1,2}\geq 12M_\odot$
and $1\leq q\leq 10$. The number of templates required to provide coverage
over this region was found to be comparable to a bank constructed using 
the second-order post-Newtonian TaylorF2 hexagonal template placement 
method~\cite{SathyaBankPlacementTauN,BabaketalBankPlacement,
SathyaMetric2PN,Cokelaer:2007kx}.
The two algorithms we demonstrate yield grids of $667$ and $627$ templates,
respectively; while the metric based placement method yields a grid of $522$ 
and $736$ templates, for $97\%$ and $98\%$ minimal match, respectively.


At lower mass, the length of the waveform in the sensitive frequency
band of the detectors increases, increasing the resolution of 
the matched-filter. We therefore see regions of undercoverage 
between mass ratios for which we have NR/hybrid templates 
(see, e.g. Fig.~\ref{fig:Current-hybrids-FF} at the left edge).
For $M\lesssim 12M_\odot$, existing PN waveforms were shown to 
be sufficient for aLIGO searches. 
We find the additional simulations that would be needed to extend
the hybrid tempalte bank down to $12M_\odot$. We show that a bank
of hybrids restricted to the $26$ mass-ratios listed in 
Table~\ref{table:fullqlist} would be sufficiently dense at $12M_\odot$.
This demonstrates that the method proposed here can be used 
to decide which NR simulations should be prioritized for the purpose
of the GW detection problem.
By filtering a population of $100,000$ BBH signals
through this bank, we show that the SNR loss due to its discreteness
stays below $2\%$ over the entire relevant range of masses.
The fraction of optimal SNR recovered is shown in 
Fig.~\ref{fig:templatebank_halfMassRatios}. Constraining the 
detection rate loss below $10\%$ requires that detection template 
banks recover more than $96.5\%$ of the optimal SNR. Therefore
our bank would need hybrids with hybridization mismatches below 
$1.5\%$. From this accuracy requirement, we obtain the length
requirement for all the $26$ simulations. This is depicted in the 
left panel of Fig.~\ref{fig:NRorbits2merger}, where we show the region
of the mass space that can be covered with hybrids, as the length of 
their NR portion varies. We find that for $1\leq q\leq 10$  the 
new simulations should be about $50$ orbits in length. In the right 
panel of Fig.~\ref{fig:NRorbits2merger} we show the corresponding
initial GW frequencies. The requirement of $\sim 50$ orbit long NR
simulations is ambitious, but certainly feasible with the current BBH 
simulation technology~\cite{BelaLongSimulation}.

In summary, we refer to the right panel of
Fig.~\ref{fig:templatebank_halfMassRatios}.
The region above the dashed (red) line and above the solid (blue) line can
be covered with a bank of purely-NR waveforms currently available. The region
above the dashed (red) and the dash-dotted (black) line can be covered with
the same simulations hybridized to long PN inspirals. With an additional set
of NR simulations summarized in Table~\ref{table:fullqlist}, the coverage
of the bank can be extended down to the magenta (solid) line in the lower
left corner of the figure. Thus between hybrids and PN waveforms, 
we can cover the entire non-spinning BBH space. The ability to use hybrid
waveforms within the software infrastructure of the LIGO-Virgo collaboration 
has been demonstrated in the NINJA-2 collaboration~\cite{NINJA2:2013inPrep}.
The template banks we present here can be directly used in aLIGO searches. 
This work will be most useful when extended to aligned spin and 
precessing binaries~\cite{Boyle:2013nka,Schmidt:2012rh}, which is the
subject of a future work.

The detector noise power is modeled using the zero-detuning high-power 
noise curve for Advanced LIGO~\cite{aLIGONoiseCurve}.
The construction of our template banks is sensitive to the breadth
of the frequency range that the detector would be 
sensitive to. The noise curve we use is the broad-band final design 
sensitivity estimate. For lower sensitivities
at the low/high frequencies, our results would become more conservative, 
i.e. the template banks would over-cover (and not under-cover).

We finally note that in this paper we have only considered the
dominant $(2,2)$ mode of the spherical decomposition of the 
gravitational waveform. For high mass-ratios and high binary masses,
other modes would also become important, both for spinning as well
as non-spinning black hole binaries~\cite{Pekowsky:2012sr,
Brown:2012nn,Capano:2013inPrep}.
Thus, in future work, it would be relevant to examine the
sub-dominant modes of the gravitational waves. Lastly, though we have
looked at the feasibility of using this template bank for Advanced
LIGO as a single detector, this instrument will be part of a network
of detectors, which comes with increased sensitivity and sky
localization. For this reason, in subsequent studies it would be
useful to consider a network of detectors.
