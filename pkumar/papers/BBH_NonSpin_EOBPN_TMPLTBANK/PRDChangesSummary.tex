\documentclass[a4paper,10pt]{article}
%\documentclass[a4paper,10pt]{scrartcl}

\usepackage[utf8]{inputenc}
\usepackage{amssymb}

\title{DY11131/Brown -- Summary of changes to the manuscript}
\author{Duncan A. Brown, Prayush Kumar, and Alexander H. Nitz}
\date{\today}

\pdfinfo{%
  /Title    (DY11131/Brown -- Summary of changes to the manuscript)
  /Author   (Duncan A. Brown, Prayush Kumar, and Alexander H. Nitz)
  /Creator  ()
  /Producer ()
  /Subject  ()
  /Keywords ()
}

\begin{document}
\maketitle
We summarize the changes that have been made to the manuscript, in response
to the referee's comments and suggestions. We would like to the thank the 
referee for carefully reading through the manuscript and their constructive
suggestions.
\newline

\textbf{Major Changes:}
\newline

1. We have added the analysis for the loss in detection volume/observation
rate, as is implied by our fitting factor results. The fitting-factors give
an estimate of the fractional loss in the SNR, for binaries in different 
configurations. The loss is observed to be higher for binaries that are 
oriented sub-optimally to the detector, and hence their relatively larger
fractional loss in SNR contributes proportionately lesser to the over-all
observable volume available to the detector. We estimate the loss in the 
volume observable by the detector (at a fixed SNR threshold) using volume 
weighted fitting-factors, and this has been described in detail at the beginning
of Sec.~III and results summarized in Sec.~IIIC. For a population of binaries 
distributed uniformly in spacial volume, this corresponds to the loss in 
detection rate for Advanced LIGO. Also, Fig.~(7) has been
added with this quantity plotted on the component-mass plane and the mass-ratio-
inclination-angle plane, while marginalizing over the other parameters 
respectively.
\newline

2. We have added Fig.~(6) showing the loss in fitting-factors on the mass-ratio-
inclination-angle plane, and the inclination-angle-total-mass plane. These
figures show the orientations of the binary which suffer from the highest
SNR loss because of ignoring the higher-order modes in search templates.
The description has also been added in Sec.~IIIC.
\newline

3. We have added Fig.~(4) showing the minimal-fitting-factor (minimal-match)
of the TaylorF2 bank, in the region restricted to systems with total masses
below different threshold values. We have also simulated 100,000 more points 
to estimate the effectualness of the TaylorF2 bank more precisely, where we had 
25,000 points earlier. We observe that the region where the TaylorF2 bank is 
effectual is the one bounded by the total mass threshold of $\sim 11.4$ solar 
masses. This threshold supercedes the one stated in our earlier manuscript.
The description has been added to Sec.~IIIB.
\newline

4. Similar to (3.) above, we have added Fig.~(5) (right panel), showing the minimal
match of a MM$=0.97$ bank with $l=m=2$ EOB multipoles as templates, in regions
restricted to having mass-ratios below different thresholds. We find that this
bank is effectual for systems with mass-ratios $\lesssim 1.68\,(4)$, corresponding
to the acceptable loss in detection rate (at a fixed SNR threshold) being
$10\%\,(15\%)$.
The description has been added to Sec.~IIIC.
\newline

5. The Abstract, Introduction and the Conclusions sections have been updated
with the results described in (1.) - (4.) above.
\newline

\textbf{Minor Changes:}
\newline

1. We have changed Fig.~(3) and Fig.~(5) (left panel) from scatter plots to 
contour plots, showing the drop in fitting-factors in a more contrasted 
color scheme.  This has been done in hope that these plots are better readable.
\newline

2. We have added a description of the calculation of the initial conditions
to generate an EOBNRv2 waveform, at the end of Sec.~IIA.
\newline

3. We have added a discussion of the bank placement metric, which we use to 
construct different template bank grids, as Sec.~IIB.
\newline

4. Some of the description of mathematical quantities has been moved from the
beginning of Sec.~III to Sec.~IIB, where we need those to discuss the bank 
placement metric.
\newline

5. We have added a description of volume-weighted fitting-factors at the beginning
of Sec.~III as well. We use this quantity to quantify the loss in event 
observation rate for the detector, as a function of a few parameters of the binary
system, while marginalizing over the others.
\newline

6. We have acknowledged the anonymous referee for their positive feedback.
\end{document}
