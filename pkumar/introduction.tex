\newcommand{\FILL}{\textcolor{red}{FILLME}}

%%%%%%%%%%%%%%%%%%%%%%%%%%%%%%%%%%%%%%%%%%%%%%%%%%%%%%%%%%%%%%%%%%%%%%%%%%%%%%%
%%% Introduction (Combine other papers' introductions..)
%%%%%%%%%%%%%%%%%%%%%%%%%%%%%%%%%%%%%%%%%%%%%%%%%%%%%%%%%%%%%%%%%%%%%%%%%%%%%%%

% % What are gravitational waves, in theory?

Gravitational waves
% , as predicted by General Relativity, 
are ripples in the curvature of spacetime that propagate at the speed of light,
% . Like water waves on the ocean, the concept of gravitational wave requires an
% ideal, i.e. a smooth and unperturbed background on which the waves 
% propagate. Unlike water waves, however, gravitational waves are not the motion 
% of a material medium, but are ripples in the fabric of spacetime itself.
% The phonemena can be intuitively described with an analogy to electromagnetism. 
% When an electromagnetic wave is incident upon a charged particle, it 
% accelerates the particle in a direction transverse to its own direction
% of propagation. This acceleration depends on the electric charge 
% and the inertial mass of the particle.
% Incident gravitational waves, similarly, produce an acceleration that is 
% transverse to their direction of propagation; with the difference that the 
% acceleration produced is independent of the mass of the particle experiencing
% it and will be the same for any particle at that point. 
% Free falling particles also exerience the same acceleration, and an observer
% sitting in their frame, also known as a local inertial frame,
% will not be able to measure the local acceleration. 
% It is only through non-local measurements between points that are separated 
% in spacetime that one would be able to discern the effect of a passing 
% gravitational wave.
% 
% % How do we know they exist? Why are they difficult to detect? 
% 
% Gravitational waves carry 
carrying energy away from their source. These waves couple weakly with matter as 
they propagate through the universe, 
preserving information that is otherwise inaccessible with electromagnetic 
observations. The detection of gravitational waves will enable 
us to do precision observation and characterization of astrophysical sources. 
On the other hand, the relatively weak coupling with matter makes gravitational
waves difficult to detect directly in a laboratory.
The first indirect detection of gravitational waves was made by Russell Hulse 
and Joseph Taylor, when they discovered a binary system with a pulsating neutron
star PSR 1913+16 in 1974~\cite{hulse}. They observed pulses of radio waves
emitted by the pulsar, and through precise measurement of their timing,
they were able to determine the rate at which the binary orbit was shrinking.
The orbital energy depends on the separation of the objects in 
the binary, and so Hulse and Taylor's
measurement allowed them to calculate the rate at which the binary 
was losing energy. It has been shown that this rate agreed to within a percent
with that predicted by General Relativity~\cite{Weisberg:1981mt,Taylor:1989}.
In the following years, more binary pulsars
have been discovered, allowing for more precise measurements of the effect
of gravitational waves as predicted by General Relativity~\cite{Burgay:2003jj}.


% % What is LIGO trying to do?

Felix Pirani first suggested the use of an interferometer for precise
measurement of the distance between two test masses in 1956~\cite{Pirani:1956}.
In 1971, Moss, Miller and Forward built and tested the first laser
interferometric gravitational wave transducer~\cite{Forward:1971}. The design
of the modern interferometric detectors is based on the contributions 
of Weiss~\cite{Weiss:1972} and Drever~\cite{Drever:1980} from the 1970s. The
core of such detectors is a Michelson interferometer, with two laser 
cavities oriented perpendicular to each other. The end mirrors of each 
cavity are held vertically, and are practically freely falling in the 
transverse direction. Incoming gravitational waves will cause different 
changes in the length of each cavity, and these instruments measure with 
great precision the change in the difference in the length of both 
cavities~\cite{Saulson:1995zi}. 
%
There is currently a concerted effort worldwide towards the direct detection 
of gravitational waves, and several ground based gravitational wave observatories 
are being planned and built. These include:
%
\begin{enumerate}
 \item Laser Interferometer Gravitational-wave Observatory (LIGO): This
 United States funded project is comprised of two independent detectors, one 
 located in Hanford, Washington and the other in Livingston, Louisiana. 
 \item Virgo observatory: A French-Italian project based in Cascina, Italy.
 \item GEO 600, near Sarstedt, Germany.
 \item The Kamioka Gravitational Wave Detector (KAGRA), formerly the Large-scale
 Cryogenic Gravitational-wave Telescope (LCGT), is a Japanese project, and has
 begun construction in the tunnels of Kamioka mines in Japan.  
 \item LIGO-India: This is a tentative project between LIGO and several 
 institutions in India, aimed at having a third LIGO-like instrument in the 
 eastern hemisphere.
\end{enumerate}
% 
% % 
% % % The coupling of the gravitational force to matter is significantly weaker than
% % % the other fundamental fources. The direct observation of the fluctuation in 
% % % the proper distances, or the spacetime metric, would require a massive system
% % % with a changing kinematic arrangement. The magnitude of these observable
% % % fluctuations depends directly on the amount of mass and its kinematic
% % % configuration in a system, and inversely on the physical distance between the 
% % % detectors and the system. The problem arises that the limits 
% % % of the current length-change measurement technology requires that the 
% % % system be so massive as one could imagine only of stellar-scale compact
% % % objects. This brings a factor in the picture which humans cannot affect, 
% % % requiring us to adjust our technology to be able to cope with the 
% % % distances to the nearest compact binary systems. While we have seen binary
% % % star systems with one or both objects as neutron stars, and have observed
% % % their evolution to follow the path predicted by General Relativity, these 
% % % systems have mostly be seen either (i) during their early stages of 
% % % coalescence, with billions of years to their mergers, or (ii) at distances
% % % beyond the horizon of the initial LIGO-Virgo detectors. 
% % % The only way is to push the threshold of length-measurement technology 
% % % beyond its current frontier, by about an order of magnitude in the Hz-kHz 
% % % frequency band. This requires considerable effort and is the limit being
% % % pushed to move towards the era of gravitational-wave astronomy.
% % % With Advanced LIGO, we expect to extend the horizon of strain-based 
% % % gravitational-wave detectors to the regime where a number of binaries are
% % % expected to merge within it. 
% % 
% % % % Describe LIGO..?
% % 
The first-generation LIGO and Virgo detectors were constructed in stages, and
began observation in 2002 and 2007, respectively. There were 6 observation 
runs performed with LIGO, called Science run 1 (S1) to Science run 6 (S6). 
During the fifth Science run (S5) in 2007, the LIGO detectors (Hanford and
Livingston) reached their design sensitivity, being sensitive to gravitational 
waves from systems similar to the Hulse-Taylor binary pulsar out to a
(sky-averaged) distance of $\sim 15$~Mpc.
For comparison, the Virgo supercluster, of which our own galaxy is a part, has
a diameter of $33$~Mpc. In the sixth Science run, the sensitivity of LIGO 
was improved by increasing the laser power in its cavities, adding 
an output-mode cleaner and implementing other technological upgrades. 
After these upgrades, the LIGO detectors operated from July 2009
to October 2010, being sensitive to fiducial binary pulsar systems out to 
$\sim 20$~Mpc. 
Similarly, the Virgo detector observed in three disjoint periods called Virgo 
Science Runs (VSR1-3), extending over late 2007 to 2011.
% At the peak of their combined sensitivity, the three detectors were sensitive 
% to binary pulsars out to a sky-averaged distance of $\sim $\FILLME Mpc. 
While no detection was made, the LIGO-Virgo Scientific Collaboration
(LVC) was able to put upper limits on the rate of compact object binary mergers
within its range. Results from these observation periods have been published 
since, e.g.~\cite{Messaritaki:2005wv,Abadie:2010mt,Abadie:2012rq,Abbott:2009km,
Colaboration:2011nz,Abadie:2010yb,Abbott:2009qj,Abbott:2009tt,Abadie:2011kd,
Aasi:2012rja,Abbott:2003yq,Abbott:2005pu,Sintes:2005fp,Abadie:2011md,
Palomba:2012wn}. 

% % Advanced LIGO
The second generation Advanced LIGO (aLIGO) and Virgo detectors are currently
under construction and commissioning~\cite{Harry:2010zz,aVIRGO}.
The two aLIGO detectors are expected to begin operation in $2015$, and
plan to achieve their design sensitivity by 2019~\cite{Aasi:2013wya}. 
The Virgo detector is also being upgraded on a similar timeline. 
These upgrades will increase the sensitivity of both detectors in two 
important ways, (i) provide a factor of 10 improvement in sensitivity 
across the entire LIGO frequency band, and (ii) lower the lower frequency
bound from $40$~Hz down to about $15$~Hz~\cite{Harry:2010zz}. 
These enhancements would mean a thousandfold increase in the volume of the
universe that we would be able to probe for sources of gravitational waves. 
Astrophysical estimates and numerical simulations suggest that the Advanced
detectors would detect about 40 mergers a year of Hulse-Taylor type compact
binaries~\cite{LSCCBCRates2010}. 

% % What are compact binaries, and how are they formed?

Gravitational-wave astronomy can probe regions that the electromagnetic
spectrum cannot, like black holes, interiors of neutron stars, and 
supernovae core collapse mechanism.
They also offer the unique avenue of probing gravity in 
the strong-field regime, providing a valuable test of General Relativity.
There are several astrophysical sources of gravitational waves, such as 
supernovae explosions, cosmic strings, black hole ringdowns, etc. 
Binaries of stellar-mass compact objects are of special interest for 
ground based detectors like LIGO and Virgo as they emit gravitational 
radiation in the operating frequency range of these detectors, which
extends from about 15~Hz to a few kHz. As binary systems of 
stars evolve and undergo supernovae, a significant 
fraction of them get disrupted. The systems that survive form binaries
containing neutron stars and/or black holes. While the mass of neutron
stars is observationally constrained between 
$1-3\msun$~\cite{2013ApJ...778...66K,Freire:2011}, the upper bound on the
mass distribution of stellar-collapse black holes is less well known. 

Searches for gravitational waves from compact binaries operate by 
matched-filtering the detector data to dig out the weak gravitational wave
signatures that are otherwise buried in the instrumental noise. 
Matched-filter searches take
advantage of the fact that we can theoretically model the expected form of the 
gravitational wave signatures, called \textit{waveforms}, and use those 
as filtering templates. While an exact analytic solution in General Relativity
for the dynamics of compact binary systems has not been found, several perturbative
formalisms exist that are capable of describing the inspiral dynamics.
In the slow-motion large-separation post-Newtonian (PN) 
approximation~\cite{PNtheoryLivingReviewBlanchet}, orbital quantities have 
been calculated as Taylor series in the binary 
velocity parameter $v/c$. PN theory provides for faithful modeling of the 
early inspiral for comparable mass binaries when $v/c\ll 1$. 
Careful resummation techniques, e.g. within the Effective-One-Body
(EOB) framework~\cite{EOBOriginalBuonannoDamour}, can extend PN theory 
results to the strong-field fast-motion regime.
On the other hand, for binaries where one body is substantially more massive
than the other, the recently developed self-force (SF) formalism provides for
accurate waveform models~\cite{grallaI,grallaII}. 
% 
There has also been tremendous progress in Numerical Relativity, which 
includes high-accuracy numerical simulations of binary black
hole mergers in non-perturbative General Relativity. The first breakthrough
simulations were performed in 2005 by Frans Pretorius~\cite{Pretorius2005,
Pretorius2006}, the Goddard group~\cite{Campanelli:2005dd} and the research
group at the University of Texas -- Brownville and Florida Atlantic 
University~\cite{Campanelli:2005dd}. More recently, the Simulating eXtreme
Spacetimes (SXS) collaboration~\cite{SXSWebsite} between California Institute
of Technology, Cornell University, Canadian Institute of Theoretical 
Astrophysics, and the California State University at Fullerton has made rapid
progress in increasing the stability and efficiency of the numerical 
methods involved in simulating mergers of black holes~\cite{Mroue:2013xna}.
Owing to the computational complexity of these simulations, they are available 
only for a restricted set of binary configurations. 
The information from these simulations can be used to calibrate
semi-analytic waveform models~\cite{BuonannoEOBv2Main}, as well as purely
phenomenological closed-form models~\cite{Santamaria:2010yb}. 
There has also been concerted progress in including the effect of the 
internal matter structure of the neutron stars in binaries, 
e.g.~\cite{Deaton:2013sla}, and in simulating other matter systems,
e.g. supernovae~\cite{Mosta:2014jaa}.
In this dissertation, we will restrict ourselves to systems in vacuum, 
neglecting possible matter effects. 

With its increasead sensitivity, the Advanced detectors will need improved
search techniques and more accurate waveform templates, to efficiently
filter out true gravitational wave signals from instrument noise. 
The goal of the research presented in this dissertation is to devise, and to
improve the techniques of using waveform templates in gravitational wave searches. 
The rest of the dissertation is organized as follows:
\begin{enumerate}
 \item In chapter~\ref{ch:theory}, we describe the production of gravitational
 waves from compact binaries. %, and its effect on the interferometric detectors.
 \item In chater~\ref{ch:ligo}, we discuss the construction of terrestrial 
 interferometers like LIGO, and their response to incident gravitational waves.
 \item In chapter~\ref{ch:EOBF2_Effectualness} 
 we study the importance of the accuracy of the PN theory in the predictive 
 modeling of the late-inspiral of binaries. We estimate the range of binary 
 parameters where the weak-field slow-motion PN theory is accurate and 
 sufficient for aLIGO filter templates.
 \item In chapter~\ref{ch:NRHyb_bank}, we study the possibility of using a
 restricted set of accurate Numerical Relativity simulations as filter template
 in aLIGO searches. 
 \item In chapter~\ref{ch:ninja2}, we describe work done as part of the 
 NINJA-2 project. This involved using accurate NR waveforms as simulated
 signals injected in LIGO noise that is recolored to aLIGO sensitivity, in
 order to test the search algorithms that are being developed by the
 LIGO-Virgo collaboration.
%  \item In chapter~\ref{ch:insSFIMRI}, we describe a waveform model that we 
%  developed using recent calculations in the self-force picture. This model 
%  suffices to describe the inspiral dynamics of compact binaries with 
%  $m_1/m_2\gtrsim 6$.
 \item In chapter~\ref{ch:imrSFIMRI}, we describe a waveform model that 
 captures the inspiral, plunge and merger phases of compact binary 
 coalescence, for non-spinning intermediate mass-ratio binaries.
 \item Finally, chapter~\ref{ch:conclusions} is a summary of the conclusions
 from the research presented in this work.
\end{enumerate}




