\newcommand{\xhat}{\vec{e}_x}
\newcommand{\yhat}{\vec{e}_y}
\newcommand{\zhat}{\vec{e}_z}
\newcommand{\ihat}{\vec{e}_i}
\newcommand{\jhat}{\vec{e}_j}
\newcommand{\rhat}{\vec{e}_{r}}
\newcommand{\iotahat}{\vec{e}_{\iota}}
\newcommand{\phihat}{\vec{e}_{\phi}}
\newcommand{\eplus}{\tens{e}_+}
\newcommand{\ecross}{\tens{e}_\times}
\newcommand{\Sl}{S_\ell}
\newcommand{\Sigmal}{\Sigma_\ell}
\newcommand{\Flux}{\mathcal{F}}

\section{Post-Newtonian waveforms in the adiabatic approximation}
\renewcommand{\theequation}{A.\arabic{equation}}

There are two basic elements to obtaining a post-Newtonian waveform:
(1) finding the orbital phase of the binary, and (2) using that phase
to find the so-called ``amplitude'' of the waveform.  Previous
references have been incomplete, or predate recent errata involving
spin terms \cite{Blanchet:2006gy,Arun:2009}.  Here, we gather together the most complete and current
formulas for phase and amplitude when spins are \emph{non-precessing}
 (i.e. spins aligned or anti-aligned with the
  orbital angular momentum), using consistent
notation.

\subsection{Phasing}
The orbital phase evolution $\Phi(t)$ of the binary can be computed by
considering energy conservation.  The energy of the full system is
accounted for in three parts, each computed as a function of the
post-Newtonian expansion parameter
\begin{equation}
  \label{eq:DefinevParameter}
  v \defeq \left( M \frac{d\Phi}{dt} \right)^{1/3}~.
\end{equation}
The first part is the kinetic and gravitational binding energy of the
binary---the orbital energy $E(v)$.  As the system evolves, it gives
off energy to infinity in the form of gravitational waves accounted
for as the flux $\Flux(v)$ leaving the system.  Finally, we must also
account for the tide raised on each black hole by the other, and the
flow of energy into the black holes due to the motion of these tides,
given as a rate of change in the mass of the black holes $\dot{M}(v)$.
Using this threefold accounting for the energy, we can express the
conservation of energy as
\begin{equation}
  \label{eq:EnergyConservation}
  \frac{dE}{dt} + \Flux + \dot{M} = 0~.
\end{equation}
Now, because the expression for the orbital energy is written in terms
of $v$, we can straightforwardly differentiate to find $E'(v)$.  With
the chain rule, $dE/dt = E'(v)\, dv/dt$, we can rearrange this into a
differential equation for $v$:
\begin{equation}
  \label{eq:BalanceEquation}
  \frac{dv}{dt} = - \frac{\Flux(v) + \dot{M}(v)} {E'(v)}~.
\end{equation}
Given the expressions for $\Flux(v)$, $\dot{M}(v)$, and $E'(v)$, this
equation can be integrated to find $v(t)$.  Then, using the definition
of $v$, we see that
\begin{equation}
  \label{eq:PhaseFormula}
  \frac{d\Phi}{dt} = \frac{v^{3}}{M}~,
\end{equation}
which can be integrated in turn to find $\Phi(t)$.  We now exhibit the
formulas for $\Flux(v)$, $\dot{M}(v)$, and $E'(v)$, and discuss
various methods for integrating the balance
Eqn.~\eqref{eq:BalanceEquation}.

Given the masses $M_1$ and $M_2$ and spin vectors $\mathbf{S}_1$ and
$\mathbf{S}_2$, we define the following parameters:
\begin{align}
  \label{eq:SSigma}
  M &\defeq M_1 + M_2~, \\
  \eta  &\defeq M_1 \, M_2 / M^2~, \\
  \delta  &\defeq (M_1 - M_2) / M~,\\
  % \mathbf{S} &\defeq \mathbf{S}_1 + \mathbf{S}_2~,\\
  % \bm{\Sigma} &\defeq M \left[\frac{\mathbf{S}_2}{M_2} -
  %   \frac{\mathbf{S}_1}{M_1}\right]~,\\
  \bm{\chi}_i &\defeq \mathbf{S}_i/M_i^2~, \\
  \bm{\chi}_s &\defeq (\bm{\chi}_1 + \bm{\chi}_2)/2~, \\
  \bm{\chi}_a &\defeq (\bm{\chi}_1 - \bm{\chi}_2)/2~.
\end{align}
We also define the quantities $\chi_s$ and $\chi_a$ to be the
components of the spin vectors perpendicular to the orbital plane,
namely $\chi_s\defeq \bm{\chi}_s \cdot \bm{\ell}$ and $\chi_a \defeq
\bm{\chi}_a \cdot \bm{\ell}$, where $\bm{\ell}$ is the unit vector
along the Newtonian angular momentum.

The orbital energy function can be written in terms of the PN
expansion parameter $v$ defined above as~\cite{Blanchet:2002,
  Blanchet:2005a, Blanchet:2005b, Arun:2009, Blanchet:2006gy,
  Blanchet:2007, Blanchet:2010}\footnote{The 1.5PN and 2.5PN spin
  terms were taken from Eq.~(7.9) of \cite{Blanchet:2006gy}. The 2PN
  spin term and all nonspinning terms were taken from Eq.~(C4) of
  \cite{Arun:2009}.  Note that Eq.~(C5) in the original published
  version of \cite{Arun:2009} is erroneous.}
\begin{equation}\label{eq:Energy}
  \begin{split}
    E(v) & = -\frac{M \eta v^2}{2}\, \left\{1 + v^2\left( -
\frac{3}{4}
        - \frac{\eta}{12} \right) + v^{3} \left[\frac{8 \, \delta
          \chi_a}{3}+\left(\frac{8}{3}
          -\frac{4 \eta }{3}\right) \chi_s \right] \right. \\
    &\qquad\qquad\quad + v^{4} \left[-2 \delta \chi_a
      \chi_s-\frac{\eta ^2}{24}
      +(4 \eta -1) \chi_a^2+\frac{19 \eta }{8}-\chi_s^2-\frac{27}{8}\right] \\
    &\qquad\qquad\quad + v^{5} \left[\chi_a \left(8 \, \delta
        -\frac{31 \delta \eta }{9}\right)
      +\left(\frac{2 \eta^2}{9}-\frac{121 \eta }{9}+8\right) \chi_s \right] \\
    &\qquad\qquad\quad \left. + v^{6} \left[-\frac{35 \eta
          ^3}{5184}-\frac{155 \eta ^2}{96}
        +\left(\frac{34445}{576}-\frac{205 \pi^2}{96}\right) \eta
        -\frac{675}{64} \right] \right\}.
  \end{split}
\end{equation}
We simply take the derivative of this formula with respect to $v$ to
find the energy function appearing in the phasing formula:
\begin{equation}\label{eq:dEnergy}
  \begin{split}
    E'(v) & = -M \eta v\, \left\{ 1 + v^2\left( - \frac{3}{2} -
        \frac{\eta}{6} \right) + v^{3} \left[ \frac{20 \delta
          \chi_a}{3}+\left(\frac{20}{3}
          -\frac{10 \eta }{3}\right) \chi_s \right] \right. \\
    & \hphantom{\quad -\eta v\, \Bigg\{} + v^{4} \left[ -6 \, \delta
      \chi_a \chi_s-\frac{\eta ^2}{8}
      +(12 \eta -3) \chi_a^2+\frac{57 \eta }{8}-3 \chi_s^2-\frac{81}{8} \right]\\
    & \hphantom{\quad -\eta v\, \Bigg\{} + v^{5} \left[ \chi_a
      \left(28 \delta -\frac{217 \delta \eta }{18}\right)
      +\left(\frac{7 \eta^2}{9}-\frac{847 \eta }{18}+28\right) \chi_s \right] \\
    & \hphantom{\quad -\eta v\, \Bigg\{} \left. + v^{6} \left[
        -\frac{35 \eta ^3}{1296}-\frac{155 \eta ^2}{24}
        +\left(\frac{34445}{144}-\frac{205 \pi^2}{24}\right) \eta
        -\frac{675}{16} \right] \right\}~.
  \end{split}
\end{equation}
Similarly, the flux function can be written as~\cite{Blanchet:2002,
  Blanchet:2005a, Blanchet:2005b, Arun:2009, Blanchet:2006gy,
  Blanchet:2007, Blanchet:2010}\footnote{The 1.5PN and 2.5PN spin
  terms were taken from Eq.~(7.11) of \cite{Blanchet:2006gy}. The 2PN
  spin term and all nonspinning terms were taken from Eq.~(C10) of
  \cite{Arun:2009}, except that the term $\eta
  \left\{-\frac{103}{48}(\bm{\chi}_s^2 - \bm{\chi}_a^2) +
    \frac{289}{48} [(\bm{\chi}_{s} \cdot \bm{\ell})^{2} -
    (\bm{\chi}_{a} \cdot \bm{\ell})^{2}] \right\}$ is omitted.  (The
  authors of \cite{Arun:2009} have confirmed that this term should not
  be present.)  Also note that Eq.~(C11) in the original published
  version of \cite{Arun:2009} is erroneous.}
{\footnotesize
\begin{equation}\label{fluxx}
  \begin{split}
    \Flux(v) & = \frac{32}{5}\, v^{10}\, \eta^2 \left\{ 1 + v^2 \left(
        - \frac{1247}{336} - \frac{35}{12} \eta \right) + v^3
      \left[-\frac{11 \delta \chi_a}{4}+\left(3 \eta
          -\frac{11}{4}\right)
        \chi_s+4 \pi \right] \right.\\
    %
    &\qquad\qquad\qquad + v^4 \left[\frac{33 \delta \chi_a
        \chi_s}{8}+\frac{65 \eta^2}{18} +\left(\frac{33}{16}-8 \eta
      \right) \chi_a^2+\left(\frac{33}{16} -\frac{\eta }{4}\right)
      \chi_s^2
      +\frac{9271 \eta}{504}-\frac{44711}{9072} \right] \\
    %
    &\qquad\qquad\qquad + v^5 \left[\left(\frac{701 \delta \eta }{36}
        -\frac{59 \delta }{16}\right) \chi_a +\left(-\frac{157 \eta
          ^2}{9} +\frac{227 \eta }{9}-\frac{59}{16}\right) \chi_s
      -\frac{583 \pi  \eta }{24}-\frac{8191 \pi }{672} \right] \\
    %
    &\qquad\qquad\qquad + v^6 \left[-\frac{1712}{105} \ln (4
      v)-\frac{1712 \gamma }{105} -\frac{775 \eta ^3}{324}-\frac{94403
        \eta ^2}{3024}+\left(\frac{41 \pi
          ^2}{48}-\frac{134543}{7776}\right) \eta +\frac{16 \pi^2}{3}
      +\frac{6643739519}{69854400} \right] \\
    %
    &\qquad\qquad\qquad \left.  + v^7 \left[\frac{193385 \pi \eta
          ^2}{3024} +\frac{214745 \pi \eta }{1728}-\frac{16285 \pi
        }{504} \right] \right\}~,
  \end{split}
\end{equation}
}
where $\gamma$ is the Euler Gamma.

Alvi~\cite{Alvi:2001} derived an expression for the transfer of energy
from the orbit to each black hole by means of tidal heating.  The
calculation involves computing the deformation of each hole's horizon
due to the Newtonian field of the other, then using that expression in
formulas for energy absorption due to tidal deformation.  In
particular, his expression is applicable in the comparable-mass case.
By combining the rates of mass change for both black holes, we obtain
the total rate of change:
\begin{equation}
  \label{eq:MassRateOfChange}
  \begin{split}
    % \dot{M}(v) &= \frac{32}{5} v^{10} \eta^{2} \left\{
    %   -\frac{v^{5}}{4}\, \left[ \left( \frac{M_{1}}{M} \right)^{3}
    %     (1+3\chi_1^2) \frac{\mathbf{S}_{1} \cdot \bm{\ell}}{M^{2}} +
    %     \left( \frac{M_{2}}{M} \right)^{3} (1+3\chi_2^2)
    %     \frac{\mathbf{S}_{2} \cdot \bm{\ell}}{M^{2}} \right]
    % \right\}
    % \\
    % &= \frac{32}{5} v^{10} \eta^{2} \left\{ -\frac{v^{5}}{4}\,
    %   \left[ \frac{(1 - 2\eta) \Sl - \eta \delta \Sigmal}{M^{2}} +
    %     \frac{3(2 \Sl + \delta \Sigmal) \big( \Sl^{2} + \delta \Sl
    %       \Sigmal + (1-\eta)\Sigmal^{2} \big)}{M^{6}} \right]
    % \right\}
    % \\
    \dot{M}(v) &= \frac{32}{5} v^{10} \eta^{2} \left\{
      -\frac{v^{5}}{4}\, \Big[ (1 - 3\eta) \chi_{s} (1 + 3\chi_{s}^{2}
      + 9\chi_{a}^{2}) + (1 - \eta) \delta \chi_{a} (1 + 3\chi_{a}^{2}
      + 9\chi_{s}^{2}) \Big] \right\}~.
  \end{split}
\end{equation}
The coefficient above is the leading-order term in the flux, meaning
that this term is comparable to a relative 2.5PN spin effect in the
flux.  A similar calculation has been carried out in the
extreme-mass-ratio limit~\cite{Tagoshi:1997}, and agrees with this
formula in that limit.  Note that higher-order spin terms were
calculated in~\cite{Alvi:2001}, but are not included here, as they are
at relative 3.5PN order, which is higher than the relative 2.5PN order
to which other spin terms are known.  Except for its explicit presence
in the balance equation~\eqref{eq:BalanceEquation}, we always treat
the mass as a constant.  This leads to additional errors at the 3.5PN
spin level, which we ignore.


Below we will define some of the standard variants of computing
the post-Newtonian phase from the energy and flux functions,
using the naming convention of \cite{Damour:2000zb}.

\subsubsection{TaylorT1 phasing}
The TaylorT1 approximant is computed by numerically integrating the
ordinary differential equation for $v(t)$ in
Eq.~\eqref{eq:BalanceEquation}, using the expressions for orbital
energy, flux, and mass change given in Eqs.~\eqref{eq:dEnergy},
\eqref{fluxx}, and~\eqref{eq:MassRateOfChange}.  The phase is then
computed using this result for $v(t)$ in Eq.~\eqref{eq:PhaseFormula}.

\subsubsection{TaylorT4 phasing}
The TaylorT4 approximant is similar to the TaylorT1 approximant,
except that the ratio of the polynomials on the right-hand side of
Eq.~\eqref{eq:BalanceEquation} is first expanded as a Taylor series,
and truncated at consistent PN order.  Explicitly, the formula to be
integrated is
{\footnotesize
\begin{equation}\label{eq:dvByDt}
  \begin{split}
    \frac{dv}{dt} & = \frac{32}{5M} v^9 \eta \left\{1 + v^2
      \left[-\frac{11 \eta }{4}-\frac{743}{336} \right] + v^3
      \left[-\frac{113 \delta \chi_a}{12}+\left(\frac{19 \eta }{3}
          -\frac{113}{12}\right) \chi_s+4 \pi \right] \right. \\
    %
    & \qquad\qquad\quad + v^4 \left[\frac{81 \delta \chi_a
        \chi_s}{8}+\frac{59 \eta ^2}{18} +\left(\frac{81}{16}-20 \eta
      \right) \chi_a^2+\left(\frac{81}{16}-\frac{\eta }{4}\right)
      \chi_s^2+\frac{13661 \eta
      }{2016}+\frac{34103}{18144} \right] \\
    %
    & \qquad\qquad\quad + v^5 \left[-\frac{189 \pi \eta
      }{8}-\frac{4159 \pi }{672} + \left(\frac{3 \eta
        }{4}-\frac{3}{4}\right) \delta \chi_a^3 + \left(\frac{9 \eta
        }{4}-\frac{9}{4}\right) \delta \chi_a \chi_s^2
      + \left(\frac{1165 \eta }{24}-\frac{31571}{1008}\right) \delta \chi_a \right. \\
    & \qquad\qquad\qquad \qquad \left. + \left(-\frac{79 \eta ^2}{3}+
        \frac{27 \eta \chi_a^2}{4} -\frac{9 \chi_a^2}{4}+\frac{5791
          \eta }{63}-\frac{31571}{1008}\right) \chi_s
      +\left(\frac{9 \eta }{4}-\frac{3}{4}\right) \chi_s^3 \right] \\
    %
    & \qquad\qquad\quad + v^6 \left[-\frac{1712 \gamma }{105}
      -\frac{5605 \eta ^3}{2592}+\frac{541 \eta
        ^2}{896}+\left(\frac{451 \pi ^2}{48}
        -\frac{56198689}{217728}\right) \eta +\frac{16 \pi ^2}{3}
      +\frac{16447322263}{139708800} \right. \\
    & \qquad\qquad\qquad \qquad -\frac{1712 \ln (4v)}{105} +
    \left(\frac{1517 \eta ^2}{72}-\frac{23441 \eta }{288}
      +\frac{128495}{2016} \right) \chi_s^2 + \left(\frac{565 \delta
        ^2}{9}+\frac{89 \eta ^2}{3}-\frac{2435 \eta }{224}
      +\frac{215}{224} \right) \chi_a^2 \\
    & \qquad\qquad\qquad \qquad \left. + \left(\left(\frac{128495
            \delta }{1008} -\frac{12733 \delta \eta }{144}\right)
        \chi_a +\frac{40 \pi \eta }{3}
        -\frac{80 \pi }{3}\right) \chi_s -\frac{80 \pi  \delta \chi_a}{3} \right] \\
    %
    & \qquad\qquad\quad + v^7 \left[\frac{91495 \pi \eta ^2}{1512}
      +\frac{358675 \pi  \eta }{6048}-\frac{4415 \pi }{4032} \right. \\
    & \qquad\qquad\qquad \qquad \left.  + \left(-\frac{11 \eta ^2}{24}
        +\frac{979 \eta }{24}-\frac{505}{8} \right) \chi_s^3 +
      \left(\frac{\delta \eta ^2}{8}+\frac{742 \delta \eta }{3}
        -\frac{505 \delta }{8} \right) \chi_a^3 \right.\\
    & \qquad\qquad\qquad \qquad \left. + \left(\left(\frac{3 \eta
            ^2}{8}+\frac{917 \eta }{12}
          -\frac{1515}{8}\right) \delta \chi_a + 12 \pi \right) \chi_s^2 \right. \\
    & \qquad\qquad\qquad \qquad \left.  + \left( \left(-124 \delta ^2
          -\frac{3397 \eta ^2}{24}+\frac{7007 \eta}{24}
          -\frac{523}{8}\right) \chi_s -48 \pi  \eta +12 \pi \right) \chi_a^2 \right. \\
    & \qquad\qquad\qquad \qquad \left. + \left(\frac{2045 \eta
          ^3}{216} -\frac{398017 \eta ^2}{2016} +\frac{10772921
          \eta}{54432} -\frac{2529407}{27216}\right) \chi_s
      + 24 \pi  \delta  \chi_a \chi_s \right. \\
    & \qquad\qquad\qquad \qquad \left. \left. + \left(-\frac{41551
            \delta \eta ^2}{864} +\frac{845827 \delta
            \eta}{6048}-\frac{2529407 \delta }{27216}\right) \chi_a
      \right] \right\}~.
  \end{split}
\end{equation}
}%
Note that this expression does not include \emph{all} of the
spin-dependent terms at 3PN and 3.5PN, since the spin terms in the
energy and flux functions are known only up to 2PN and 2.5PN,
respectively.  However, the 3PN and 3.5PN terms shown here will still
be present in this formula when the higher-order terms are included in
the energy and flux formulas.


\subsubsection{TaylorT2 phasing}

Expanding the inverse of Eq.~\eqref{eq:BalanceEquation} allows for
the analytical integration of $t(v)$. The result reads
{\footnotesize
\begin{align} \label{eq:T2time}
\begin{split}
 t(v) &=  t_0 -  \frac{5M}{256 \eta \, v^8} \Bigg \{ 1 + v^2 \left[
\frac{11 \eta }{3}+\frac{743}{252} \right] + 
v^3 \left[ -\frac{32 \pi}{5} + \frac{226 \delta  \chi_a}{15} + \left(
 \frac{226}{15} -\frac{152 \eta}{15} \right) \chi_s
\right] \\
& \quad + v^4 \left[ 
\frac{3058673}{508032} + \frac{5429 \eta }{504} + \frac{617 \eta
^2}{72} -\frac{81}{4} \delta  \chi _a \chi _s
- \left(\frac{81}{8} - \frac{\eta}{2} \right) \chi_s^2 -
\left(\frac{81}{8} - 40 \eta \right)\chi _a^2 \right] \\
& \quad + v^5 \left[
-\frac{7729\pi }{252} - \frac{13 \pi  \eta }{3} + 
\left( \frac{147101}{756} -\frac{4906 \eta }{27}-\frac{68 \eta
^2}{3}\right) \chi_s
+ \left( \frac{147101
}{756} + \frac{26 \eta }{3} \right) \delta \chi_a \right. \\
& \qquad +  (6- 6\eta) \delta \chi_s^2 \chi_a + (6 - 18 \eta)
\chi_s \chi_a^2 + (2- 6 \eta) \chi_s^3 + (2 - 2\eta) \delta \chi_a^3 
\Bigg] \\
& \quad + v^6 \left[  \frac{6848\gamma}{105} -
\frac{10052469856691}{23471078400} +\frac{128 \pi ^2}{3}
+\left(\frac{3147553127 }{3048192} -\frac{451 \pi ^2}{12}
\right) \eta -\frac{15211 \eta ^2}{1728} + \frac{25565 \eta ^3}{1296} 
\right. \\
& \qquad  + \frac{6848 \ln (4v)}{105} - \left( \frac{584 \pi }{3} -
\frac{448 \pi  \eta }{3} \right) \chi_s -\frac{584 \pi  \delta \,
\chi_a }{3} + \left( \frac{6845}{672} -\frac{43427 \eta }{168} +
\frac{245 \eta ^2}{3} \right) \chi_s^2 \\
& \qquad + \left. \left(  \frac{6845}{672} -\frac{1541 \eta }{12} +
\frac{964 \eta ^2}{3} \right) \chi_a^2 
+ \left( \frac{6845}{336}-\frac{2077 \eta }{6} \right) \delta \,
\chi_s \chi_a
\right] \\
& \quad + v^7 \left[  -\frac{15419335 \pi }{127008} -\frac{75703 \pi 
\eta }{756} + \frac{14809 \pi  \eta ^2}{378} 
+ \left(\frac{4074790483}{1524096} +\frac{30187 \eta }{112}
-\frac{115739 \eta ^2}{216} \right) \delta \, \chi_a
\right. \\
& \qquad + \left( 
\frac{4074790483}{1524096} -\frac{869712071 \eta }{381024}
-\frac{2237903 \eta ^2}{1512} + \frac{14341 \eta ^3}{54}
\right) \chi_s + \left( 228 \pi -16 \pi  \eta \right) \chi_s^2 \\
& \qquad + \left( 228 \pi -896 \pi  \eta \right) \chi_a^2 + 456 \pi 
\delta \, \chi_s \chi_a - \left( \frac{3237}{14} - \frac{14929 \eta
}{84} +  \frac{362 \eta ^2}{3}\right) \chi_s^3 \\
& \qquad - \left( \frac{3237}{14} - \frac{87455 \eta }{84} + 34 \eta
^2 \right) \delta \chi_a^3 - \left( \frac{9711}{14} - \frac{39625 \eta
}{84} +  102 \eta ^2\right) \delta \, \chi_s^2 \chi_a \\
& \qquad \left . - \left(
\frac{9711}{14} -\frac{267527 \eta }{84} + \frac{3574 \eta
^2}{3} \right) \chi_s \chi_a^2 \right] \Bigg\}~.
\end{split}
\end{align}
}
The comment made below Eq.~\eqref{eq:dvByDt} about spin
contributions at 3PN and 3.5PN order is valid for Eq.~\eqref{eq:T2time}
and the following expansions as well.

The orbital phase $\Phi$ can be integrated similarly to the time $t$. 
Eq.~\eqref{eq:PhaseFormula} and Eq.~\eqref{eq:BalanceEquation} yield
\begin{equation}
 \frac{d\Phi}{dv} = \frac{v^{3}}{M} \, \frac{dt}{dv} = -
\frac{v^{3}}{M} \, \frac{E'(v)}{\Flux(v) + \dot{M}(v)} ~,
\end{equation}
which, after re-expanding in a Taylor series, can be integrated
analytically. The final result reads
{\footnotesize
{ \allowdisplaybreaks
\begin{align} \label{eq:T2phase}
 \Phi (v) &= \Phi_0 - \frac{1}{32 \eta \, v^5} \Bigg\{
1+ v^2 \left[ \frac{3715}{1008} + \frac{55 \eta }{12} \right] 
+ v^3 \left[ -10 \pi  + \frac{565 \delta  \chi _a}{24} + \left(
 \frac{565}{24} -\frac{95 \eta }{6} \right) \chi_s
\right] \nonumber \\
& \quad + v^4 \left[ \frac{15293365}{1016064} +\frac{27145 \eta
}{1008} + \frac{3085 \eta ^2}{144} -\frac{405 }{8} \delta \,
\chi_a \chi_s - \left( \frac{405}{16} - \frac{5 \eta }{4} \right)
\chi_s^2 - \left( \frac{405}{16} - 100 \eta \right) \chi_a^2 \right]
\nonumber \\
& \quad + v^5 \ln v \left[  \frac{38645 \pi }{672}-\frac{65 \pi  \eta
}{8} - \left( \frac{735505}{2016} - \frac{12265 \eta }{36} - \frac{85
\eta ^2}{2} \right) \chi_s - \left( \frac{735505}{2016}+\frac{65 \eta
}{4} \right) \delta \chi_a  \right. \nonumber \\
& \qquad \left. - \left( \frac{45}{4}-\frac{45 \eta }{4} \right)
\delta \chi_s^2 \chi_a - \left( \frac{45}{4}-\frac{135 \eta }{4}
\right) \chi_s \chi_a^2 - \left( \frac{15}{4}-\frac{45 \eta }{4}
\right) \chi_s^3 - \left( \frac{15}{4}-\frac{15 \eta }{4} \right)
\delta \chi_a^3 \right]  \\
& \quad + v^6 \left[ \frac{12348611926451}{18776862720}-\frac{1712
\gamma}{21}-\frac{160 \pi ^2}{3}-\left( \frac{15737765635 
}{12192768} - \frac{2255 \pi ^2}{48} \right) \eta +\frac{76055 \eta
^2}{6912}-\frac{127825 \eta ^3}{5184} \right. \nonumber \\
& \qquad -\frac{1712 \ln{(4v)}}{21} + \left(\frac{730 \pi
}{3}-\frac{560 \pi  \eta }{3} \right) \chi_s + \frac{730 \pi \delta
\chi_a }{3} -
\left( \frac{34225}{2688}-\frac{217135 \eta }{672}+\frac{1225 \eta
^2}{12} \right) \chi_s^2  \nonumber \\
& \qquad - \left. \left( \frac{34225}{2688}-\frac{7705 \eta
}{48}+\frac{1205 \eta ^2}{3} \right) \chi_a^2 - \left(
\frac{34225}{1344}-\frac{10385 \eta }{24} \right) \delta \, \chi_s
\chi_a \right] \nonumber \\
& \quad + v^7 \left[ \frac{77096675 \pi }{2032128}+\frac{378515 \pi 
\eta }{12096}-\frac{74045 \pi  \eta ^2}{6048} - \left( 
\frac{20373952415}{24385536}+\frac{150935 \eta }{1792}-\frac{578695
\eta ^2}{3456}\right) \delta \chi_a
\right. \nonumber \\
& \qquad - \left( \frac{20373952415}{24385536}-\frac{4348560355 \eta
}{6096384}-\frac{11189515 \eta ^2}{24192}+\frac{71705 \eta ^3}{864}
\right) \chi_s - \left( \frac{285 \pi }{4}-5 \pi  \eta \right)
\chi_s^2 \nonumber \\
& \qquad - \left( \frac{285 \pi }{4}-280 \pi  \eta \right) \chi_a^2 -
\frac{285 \pi }{2} \delta \, \chi_s \chi_a + \left(
\frac{16185}{224}-\frac{74645 \eta }{1344}+\frac{905 \eta ^2}{24}
\right) \chi_s^3 \nonumber \\
& \qquad + \left( \frac{16185}{224}-\frac{437275 \eta }{1344}+\frac{85
\eta ^2}{8} \right) \delta \chi_a^3 + \left(
\frac{48555}{224}-\frac{198125 \eta }{1344}+\frac{255 \eta ^2}{8}
\right) \delta \, \chi_s^2 \chi_a \nonumber \\
& \qquad + \left. \left(  \frac{48555}{224}-\frac{1337635 \eta
}{1344}+\frac{8935 \eta ^2}{24} \right) \chi_s \chi_a^2 \right]
\Bigg\}. \nonumber
\end{align}}}
Eq.~\eqref{eq:T2time} and Eq.~\eqref{eq:T2phase} together define
$\Phi(t)$ implicitly.


\subsubsection{TaylorF2 phasing}

Starting from the explicit expressions for time and orbital phase in
the TaylorT2 approximant, it is possible to analytically construct the
Fourier transform of the GW strain in the
framework of the stationary phase approximation (SPA)
\cite{Damour:2000zb,Damour:2002kr,Arun:2004hn}. Denoting the Fourier
transform of the strain by $\tilde A_{\ell m} e^{i
\psi_{\ell m}}$, the phase in the frequency domain can be approximated
by
\begin{equation}
 \psi_{\ell m}(f) = 2 \pi f \, t_f - m \Phi(t_f) - \frac{\pi}{4}. 
\end{equation}
Here, $f$ is the Fourier variable and $t_f$ corresponds to the time
when the instantaneous GW frequency coincides with $f$, i.e.,
\begin{equation}
 \frac{d (m \Phi)}{dt} (t_f) = 2 \pi f \qquad \Rightarrow \quad
v(t_f) = \left( \frac{2 \pi M f}{m} \right)^{1/3}. \label{eq:vfreq}
\end{equation}
%
The form of the Taylor series of $\psi_{\ell m}$ obviously depends on
the spherical harmonic mode's $m$. For the sake of brevity, only the
expansion
for $m=2$ is given below.
{\footnotesize
{\allowdisplaybreaks
\begin{align}
 \psi_{\ell 2} (v) &= 2 t_0 v^3 - 2\Phi_0 - \frac{\pi}{4} +
\frac{3}{128 \eta \, v^5} \Bigg\{ 1 + v^2
\left[\frac{3715}{756}+\frac{55 \eta }{9}  \right] + v^3 \left[ 
 -16 \pi +\left( \frac{113}{3}-\frac{76 \eta }{3} \right) \chi_s
+\frac{113 \delta  \chi _a}{3} \nonumber \right] \\
& \quad + v^4 \left[ \frac{15293365}{508032}+\frac{27145 \eta
}{504}+\frac{3085 \eta ^2}{72} - \left( \frac{405}{8}-\frac{5 \eta
}{2} \right) \chi_s^2 - \left( \frac{405}{8}-200 \eta \right) \chi_a^2
-\frac{405  }{4} \delta \, \chi_s \chi_a \right] \nonumber \\
& \quad + v^5 \left( 1+ 3  \ln v \right) \left[ \frac{38645
\pi }{756}-\frac{65 \pi  \eta }{9} - \left(
\frac{735505}{2268}-\frac{24530 \eta }{81}-\frac{340 \eta ^2}{9}
\right) \chi_s - \left( \frac{735505}{2268}+\frac{130 \eta }{9}
\right) \delta \chi_a \nonumber \right. \\
& \qquad - \left. (10-10 \eta) \delta \, \chi_s^2 \chi_a - (10-30
\eta) \chi_s \chi_a^2 - \left( \frac{10}{3}-10 \eta \right) \chi_s^3 -
\left( \frac{10}{3}-\frac{10 \eta }{3} \right) \delta \chi_a^3
\right] \nonumber \\
& \quad + v^6 \left[  \frac{11583231236531}{4694215680}-\frac{6848
\gamma}{21}-\frac{640 \pi ^2}{3}-\left(
\frac{15737765635}{3048192}-\frac{2255 \pi ^2}{12} \right) \eta
+ \frac{76055 \eta ^2}{1728} - \frac{127825 \eta ^3}{1296} 
\right. \nonumber \\
& \qquad -\frac{6848  \ln(4v)}{21} + \left( \frac{2920 \pi
}{3}-\frac{2240 \pi  \eta }{3} \right) \chi_s + \frac{2920 \pi 
}{3}\delta \chi_a - \left( \frac{34225}{672}-\frac{217135 \eta
}{168}+\frac{1225 \eta ^2}{3} \right) \chi_s^2 \nonumber \\
& \qquad - \left. \left( \frac{34225}{672}-\frac{7705 \eta
}{12}+\frac{4820 \eta ^2}{3} \right) \chi_a^2 
- \left( \frac{34225}{336}-\frac{10385 \eta }{6} \right) \delta \,
\chi_s \chi_a
\right] \nonumber \\
& \quad + v^7 \left[  \frac{77096675 \pi }{254016}+\frac{378515 \pi 
\eta }{1512}-\frac{74045 \pi  \eta ^2}{756} -
\left( \frac{20373952415}{3048192}+\frac{150935 \eta
}{224}-\frac{578695 \eta ^2}{432} \right) \delta \chi_a
\right. \nonumber \\
& \qquad - \left( \frac{20373952415}{3048192}-\frac{4348560355 \eta
}{762048}-\frac{11189515 \eta ^2}{3024}+\frac{71705 \eta ^3}{108} 
\right) \chi_s - (570 \pi -40 \pi  \eta) \chi_s^2 \nonumber \\
& \qquad - (570 \pi -2240 \pi  \eta) \chi_a^2 -1140 \pi  \delta \,
\chi_s \chi_a + \left( \frac{16185}{28}-\frac{74645 \eta
}{168}+\frac{905 \eta ^2}{3} \right) \chi_s^3 \nonumber \\
& \qquad + \left( \frac{16185}{28}-\frac{437275 \eta }{168}+85 \eta ^2
\right) \delta \chi_a^3 + \left( \frac{48555}{28}-\frac{198125 \eta
}{168}+255 \eta ^2 \right) \delta \, \chi_s^2 \chi_a \nonumber \\
& \qquad \left. + \left( \frac{48555}{28}-\frac{1337635 \eta
}{168}+\frac{8935 \eta ^2}{3} \right) \chi_s \chi_a^2  \right] \Bigg
\}.
\end{align}
}}%
%
According to Eq.~(\ref{eq:vfreq}), $v$ should be understood as $v =
(M \pi f)^{1/3}$ in the equation above.

\subsection{Waveform amplitudes}
\label{sec:WaveformAmplitudes}
Now, given the orbital phase $\Phi$ and the related post-Newtonian
expansion parameter $v$ defined in Eq.~\eqref{eq:DefinevParameter}, we
can obtain the waveform observed at infinity.  Currently, the most complete
expressions for the nonspinning parts of the waveform are
found in~\cite{Blanchet:2008}.  In particular, Eqs.~(9.3) and~(9.4) of
that reference give the decomposition of $h_+ - ih_\times$ into
harmonics.  Due to space
considerations and the danger of transcription errors, we do not
reproduce those equations here, but simply refer the reader to that
paper.  To these, we must add\footnote{Note that
  Refs.~\cite{Blanchet:2008} and~\cite{WillWiseman:1996} share the
  notation set forth in~\cite{Blanchet:1996}, so that we can simply
  add the relevant terms.  Also note that each of those references
  uses a different normalization for the variable $H$ compared to the
  one used here.}  the spin terms given most completely
in~\cite{WillWiseman:1996}.  There, the spin terms were not explicitly
decomposed into harmonics, however, using Eq.~(9.2)
of~\cite{Blanchet:2008}, it is a simple matter to deduce them.  Using
Eqs.~(F24) and (F25) of~\cite{WillWiseman:1996}, and noting the
overall sign error in Eq.~(F25c), we obtain the only nonzero spin
contributions to the harmonics:
\begin{align}
  \label{eq:SpinAmplitudeTerms}
  H_{2,2} &= -\frac{16}{3} \sqrt{\frac{\pi}{5}} v^5 \eta \left[2 \delta
    \chi_{a} + 2(1-\eta) \chi_{s} + 3 v \eta \left(\chi_{a}^2 -
      \chi_{s}^2 \right) \right] e^{-2 i \Phi}~, \\
  H_{2,1} &= 4 i \sqrt{\frac{\pi}{5}} v^4 \eta (\delta \chi_{s} +
  \chi_{a}) e^{-i \Phi}~, \\
  H_{3,2} &= \frac{32}{3} \sqrt{\frac{\pi}{7}} v^5 \eta^2 \chi_{s}
  e^{-2 i \Phi}~.
\end{align}
In all cases, modes with negative values of $m$ can be obtained from
\begin{equation}
  \label{eq:NegativeMModes}
  H_{\ell, -m} = (-1)^{\ell}\, \bar{H}_{\ell,m}~.
\end{equation}

The appropriate SPA amplitude in Fourier space can easily be deduced
from its time-domain description $A_{\ell m}$ by
\begin{equation}
 \tilde A_{\ell m} = A_{\ell m} \sqrt{\frac{2\pi}{m \ddot \Phi}} =
A_{\ell m} \sqrt{\frac{2 \pi M}{3m v^2 \, \dot v}}~,
\end{equation}
where $\dot v$ can be taken for instance from Eq.~(\ref{eq:dvByDt})
and all arguments should be replaced according to
Eq.~(\ref{eq:vfreq}).


%%% Local Variables: 
%%% mode: latex
%%% TeX-master: "NRDataFormat"
%%% End: 
